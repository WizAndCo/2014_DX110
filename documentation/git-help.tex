\documentclass[10pt,a4paper]{report}
\usepackage[latin1]{inputenc}
\usepackage{amsmath}
\usepackage{amsfonts}
\usepackage{amssymb}
\usepackage{hyperref} %pr inserer des liens internet

\usepackage{verbatim}%pour linsertion brute de commande LaTeX dans le texte
\usepackage{moreverb}

\usepackage{graphicx} %pour linsertion dimages

%Lignes ajout\'ees pour les documents en fran\c{c}ais, pour que soit prises en compte les particularit\'es de la typographie fran\c{c}aise; c'est aussi grâce \`{a} ce package que la date est affich\'ee automatiquement en fran\c{c}ais, que le titre de la table des mati\`eres est « Table des mati\`eres » et non « Contents », etc. Les deux lignes suivantes permettent d'avoir des caract\`eres correctement accentu\'es en toutes circonstances.
\usepackage[french]{babel}
\usepackage[latin1]{inputenc}


\usepackage{listings} %pr inserer du code
\usepackage{xcolor}
% pour une jolie insertion de code (https://www.sharelatex.com/project/51b8f1c34c2bd70430a90c23)
\lstset{
    backgroundcolor=\color{black!5}, % set backgroundcolor
    basicstyle=\footnotesize,% basic font setting
    language=C++,
    %basicstyle=\ttfamily\small,
    breaklines=true,
    prebreak=\raisebox{0ex}[0ex][0ex]{\ensuremath{\hookleftarrow}},
    frame=lines,
    showtabs=false,
    showspaces=false,
    showstringspaces=false,
    keywordstyle=\color{red}\bfseries,
    stringstyle=\color{green!50!black},
    commentstyle=\color{gray}\itshape,
    numbers=left,
    captionpos=t,
    escapeinside={\%*}{*)}
}

%Pour faire un index
\usepackage{makeidx}

%Pour faire un index 
\makeindex

\title{GIT HELP}
\author{AdKoba}


\begin{document}

\section{Les branches}

Les branches sont tr�s utiles pour travailler sur des stories diff�rente. L'id�e est de cr�er une branche par story et d'y effectuer tout le travail relatif � celle-ci sans impacter directement la branche master. 

Pour cr�er une branche il faut proc�der comme suit:
$git branch ``branch name''
$git checkout ``branch name''

ou directement:
$git checkout -b ``branch name''
Switched to a new branch ``branch name''

A partir de cet instant notre HEAD pointe sur cette nouvelle branche. Tout les modifications qui y seront r�alis�es le seront uniquement dans celle-ci

On peut ainsi cr�er plusieurs branches en parall�le pour travailler sur des diff�rents sujets.

Quand le travail sur une branche est termin�, il faut commiter les modifications puis merger la branche avec la branche master pour pouvoir les publier.

On commite
$ git commit -a -m 'commit message'
On rebascule sur la branche master
$ git checkout master
Et on merge
$ get merge ma-branche
Il ne reste plus qu'� publier les modifications
$ git pull

%l index sera ecrit ici
\printindex

\end{document}
