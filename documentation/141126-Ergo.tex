%\documentclass[10pt,a4paper]{article}
\documentclass[10pt,a4paper]{book}
\usepackage[utf8]{inputenc}
\usepackage{amsmath}
\usepackage{amsfonts}
\usepackage{amssymb}




%----------------------- Mes packages--------------------
\usepackage{hyperref} %pr inserer des liens internet


%---------- Insertion de code
\usepackage{listings} %pr inserer du code


%---------- Pour la definition de mes couleurs (pompees sur http://eclipsecolorthemes.org/?view=theme&id=1)
\usepackage[usenames,dvipsnames]{color}%cf: http://en.wikibooks.org/wiki/LaTeX/Colors
\definecolor{graun}{RGB}{30,30,30}	%couleur de fond des codes inlines
\definecolor{commentaire}{RGB}{199,221,12}%comment
\definecolor{chaine}{RGB}{255,198,0}% string
\definecolor{stlNrml}{RGB}{255,255,255} %style normal
\definecolor{lettrebleue}{RGB}{141,203,226}%pour les mot cles (for, return, int)
\definecolor{functionOgre}{RGB}{246,84,106}%utiliser par la liste de mots particuliers a colorer (cf plus bas)

%\definecolor{chaineCte}{RGB}{239,192,144}
%\definecolor{functionClr}{RGB}{210,82,82}

\definecolor{indianRed1}{RGB}{255,106,106}%utilise non pas pour le code mais pour des box warning
\definecolor{pastelBlue}{RGB}{109,207,246}%utilise non pas pour le code mais pour des box note/memento
\definecolor{pastelGreen}{RGB}{224,243,176}%utilise non pas pour le code mais pour des box note
%\definecolor{}{RGB}{}	% 	

%-----------------------------------------------------------------------------------------------------
%---------- Insertion de code
\usepackage{listings} %pr inserer du code
\usepackage{xcolor}




% pour une jolie insertion de code (https://www.sharelatex.com/project/51b8f1c34c2bd70430a90c23)
\lstset{
    backgroundcolor=\color{graun}, % set backgroundcolor
    basicstyle=\ttfamily,% basic font setting 
    language=C++,   
    breaklines=true,
    prebreak=\raisebox{0ex}[0ex][0ex]{\ensuremath{\hookleftarrow}},%displaying mark on the end of breaking line    
    frame=lines,
    showtabs=false,
    showspaces=false,
    showstringspaces=false,    
    keywordstyle=\color{lettrebleue}\bfseries,
    stringstyle=\color{chaine},
    commentstyle=\color{commentaire}\itshape,
    basicstyle= \color{stlNrml},
    numbers=left,
    captionpos=t,
    escapeinside={\%*}{*)},
    %mathescape,	%permet l utilisation de fonction math dans les listinge de code (comme les indices, ...)
}


%les methodes de l'API Ogre
\lstset{emph={%  
    Ogre, Entity, createChildSceneNode, getRootSceneNode%
    },emphstyle={\color{functionOgre}\bfseries}%
}



%-----------------------------------------------------------------------------------------------------
%----------utilisation du bloc comment
%pour pouvoir utiliser le bloc \begin{comment}...\end{comment}
%pour linsertion brute de commande LaTeX dans le texte
\usepackage{verbatim}	


%-----------------------------------------------------------------------------------------------------
%----------Creation de boites pour les notes
%pour faire de jolies boite latérales contenant des notes
\usepackage[verbose]{wrapfig}
\usepackage{bclogo}
\usepackage{environ}
\usepackage{lipsum}

\NewEnviron{mybox}[1]
  {\wrapfigure{r}{.5\textwidth}
  \setlength\fboxrule{1.5pt}
  \fcolorbox{blue!70}{white}{%
  \begin{minipage}{\dimexpr\linewidth-2\fboxsep-2\fboxrule\relax}
  \parbox[t][1cm][t]{1cm}{\bccrayon}%
  \parbox[t][1cm][t]{\dimexpr\linewidth-1cm\relax}{\bfseries#1}
  \BODY
  \end{minipage}}%
\endwrapfigure}

%pour faire de jolies boites contenant des notes
\usepackage{tikz}
  \usetikzlibrary{shapes,shadows}
  \tikzstyle{abstractbox} = [draw=black, fill=white, rectangle, 
  inner sep=10pt, style=rounded corners, drop shadow={fill=black,
  opacity=1}]
  \tikzstyle{abstracttitle} =[fill=white]
 
  \newcommand{\boxabstract}[2][fill=white]{
    \begin{center}
      \begin{tikzpicture}
        \node [abstractbox, #1] (box)
        {\begin{minipage}{0.80\linewidth}
%            \setlength{\parindent}{2mm}
            \footnotesize #2
          \end{minipage}};
        \node[abstracttitle, right=10pt] at (box.north west) {Abstract};
      \end{tikzpicture}
    \end{center}
  }
  
%pour faire de jolies boites
\usepackage{color}
%\usepackage{lipsum}
\newcommand\tipbox[1]{%
  \begin{center}%
    \fcolorbox{black}{white}{%
      \begin{minipage}[t]{\dimexpr\textwidth-2\fboxsep-2\fboxrule}%
        #1%
      \end{minipage}}%
  \end{center}}
  
  


%pour faire de jolies boites
\usepackage{color}
%\usepackage{lipsum}
\newcommand\tipboxnote[1]{%
  \begin{center}%
    \fcolorbox{black}{pastelGreen}{%Pour signaler les notes
      \begin{minipage}[t]{\dimexpr\textwidth-2\fboxsep-2\fboxrule}%
        #1%
      \end{minipage}}%
  \end{center}}
  
\newcommand\tipboxWarning[1]{%Pour signaler les points dangereux
  \begin{center}%
    \fcolorbox{black}{indianRed1}{%
      \begin{minipage}[t]{\dimexpr\textwidth-2\fboxsep-2\fboxrule}%
        #1%
      \end{minipage}}%
  \end{center}} 
  
\newcommand\tipboxMemento[1]{%Pour signaler les points dont je dois me souvenir
  \begin{center}%
    \fcolorbox{black}{pastelBlue}{%
      \begin{minipage}[t]{\dimexpr\textwidth-2\fboxsep-2\fboxrule}%
        #1%
      \end{minipage}}%
  \end{center}}



%-----------------------------------------------------------------------------------------------------
%----------pour inserer des images
\usepackage{graphicx}


%-----------------------------------------------------------------------------------------------------
%----------pour faire un index
%\usepackage{makeidx}
%\makeindex
%----------pour faire un index et qu'il soit inserer dans la table des matieres
\usepackage{imakeidx}
\makeindex[columns=3, title=Index, intoc]

%----------------------------------------------------------

\title{OGRE}
\author{Adkoba \& Aklal}



\begin{document}


%------------------ affichage du titre
\maketitle
\newpage
%------------------ affichage de la table de matières
\tableofcontents
%------------------ affichage de la liste des codes
\lstlistoflistings%pour avoir la liste des codes
\newpage






\part*{Avant-Propos}
\section*{Pr\'esentation du document}
This document is a re-lecture of the tutorial \href{http://fr.openclassrooms.com/informatique/cours/decouvrez-ogre-3d}{''D\'ecouvrez Ogre 3D''}. The original tutorial was written by Julien Chichignoud on \href{fr.openclassrooms.com}{Openclassroom} (aka ''Le site des z\'eros'').\newline

Writing this document is an attempt to make something like an active reading, and thus have a more attentive lecture and deepest understanding of Ogre's concepts. We modified a lot (sentences, structure, ...) but the essence is the same as the original tutorial.\newline

We wish to highligh the fact that this document is a personal interpretation, and a in progress work.\newline

The original tutorial is in French and as we are spending most of our strength in the comprehension of Ogre we did not take time to translate the original tutorial in French.\newline


\section*{Licence}
The original tutorial is released under the 
\href{http://creativecommons.org/licenses/by-nc-sa/2.0/}{Creative Common licence} and so is also released this document.\newline

The rest of the document will be now in French, sorry for those who cannot take advantage of the document.


\part{Bases de Ogre}

%---------------------------------------------------------------------------------------------------------------

%!!! 3OGRE3 n est pas centre dans le titre qui suit
\chapter{Installation, Premier programme}


\section{Premier pas}
\subsection{Sources}
J'ai d'abord voulu suivre le tutoriel du site du z\'ero
\url{http://fr.openclassrooms.com/informatique/cours/decouvrez-ogre-3d/}\newline

Mais ce tutoriel du site des zéros semble plus axé visual studio, par contre un lien est donné pour la compilation de projets sous Ubuntu \url{http://geenux.wordpress.com/2010/03/18/installation-de-ogre-1-7-et-compilation-avec-cmake-sous-ubuntu/}


\subsection{Installation}
Installation avec synaptic des packages suivants:
\begin{enumerate}
\item libogre-dev
\item ogre-samples
\item ogre-samples-data
\item libogre-1.7.4
\item libois-1.3.0
\item libois-dev 
\end{enumerate}






\section{Premiere compilation}

\subsection{Premier programme}

\begin{itemize}
\item Création d'un répertoire contenant les fichiers suivants:
\end{itemize}


\begin{lstlisting}
Hraesvelg:~/Documents/workspace/3D/OGRE/ExampleApplication> ls -l
total 24
-rw-rw-r-- 1 oliver oliver 1109 fvr. 23 12:16 CMakeLists.txt
-rw-rw-r-- 1 oliver oliver  844 fvr. 23 12:02 helloworld.cpp
-rw-rw-r-- 1 oliver oliver 4894 fvr. 23 12:18 Makefile
-rw-r--r-- 1 oliver oliver  446 fvr. 23 12:20 plugins.cfg
-rw-r--r-- 1 oliver oliver 1391 fvr. 23 12:20 resources.cfg
\end{lstlisting}




\begin{itemize}
\item helloworld.cpp:
\end{itemize}


\begin{lstlisting}
\#include ''ExampleApplication.h''
 
class TutorialApplication: public ExampleApplication
{
protected:
public:
    TutorialApplication()
    {
    }
 
    ~TutorialApplication()
    {
    }
protected:
    void createscene(void)
    {
    }
};
 
#if OGRE_PLATFORM == OGRE_PLATFORM_WIN32
#define WIN32_LEAN_AND_MEAN
#include ''windows.h''
 
INT WINAPI WinMain( HINSTANCE hInst, HINSTANCE, LPSTR strCmdLine, INT )
#else
int main(int argc, char **argv)
#endif
{
    // Create application object
    TutorialApplication app;
 
    try {
        app.go();
    } catch( Exception& e ) {
#if OGRE_PLATFORM == OGRE_PLATFORM_WIN32
        MessageBox( NULL, e.what(), ''An exception has occurred!'', MB_OK | MB_ICONERROR | MB_TASKMODAL);
#else
        fprintf(stderr, ''An exception has occurred: %s\n'',
                e.what());
#endif
    }
 
    return 0;
}
\end{lstlisting}


\begin{itemize}
\item CMakeLists.txt est une copie modifiée du CMakeLists.txt donnée dans le tutoriel de compilation sous Ubuntu. Les lignes commencant par \# \~ sont les lignes originales que j'ai du modifier.
\end{itemize}



\begin{lstlisting}
project(helloworld)
cmake_minimum_required(VERSION 2.6)

\# \~ set(CMAKE_MODULE_PATH ''/usr/lib/OGRE/cmake/'')
set(CMAKE_MODULE_PATH ''/usr/share/OGRE/cmake/modules'')

#set(CMAKE_CXX_FLAGS ''-Wall -W -Werror -ansi -pedantic -g'')

# Il s sagit du tutoriel d exemple, qui utilise quelques fichiers predefini de Ogre. Il faut indiquer a cmake o\'{u} se trouvent les includes en question
# \~ include_directories (''/usr/share/OGRE/Samples/Common/include/'')
include_directories (''/usr/share/OGRE-1.7.4/Samples/Common/include/'')

# Bien sur, pour compiler Ogre, il faut le chercher, et definir le repertoire contenant les includes.
find_package(OGRE REQUIRED)
include_directories (${OGRE_INCLUDE_DIRS})

\# L'exemple depend aussi de OIS, une lib pour gerer la souris, clavier, joystick...
find_package(OIS REQUIRED)

\# On definit les sources qu'on veut compiler
SET(SOURCES
helloworld.cpp)

\# On les compile
add_executable (
helloworld ${SOURCES}
)

\# Et pour finir, on lie l'executable avec les librairies que find_package nous a gentillement trouve.
target_link_libraries(helloworld ${OGRE_LIBRARY} ${OIS_LIBRARY})
%\end{lstlisting}


plugins.cfg et resources.cfg sont copiés de /usr/share/OGRE-1.7.4

\subsection{Compilation}
les commandes \`{a} faire sont:
\begin{itemize}
\item cmake.
\item make
\item./helloworld
\end{itemize}

\section{Premiere compilation sous CodeBlocks}
A tenter!!!!\newline
\url{http://fr.openclassrooms.com/informatique/cours/decouvrez-ogre-3d/configuration-d-un-projet-3}

\section{Premiere compilation sous kDevelop}
J'ai suivi\newline
\url{http://www.ogre3d.org/tikiwiki/tiki-index.php?page=Setting+Up+An+Application+-+KDevelop+-+Linux}.\newline
Je copie dans le répertoire ogretest le répertoire d'exemple Sample\_Water. Le processus foire.




\section{Code de base}
\subsection{Source}
\url{http://fr.openclassrooms.com/informatique/cours/decouvrez-ogre-3d/le-code-de-base-1}

Ce code de base est en fait le même que celui fait dans le chapitre Premiere Compilation, sauf qu'alors on avait que un seul fichier cpp, on rajoute i\c{c}i un fichier.h qui est inclus dans le fichier main.cpp.




%---------------------------------------------------------------------------------------------------------------
\chapter{Ins\'erer des objets dans la sc\`ene }

\section{les entit\'es}

\subsection{Une entit\'e c'est quoi}
Une entit\'e est l'ensemble des informations attach\'ees aux polygones constituant un objet 3D.\newline
Un Mesh \'etant l'ensemble des polygones constituant un objet 3D, une entit\'e est l'ensemble des informations attach\'ees \`{a} un Mesh:

\begin{itemize}
\item les vertices (sommets des polygones),
\item les textures,
\item un squelette si le modele est sujet \`{a} des animations
\item...\newline
\end{itemize}


Tous les objets solides qui apparaissent \`{a} l'\'ecran sont donc des entit\'es et sont repr\'esent\'es par une seule et m\^eme classe dans Ogre: la classe Entity.

\subsection{Le sceneManager}
C est au sceneManager que revient la t\^{a}che de cr\'eer tous les objets que peut contenir la sc\`ene (tous les mod\`eles, lumi\`eres, cam\'era et autres objets) et de nous permettre ensuite d'y acc\'eder.\newline
Par cons\'equent, l'insertion d'objets dans la sc\`ene passe toujours par lui (ou par l'un des \'el\'ements d\'ej\`{a} ins\'er\'es) et par les m\'ethodes qu'il propose pour cela.
Il en existe diverses variantes en fonction de la sc\`ene que l'on veut r\'ealiser; selon que celle-ci sera par exemple en int\'erieur ou en ext\'erieur, par exemple.

Toutes les applications Ogre doivent donc avoir un sceneManager pour pouvoir fonctionner, puisque c'est lui qui s'occupe de tout ! La classe ExampleApplication ne fait pas exception et poss\`ede donc un attribut msceneMgr, qui est un pointeur sur le sceneManager de l'application et qui nous permettra dans les parties suivantes d'agr\'ementer notre sc\`ene avec des objets.


\subsection{Cr\'eer une entit\'e}

L'entit\'e doit \^etre cr\'e\'ee par le sceneManager pour pouvoir \^etre ajout\'ee \`{a} la sc\`ene. La m\'ethode createEntity() permet de faire cela.
\begin{lstlisting}
Entity *head= msceneMgr->createEntity(''Tete'',''ogrehead.mesh'');
\end{lstlisting}\footnote{Pourquoi utilise t on -> pour appeler une m\'ethode d'une classe?}

\begin{itemize}
\item Le premier param\`etre est le nom que vous souhaitez donner au mesh. Ce nom doit \^etre unique!
\item Le second param\`etre est le nom du fichier que vous voulez charger. Notez l'extension.mesh, qui est le format de fichiers pour les mod\`eles reconnu par Ogre.
\end{itemize}




Le fichier sp\'ecifi\'e (avec le deuxi\`eme param\`etre) se trouve, accompagn\'e d'autres mod\`eles d'exemples, dans le dossier OgreSDK/media/models, qui doit \^etre correctement renseign\'e dans le fichier resources.cfg pour qu'Ogre puisse le trouver lors de l'ex\'ecution.\footnote{Mais \`a quoi sert ce fichier .mesh?}

Le mesh a \'et\'e ajout\'e \`{a} la sc\`ene mais il nous manque encore une chose avant de pouvoir l'afficher \`{a} l'\'ecran...













\section{Les Noeuds de la sc\`ene}



\subsection{L'utilit\'e des Noeuds}

Dans Ogre, lorsque l'on souhaite manipuler une entit\'e (un personnage, une lumi\`ere, une cam\'era...), les d\'eplacements que l'on veut effectuer se font par l'interm\'ediaire d'un noeud de sc\`ene, ou sceneNode.

Un sceneNode est un objet invisible auquel on va pouvoir attacher un nombre ind\'efini d'entit\'es, lesquelles deviennent solidaires de ce noeud et subissent donc les m\^eme transformations que lui. C'est donc une sorte de conteneur qui contient les informations de positionnement de chacune des entit\'es de la sc\`ene qui lui sont rattach\'ees.

Bien s\^ur on pourrait d\'eplacer nos entit\'es directement mais avec un noeud on pourra d\'eplacer en une fois toutes les entit\'es attach\'ees au noeud.

Quoi qu'il en soit, attacher chaque entit\'e \`{a} un noeud est primordial, sans quoi elle ne s'affichera pas dans votre sc\`ene!



\subsection{Cr\'eer un noeud}

Pour cr\'eer un noeud de sc\`ene on devra passer par un noeud d\'ej\`{a} existant, ce qui va nous permettre d'avoir des relations d'h\'eritage entre nos noeuds.\newline

On utilisera une m\'ethode d\'edi\'ee:

\begin{lstlisting}
sceneNode *noeudEnfant = noeudParent->createChildsceneNode(''enfant'', Vector3::ZERO, Quaternion::IDENTITY);
\end{lstlisting}


Mais comment fait-on pour le premier noeud\index{noeud} qu'on va cr\'eer ? Le noeud ''racine'' existe d\`es que le sceneManager est cr\'e\'e, c'est un noeud comme un autre avec les m\^eme \footnote{''les m\^eme'' faut il un ''s'' \`{a} ''m\^eme'' ?} m\'ethodes mais ce noeud est unique.\newline

Nous r\'ecup\'erons ce noeud racine de la sc\`ene  \`{a} l'aide de l'instance du sceneManager:
\begin{lstlisting}
sceneNode *node= msceneMgr->getRootsceneNode()->createChildsceneNode(''nodeTete'', Vector3::ZERO, Quaternion::IDENTITY);
\end{lstlisting}

La m\'ethode getRootsceneNode() nous permet de r\'ecup\'erer un pointeur sur le noeud racine unique de la sc\`ene. On appelle ensuite sa m\'ethode createChildsceneNode pour lui ajouter un nouveau noeud fils.

Notez qu'aucun des param\`etres de la m\'ethode n'est obligatoire, pour information:
\begin{itemize}
\item Le premier argument est le nom que vous voulez donner \`{a} votre noeud. Sur le m\^eme principe que les entit\'es ce nom pourra \^etre utilis\'e pour r\'ecup\'erer un pointeur vers le noeud en question. 
\item Le deuxi\`eme argument est la position initiale du noeud.
\item Le troisi\`eme argument est le quaternion avec lequel vous voulez initialiser votre noeud.
\end{itemize}
	

Sachez pour le moment qu'un quaternion est un objet math\'ematique qui permet de faire faire aux objets des rotations dans l'espace\footnote{Que des rotations?}. Quaternion::IDENTITY \footnote{comment \'ecrire Quaternion::IDENTITY de manière \'el\'egante?}lui dit de ne pas faire de rotation.\newline

Avec ce code en main, vous pouvez simplement attacher l'entit\'e pr\'ec\'edemment cr\'e\'ee au noeud avec la ligne suivante:

\begin{lstlisting}
node->attachObject(head);
\end{lstlisting}

En compilant, vous devriez voir la t\^ete d'un ogre au milieu de l'\'ecran. Vous pouvez d\'eplacer la cam\'era avec Z, S, Q, D et la souris pour voir ce que \c{c}a donne de plus pr\`es.



\subsection{Code}

\subsubsection{PremiereApplication.cpp}
\begin{lstlisting}[caption={PremiereApplication.cpp: Instanciation d'entit\'e}]
#include "PremiereApplication.h"

void PremiereApplication::createScene()
{
    //creation d une entite
    Entity *head= mSceneMgr->createEntity("Tete", "ogrehead.mesh" );
    
    //creation d un noeud
    SceneNode *node= mSceneMgr->getRootSceneNode( )->createChildSceneNode( "nodeTete " , Vector3::ZERO, Quaternion::IDENTITY);
    
    node->yaw(Radian(Math::PI));
    node->yaw(Radian(Math::PI));

    Vector3 position = Vector3(30.0, 50.0, 0.0);
    node->setPosition(position);

    node->setPosition(30.0, 50.0, 0.0); 
    node->translate(-30.0, 50.0, 0.0); 
    
    
    //attachement de l entite au noeud
    node->attachObject ( head );
}
\end{lstlisting}








\section{Cr\'eer un mesh}

Nous pouvons rajouter un sol \`{a} notre sc\`ene, pour cela nous allons cr\'eer nous-m\^emes\footnote{il faut un ''s'' \`a  nous-m\^emes?} un nouveau mesh. \'Etant donn\'e que nous n'avons besoin que d'un plan pour le sol, le mesh peut tr\`es simplement \^etre cr\'eer dans le code de notre application.



\subsection{Le mesh}

Il existe une classe Plane\footnote{de m\^eme que pour quaternion.entity marquer les noms de classe de manière \'el\'egante pour les discerner du texte serait bien} qui va nous permettre de g\'en\'erer... un plan qui repr\'esentera le sol.\footnote{En plus de la classe Plane, vous trouverez aussi des classes Box (pour les cubes) et Sphere qui fonctionnent sur le m\^eme principe.}\newline

Pour cr\'eer un plan nous appelons Plane avec les param\`etres suivants:
\begin{itemize}
\item le premier param\`etre permet de d\'efinir le vecteur normal au plan \`{a} cr\'eer (ici l'axe Y pour que notre plan soit horizontal).
\item le second param\`etre est la distance \`{a} l'origine de la sc\`ene dans le sens du vecteur normal  (ici, je mets 0 pour que mon mesh plan soit centr\'e). 
\end{itemize}

\begin{lstlisting}
	Plane plan(Vector3::UNIT_Y, 0);
\end{lstlisting}

Une fois le plan cr\'e\'e, il faut que l'on cr\'ee un mesh, c'est-\`{a}-dire l'objet 3D en lui-m\^eme (la repr\'esentation du plan) qui sera visible dans la sc\`ene.
Pour cela, on utilise le Mesh Manager, qui va s'occuper de cr\'eer les faces de notre mesh.

\begin{lstlisting}
MeshManager::getSingleton().createPlane(''sol'', ResourceGroupManager::DEFAULT_RESOURCE_GROUP_NAME, plan, 500, 500, 1, 1, true, 1, 1, 1, Vector3::UNIT_Z);
\end{lstlisting}

Quelques explications sur cette ligne s'imposent. 
\begin{itemize}
\item Tout d'abord, la m\'ethode statique getSingleton() permet de r\'ecup\'erer un objet instanci\'e de fa\c{c}on unique, donc ici notre MeshManager.
\item Les deux premiers param\`etres correspondent respectivement au nom que l'on veut donner \`{a} notre mesh et au nom du groupe auquel on veut qu'il appartienne. 
\item Suivent ensuite le nom du plan \`{a} mod\'eliser, puis la largeur et la hauteur qu'il doit avoir, puis le nombre de subdivisions du plan dans ces deux sens. Plus il y a de subdivisions, plus il y a de polygones dans notre mesh. 
\item Le bool\'een suivant indique que les normales sont perpendiculaires au plan.
\item Les trois param\`etres suivants sont le nombre de textures que l'on va pouvoir assigner au plan, puis le nombre de fois que la texture sera r\'ep\'et\'ee dans les deux directions. 
\item le dernier param\`etre est le vecteur indiquant la direction du haut du mesh. Attention: il ne faut pas le confondre avec la normale du plan, qui est diff\'erente.
\end{itemize}

Il reste encore des param\`etres par d\'efaut que l'on verra plus tard.

Enfin, nous allons revenir vers un code connu: nous allons cr\'eer l'entit\'e qui repr\'esentera le plan. C'est le m\^eme principe que tout \`{a} l'heure:
\begin{itemize} 
\item tout d'abord, on cr\'ee une entit\'e \`{a} partir du sc\`ene Manager en la nommant et en lui indiquant le mesh \`{a} utiliser. 
\item On cr\'ee ensuite un nouveau noeud \`{a} partir du noeud racine et on l'attache \`{a} notre entit\'e.
\end{itemize}

\begin{lstlisting}
//creation d'une entite
Entity *ent= msceneMgr->createEntity(''EntiteSol'', ''sol'');

//creation d'un nouveau noeud
node = msceneMgr->getRootsceneNode()->createChildsceneNode();

//on attache le noeud a notre entite
node->attachObject(ent);
\end{lstlisting}




\subsection{Le mat\'eriau}

Nous allons finir en ajoutant une texture au sol: de l'herbe. Pour cela, il suffit de rajouter la ligne suivante apr\`es la cr\'eation de l'entit\'e:
\begin{lstlisting}
ent->setMaterialName(''Examples/GrassFloor'');
\end{lstlisting}

Si vous voulez conna\^itre les mat\'eriaux fournis avec Ogre, il vous suffit d'aller dans le dossier media/materials/scripts. Ici, on prend le mat\'eriau GrassFloor enregistr\'e dans le fichier Examples.material. 

Les textures correspondantes se trouvent dans le dossier media/materials/textures, si vous voulez faire des essais.


Vous pouvez maintenant ex\'ecuter votre programme.

Lancez l'application et remontez la cam\'era avec la souris et les touches de d\'eplacement, vous devriez voir quelque chose ressemblant \`{a} la capture suivante.

Image utilisateur

Euh... La t\^ete d'Ogre est coup\'ee par le sol en herbe...

En effet, notre plan est centr\'e sur l'origine de la sc\`ene, et l'on a aussi plac\'e notre t\^ete \`{a} l'origine. Mais quelle partie de la t\^ete est \`{a} l'altitude 0 ?

Ici, c'est donc un point au milieu de la t\^ete, puisque le plan d'herbe passe par l\`{a}.
Cependant, ce point n'est pas n\'ecessairement au milieu de l'objet que vous int\'egrez. Cela d\'epend de la personne qui a mod\'elis\'e l'objet et qui a donc d\'ecid\'e par rapport \`{a} quel point on allait d\'efinir la position du mesh. Pour un personnage, on pourrait mettre ce point \`{a} ses pieds, pour que l'altitude 0 corresponde effectivement au moment o\`{u} le personnage touche le sol avec ses pieds.

Pour corriger cela, il va falloir remonter notre noeud li\'e \`{a} notre entit\'e. C'est l'objet du prochain chapitre.

































\subsection{Code}

\subsubsection{PremiereApplication.cpp}
\begin{lstlisting}[caption={PremiereApplication.cpp: Cr\'eation d'un sol}]
#include "PremiereApplication.h"

void PremiereApplication::createScene()
{
    //creation d une entite
    Entity *head= mSceneMgr->createEntity("Tete", "ogrehead.mesh" );
    
    //creation d un noeud
    SceneNode *node= mSceneMgr->getRootSceneNode( )->createChildSceneNode( "nodeTete " , Vector3::ZERO, Quaternion::IDENTITY);
    
    node->yaw(Radian(Math::PI));
    node->yaw(Radian(Math::PI));

    //setPosition place le noeud aux coord passees en parametres
    Vector3 position = Vector3(30.0, 50.0, 0.0);
    node->setPosition(position);

    node->setPosition(30.0, 50.0, 0.0); 
    /*equivalent a
    Vector3 position = Vector3(30.0, 50.0, 0.0);
    node->setPosition(position);
    */

    //deplace le noeud par rapport a sa position actuelle
    node->translate(-30.0, 50.0, 0.0); //par defaut la trnslt se fait par rap a TS_WORLD
   
    //attachement de l entite au noeud
    node->attachObject ( head );

    //creation d un plan
    Plane plan(Vector3::UNIT_Y, 0);

    //creation d un mesh cad l objet 3d visible ds la scene
    MeshManager::getSingleton().createPlane("sol",
                ResourceGroupManager::DEFAULT_RESOURCE_GROUP_NAME,
                plan, 500, 500, 1, 1, true, 1, 1, 1, Vector3::UNIT_Z); 

    //entite qui representera le plan
    Entity *ent= mSceneMgr->createEntity("EntiteSol", "sol");

    //ajout du materiau a l entite
    ent->setMaterialName("Examples/GrassFloor");//texture de pelouse
    /*les differents materiaux sont sous /media/materials/scritps, par ex:
    ent->setMaterialName("Examples/WaterStream");//texture d eau animee*/

    //creation d un noeud
    node = mSceneMgr->getRootSceneNode()->createChildSceneNode();
    node->attachObject(ent);
}


\end{lstlisting}

A cause de la position de la cam\'era, il se peut alors que le sol ne soit pas visible.



%---------------------------------------------------------------------------------------------------------------


\chapter{Se repérer dans l'espace}


\section{Le système de coordonnées}


\subsection{Le repère ''main droite''}

Le repère est orthogonal et orthonormé (les axes/les vecteurs directeurs de ces axes sont perpendiculaires les uns aux autres et de longueur 1)



\subsection{Pourquoi main droite ?}

Regardez votre main droite.
\begin{itemize} 
\item Votre pouce représente l'axe X, 
\item votre index représente l'axe Y,
\item votre majeur représente l'axe Z. 
\end{itemize}

La direction dans laquelle pointe chacun de vos doigts définit le sens de chaque axe.

Image utilisateur cf(\url{fr.openclassrooms.com/informatique/cours/decouvrez-ogre-3d/le-systeme-de-coordonnees-1})





\subsection{Repère local, repère absolu}

Pour l'instant, nous n'avons toujours pas défini comment le repère était placé dans la scène. C'est une question de convention adoptée pour les applications 3D	pour le repère de la scène:
\begin{itemize}
\item l'axe Y dirigé vers le haut 
\item les axes X et Z dans un plan horizontal.
\end{itemize}

Seulement\footnote{comment faire pour sauter une ligne avant le ''Seulement''?}, la scène n'est pas la seule à avoir son repère. En effet, chaque objet possède son propre repère appelé repère local. Lorsque l'objet se déplace ou tourne sur lui-même, le repère local fait de même. L'orientation du repère local est la suivante: 
\begin{itemize}
\item l'axe Y est dirigé vers le haut de l'objet, pour la scène, 
\item l'axe X est dirigé vers sa droite
\item l'axe Z vers l'arrière de l'objet.
\end{itemize}
	


Ci-dessous\footnote{comment faire pour sauter une ligne avant le ''Ci-dessous''?}, j'ai représenté en noir le repère de la scène, et en bleu le repère local de la voiture, en respectant la convention que j'ai donnée.
Image utilisateur cf(\url{fr.openclassrooms.com/informatique/cours/decouvrez-ogre-3d/le-systeme-de-coordonnees-1})

Nous verrons à quoi servent ces différents repères lorsque l'on commencera à déplacer nos objets.








\subsection{Yaw, pitch, roll}

yaw pitch roll sont les désignations anglaises pour les rotations autour des axes Y, X et Z respectivement. On peut traduire ces termes par lacet (yaw), tangage (pitch) et roulis (roll), qui sont utilisés par exemple en aéronautique ou en navigation.

Ces trois termes se retrouveront dans les noms qu'Ogre donne aux méthodes permettant d'effectuer des rotations. 

Voici tout de suite un schéma illustrant les rotations qui s'appliquent à chaque axe:

Image utilisateur cf(\url{fr.openclassrooms.com/informatique/cours/decouvrez-ogre-3d/le-systeme-de-coordonnees-1})


Tout comme l'orientation des axes de notre repère, il y a un sens direct et un sens indirect pour les rotations! Vous tournerez dans le sens direct lors  d'une rotation contraire\footnote{''rotation contraire'' <-> ''sens direct'', est ce correct?} au sens des aiguilles d'une montre.

Pour effectuer une rotation, on appelle la méthode correspondante pour le noeud:
\begin{lstlisting}
	node->yaw(Radian(Math::PI));
\end{lstlisting}

Ceci fera faire un demi-tour au noeud par rapport à son axe vertical tandis que les méthodes pitch() et roll() s'utilisent de fa\c{c}on analogue pour les autres axes.

La rotation se fait par défaut par rapport au repère local. Il faut renseigner le second paramètre si vous voulez qu'il en soit autrement (voir la section suivante).

Les angles doivent être entrés en radians. Pour utiliser tout de même des degrés dans Ogre, vous devrez utiliser la classe Degree. La ligne de code précédente est équivalente à ceci:
\begin{lstlisting}
	node->yaw(Degree(180));
\end{lstlisting}









\section{Déplacer des objets}




\subsection{Bouger un noeud de scène}

Nous allons maintenant déplacer notre tête d'ogre pour vérifier la théorie et enfin sortir notre tête de terre!

Pour cela, nous allons donc passer par le noeud auquel est rattaché notre mesh. Celui-ci possède deux méthodes qui peuvent nous servir.




\subsection{setPosition()}

La méthode setPosition() prend en paramètres les trois coordonnées X, Y et Z du point auquel on désire placer le noeud. On peut aussi lui passer un Vector3, qui contiendra lui-même ces coordonnées.

Les deux codes suivants sont donc équivalents.
\begin{lstlisting}
	Vector3 position = Vector3(30.0, 50.0, 0.0);
	node->setPosition(position);
\end{lstlisting}

ou
\begin{lstlisting}
	node->setPosition(30.0, 50.0, 0.0);
\end{lstlisting}

La tête s'est maintenant déplacée vers la droite de 30 unités et de 50 unités vers le haut.




\subsection{translate()}

La méthode translate() déplace le noeud par rapport à sa position actuelle plutôt que par rapport à l'origine de la scène.

Elle prend les mêmes paramètres que la méthode setPosition(), mais avec un paramètre supplémentaire, défini par défaut, indiquant le noeud par rapport auquel on va se déplacer.
\begin{lstlisting}
	node->translate(-30.0, 50.0, 0.0);
\end{lstlisting}

En ajoutant cette ligne après la précédente, notre objet se retrouve donc maintenant à la position (0, 100, 0) dans la scène, ce qui est suffisant pour qu'il surplombe son petit jardin.

Image utilisateur

Le paramètre supplémentaire (par rapport à setPosition) permet de définir par rapport à quel repère on va déplacer le noeud.

Les trois valeurs possibles sont:
\begin{lstlisting}
    Node::TS_LOCAL //va deplacer le noeud par rapport au repere local
    Node::TS_PARENT //va deplacer le noeud au repere du noeud parent
    Node::TS_WORLD //va deplacer le noeud au repere de la scene, qui est le repere absolu.
\end{lstlisting}\footnote{Pour présenter les 3 valeurs possibles ne devrais je pas faire une liste itemize plutôt que d'utiliser un bloc de code?}



\subsection{Concrètement, \c{c}a veut dire quoi ?}

Tout à l'heure, lorsque l'on a effectué une translation, on l'a fait par défaut par rapport au repère TS\_WORLD, c'est-à-dire avec les axes tels que je vous les ai présentés précédemment. Maintenant, nous pouvons déplacer notre noeud par rapport au repère local de son noeud père par exemple, ou bien même par rapport à son propre repère local.

Mais à quoi \c{c}a sert de s'embêter avec ces paramètres ? On risque de faire des erreurs si l'on se place par rapport à un repère différent de la scène!

Prenons un exemple. Vous avez un vaisseau spatial qui peut se trouver dans n'importe quelles position et orientation de l'espace. Comment savoir facilement dans quelle direction je dois faire ma translation pour qu'on le voit aller en avant ?

Réponse: je n'ai pas à m'en occuper! En effet, l'axe qui va de l'avant vers l'arrière du vaisseau est l'axe Z, dans son repère local. Par conséquent, je n'ai qu'à dire à mon vaisseau d'avancer le long de l'axe Z (dans le sens négatif pour aller à l'avant) par rapport à son repère local. Et Ogre s'occupera gentiment de faire les calculs pour placer mon vaisseau correctement dans la scène.

Image utilisateur \url{http://fr.openclassrooms.com/informatique/cours/decouvrez-ogre-3d/deplacer-des-objets}

Si vous avez compris cela, le paramètre TS\_PARENT devrait suivre tout seul. Reprenons notre engin spatial.
Sur ce vaisseau, on trouve R2D2 en train de se déplacer vers la droite, correspondant donc à l'axe X local du noeud du vaisseau. Pour effectuer cette translation, je n'ai qu'à demander à Ogre de déplacer mon robot le long de l'axe des abscisses par rapport au noeud du vaisseau (qui serait logiquement le noeud parent).

Vous commencez à comprendre l'intérêt des relations de parenté entre les noeuds ?









\section{La caméra}


La caméra, c'est l'élément qui définit la position de notre point de vue dans la scène, dans quelle direction on regarde, mais aussi jusqu'à quelle distance il est possible de voir s'afficher les objets éloignés.

Comme tous les éléments de base, un attribut caméra est présent dans la classe ExampleApplication. Sans elle nous n'aurions pas encore pu voir notre scène, vu que nous n'avons rien fait pour la créer!



\subsection{Création}

La caméra est créée par la méthode  createCamera() de la classe ExampleApplication, nous allons tout de suite redéfinir cette méthode pour partir sur des bases connues. L'attribut correspondant à la caméra est appelé mCamera, nous pouvons donc l'utiliser pour créer notre caméra.

Comme c'est un objet qui se trouve dans la scène, nous allons passer par le sceneManager. Comme pour les noeuds ou les entités, vous pourrez donner un nom à votre caméra sous forme d'une cha\^ine de caractères.

Ajoutez la méthode createCamera() à votre classe PremiereApplication et ajoutez-y la ligne suivante.

\begin{lstlisting}
	mCamera = msceneMgr->createCamera(''Ma Camera'');
\end{lstlisting}





\subsection{Placement}

Maintenant, il va nous falloir placer la caméra et l'orienter. Le placement se fait avec la méthode setPosition(). 

La seconde méthode utilisée s'appelle lookAt() et, comme son nom l'indique, elle permet de déterminer le point de la scène que regarde notre caméra. On lui fournit un Vector3 ou bien trois réels correspondant aux coordonnées désirées.

\begin{lstlisting}
	//placement de la camera
	mCamera->setPosition(Vector3(-100.0, 150.0, 200.0));
	//point de la scene que regarde notre camera
	mCamera->lookAt(Vector3(0.0, 100.0, 0.0));
\end{lstlisting}


Enfin, on peut aussi indiquer les distances near clip et far clip, qui sont les distances minimale et maximale\footnote{''les distances minimale et maximale'' il faut pas de ''s'' à ''max/minimale""} auxquelles doit se trouver un objet pour être affiché à l'écran.
\begin{lstlisting}
	mCamera->setNearClipDistance(1);
	mCamera->setFarClipDistance(1000);
\end{lstlisting}




\subsection{Code}

\subsubsection{PremiereApplication.cpp}
\begin{lstlisting}[caption={PremiereApplication.cpp: Création de la caméra}]

#include "PremiereApplication.h"

void PremiereApplication::createScene()
{
    //creation d une entite
    Entity *head= mSceneMgr->createEntity("Tete", "ogrehead.mesh" );
    
    //creation d un noeud
    SceneNode *node= mSceneMgr->getRootSceneNode( )->createChildSceneNode( "nodeTete " , Vector3::ZERO, Quaternion::IDENTITY);
    
    node->yaw(Radian(Math::PI));
    node->yaw(Radian(Math::PI));

    //setPosition place le noeud aux coord passees en parametres
    Vector3 position = Vector3(30.0, 50.0, 0.0);
    node->setPosition(position);

    node->setPosition(30.0, 50.0, 0.0); 
    /*equivalent a
    Vector3 position = Vector3(30.0, 50.0, 0.0);
    node->setPosition(position);
    */

    //deplace le noeud par rapport a sa position actuelle
    node->translate(-30.0, 50.0, 0.0); //par defaut la trnslt se fait par rap a TS_WORLD
   
    //attachement de l entite au noeud
    node->attachObject ( head );

    //creation d un plan
    Plane plan(Vector3::UNIT_Y, 0);

    //creation d un mesh cad l objet 3d visible ds la scene
    MeshManager::getSingleton().createPlane("sol",
                ResourceGroupManager::DEFAULT_RESOURCE_GROUP_NAME,
                plan, 500, 500, 1, 1, true, 1, 1, 1, Vector3::UNIT_Z); 

    //entite qui representera le plan
    Entity *ent= mSceneMgr->createEntity("EntiteSol", "sol");

    //ajout du materiau a l entite
    ent->setMaterialName("Examples/GrassFloor");//texture de pelouse
    /*les differents materiaux sont sous /media/materials/scritps, par ex:
    ent->setMaterialName("Examples/WaterStream");//texture d eau animee*/

    //creation d un noeud
    node = mSceneMgr->getRootSceneNode()->createChildSceneNode();
    node->attachObject(ent);
}

/*definit la position de notre point de vue*/
void PremiereApplication::createCamera()
{
    //creation de la camera
    mCamera = mSceneMgr->createCamera("Ma Camera");

    //position de la camera
    mCamera->setPosition(Vector3(-100.0, 150.0, 200.0));

    //permet de determiner le point de la scene que regarde notre camera
    mCamera->lookAt(Vector3(0.0, 100.0, 0.0));

    //definition des distances de near clip et de far clip, qui
    //sont les distances minimale et maximale auxquelles doit se
    //trouver un objet pour etre afficher a l'ecran.
    mCamera->setNearClipDistance(1);
    mCamera->setFarClipDistance(1000);
}

\end{lstlisting}



\subsubsection{PremiereApplication.h}
\begin{lstlisting}[caption={PremiereApplication.h: Création de la caméra}]
using namespace std;

#include <ExampleApplication.h>

class PremiereApplication : public ExampleApplication
{
public:
    void createScene();
    void createCamera();
};

\end{lstlisting}


\subsubsection{main.cpp}
\begin{lstlisting}[caption={main.cpp: Création de la caméra}]
#include <Ogre.h>

#include "PremiereApplication.h"

#if OGRE_PLATFORM == PLATFORM_WIN32 || OGRE_PLATFORM == OGRE_PLATFORM_WIN32
#define WIN32_LEAN_AND_MEAN
#include "windows.h"

INT WINAPI WinMain(HINSTANCE hInst, HINSTANCE, LPSTR strCmdLine, INT)
#else
int main(int argc, char **argv)
#endif
{

    PremiereApplication app;
    
    try {
      app.go();
    } catch(Ogre::Exception& e) {
#if OGRE_PLATFORM == OGRE_PLATFORM_WIN32
        MessageBoxA(NULL, e.getFullDescription().c_str(), "An exception has occurred!", MB_OK | MB_ICONERROR | MB_TASKMODAL);
#else
        fprintf(stderr, "An exception has occurred: %s\n",
            e.getFullDescription().c_str());
#endif
    }

    return 0;
}
\end{lstlisting}




































\section{Le viewport}

Une zone de rendu est une portion de l'écran sur laquelle est affichée ce que voit la caméra. La gestion de l'affichage dans une zone de rendu est à la charge de la classe Viewport.

Cette gestion de la zone de rendu a cela d'important que:
\begin{itemize}
\item la fa\c{c}on dont la caméra rend à l'écran ce qu'elle voit ne dépend pas que d'elle. En effet, la taille de votre zone de rendu et son format seront répercutés sur la portion de scène qu'il vous sera donné de voir. Si l'on ne tenait pas compte de ces paramètres, on pourrait obtenir une image aplatie si l'on élargissait la zone de rendu, ou bien au contraire compressée si l'on diminuait la largeur en laissant la hauteur constante.
\item sur une même écran, vous pouvez afficher le rendu de plusieurs caméras dans la scène, voire des caméras de différents sceneManager.
\end{itemize}


Nous allons redéfinir la méthode createViewports() qui s'occupait jusqu'alors de ce travail pour nous. On commence par ajouter une vue:

\begin{lstlisting}
	Viewport *vue = mWindow->addViewport(mCamera);
\end{lstlisting}

Ici, mWindow est la fenêtre de notre application Ogre, c'est une instance de la classe RenderWindow dont nous verrons les détails dans un prochain chapitre. La méthode addViewport permet donc d'ajouter une vue, seul son premier argument est obligatoire. Ce premier argument est la caméra à partir de laquelle est rendu la scène affichée dans la vue.

Nous allons faire co\"incider le rapport largeur/hauteur de notre caméra avec celui du Viewport, pour avoir une image non déformée:
\begin{lstlisting}
mCamera->setAspectRatio(Real(vue->getActualWidth()) / Real(vue->getActualHeight()));
\end{lstlisting}

On applique un cast vers le format Ogre::Real pour obtenir un ratio décimal. Dans le cas contraire, le ratio serait tronqué pour être entier et la tête de notre ogre favori serait déformée.

Sachez aussi que c'est le Viewport qui définit la couleur de fond de la scène que vous voyez. 
\begin{lstlisting}
vue->setBackgroundColour(ColourValue(0.0, 0.0, 1.0));
\end{lstlisting}

Voici donc ma méthode createViewports() au complet:
\begin{lstlisting}[caption={Création d'un viewport}]
void PremiereApplication::createViewports()
{
    Viewport *vue = mWindow->addViewport(mCamera);
    
    
    //nous faisons coincider le rapport largeur / hauteur de 
    //notre camera avec celui du Viewport, pour avoir 
    //une image non deformee
    mCamera->setAspectRatio(Real(vue->getActualWidth()) / Real(vue->getActualHeight()));
    
    //couleur de fond de la vue
    vue->setBackgroundColour(ColourValue(0.0, 0.0, 1.0));
}
\end{lstlisting}


Maintenant, notre scène possède un magnifique ciel fond bleu.









\subsection{Plusieurs viewports}
La méthode addViewport permet d'ajouter des vues à notre scène, on peut donc avoir plusieurs vues à l'écran. Pour cela il faut renseigner les autres paramètres de la méthode addViewport.\newline
Voici le \href{http://www.ogre3d.org/docs/api/1.9/classOgre_1_1RenderTarget.html#a1a558e64db9dfd7cc4cec4547fca0e39}{prototype} de la méthode addViewport.


\begin{lstlisting}[caption={Prototype de addViewport}]
virtual Viewport* Ogre::RenderTarget::addViewport(
	 	Camera *  	cam,
		int  	ZOrder = 0,
		float  	left = 0.0f,
		float  	top = 0.0f,
		float  	width = 1.0f,
		float  	height = 1.0f 
		) 
\end{lstlisting}

Les paramètres sont les suivants\footnote{d'autres essais et recherches seraient nécessaires pour la pleine compréhension de ces paramètres}:
\begin{itemize}
\item ZOrder: ordre relatif des vues les unes par rapport aux autres, les Z-orders les plus élevés sont au-dessus des autres. Le nombre donnée est pas important en lui-même car il s'agit d'une relation avec les autres Z-order, ainsi peut-on utiliser des nombres qui ne suivent pas.
\item left: la position relative de la gauche du viewport sur la cible\footnote{''cible'' est visiblement le viewport principal} (entre 0 et 1).
\item top: la position relative du haut du viewport sur la cible (entre 0 et 1).
\item width: la largeur relative du viewport sur la cible (entre 0 et 1).
\item height: la hauteur relative du viewport sur la cible (entre 0 et 1).
\end{itemize}


	
\subsubsection{Code: 2 vues}
Le code suivant permet la création de deux viewports:


\begin{lstlisting}[caption={createViewports: création de plusieurs vues}]
void PremiereApplication::createViewports()
{
    //la creation du Viewport "principal"
    Viewport *vue = mWindow->addViewport(mCamera, 0, 0, 0, 0.8, 0.8);
    mCamera->setAspectRatio(Real(vue->getActualWidth()) /  Real(vue->getActualHeight()));
    vue->setBackgroundColour(ColourValue(0.980, 0.502, 0.447)); //saumon

   // creation d'un second viewport
   Viewport* vue2 = mWindow->addViewport(mCamera, 1, 0.5, 0, 0.2, 0.2);
   vue2->setBackgroundColour(ColourValue(0.561, 0.737, 0.561 ));  //darkseagreen
}
\end{lstlisting}

Et j'obtiens:
	\begin{center}
	\includegraphics[scale=0.25]{Ogre/1_Base_de_Ogre/2_Se_reperer_ds_l_espace/Images/plusieursViewport.png} 
	\end{center}






\subsubsection{Code: 3 vues}
Le code suivant permet la création de deux viewports:


\begin{lstlisting}[caption={createViewports: création de plusieurs vues}]

void PremiereApplication::createViewports()
{
    //Viewport "principal"
    Viewport *vue = mWindow->addViewport(mCamera);
    mCamera->setAspectRatio(Real(vue->getActualWidth()) /  Real(vue->getActualHeight()));
    vue->setBackgroundColour(ColourValue(0.980, 0.502, 0.447)); //saumon

	//seconde vue
    Viewport* vue2 = mWindow->addViewport(mCamera, 1, 0.5, 0, 0.8, 0.8);
    vue2->setBackgroundColour(ColourValue(0.561, 0.737, 0.561 ));  //darkseagreen

	//troisieme vue
	Viewport* vue3 = mWindow->addViewport(mCamera, 4, 0, 0.4, 0.2, 0.2);
    vue3->setBackgroundColour(ColourValue(0.878, 1.000, 1.000));  //
}
\end{lstlisting}

Et j'obtiens:
	\begin{center}
	\includegraphics[scale=0.25]{Ogre/1_Base_de_Ogre/2_Se_reperer_ds_l_espace/Images/plusieursViewport2.png} 
	\end{center}


%---------------------------------------------------------------------------------------------------------------
\chapter{La lumi\`ere}



\section{Les lumi\`eres}


\subsection{Quelques fonctions de base}

La lumi\`ere est n\'ecessaire pour pouvoir voir quelque chose dans une sc\`ene. Comment a-t-on donc pu voir nos objets depuis le d\'ebut de ce cours?

Il existe une propri\'et\'e du sc\`ene Manager qui permet de d\'efinir une lumi\`ere ambiante. Cela permet d'\'eclairer la sc\`ene de fa\c{c}on homog\`ene avec une certaine luminosit\'e. Par d\'efaut, on a un \'eclairage \`{a} la lumi\`ere blanche qui permet de voir ce qui se passe dans la sc\`ene. La ligne suivante peut \^etre ajout\'ee au d\'ebut de la m\'ethode createsc\`ene() pour appliquer une lumi\`re ambiante noire nous pourrons ainsi d\'efinir des lumi\`res et voir leur influence.

\begin{lstlisting}
msceneMgr->setAmbientLight(ColourValue(0.0, 0.0, 0.0));
\end{lstlisting}



La classe ColourValue permet de d\'efinir une couleur en entrant les quantit\'es respectives de rouge, de vert puis de bleu dans un nombre compris entre 0 et 1. Il est aussi possible de d\'efinir une composante alpha (la transparence), utile pour des textures par exemple.

Une lumi\`ere est un objet de sc\`ene, on passe donc par le sc\`ene manager pour cr\'eer une lumi\`ere:
\begin{lstlisting}
Light *light = msceneMgr->createLight(''lumiere1'');
\end{lstlisting}

Par d\'efaut, la lumi\`ere cr\'e\'ee est de type ponctuelle. Vous avez plus de d\'etails sur les diff\'erents types de lumi\`ere un peu plus bas.

Mettons tout de suite en place quelques param\`etres de base: la couleur \'emise (diffuse et sp\'eculaire, que nous d\'etaillerons dans le chapitre sur les mat\'eriaux) et la position.
\begin{lstlisting}
light->setDiffuseColour(1.0, 0.7, 1.0);
light->setSpecularColour(1.0, 0.7, 1.0);
light->setPosition(-100, 200, 100);
\end{lstlisting}




Les deux premiers param\`etres sont les couleurs diffuses et sp\'eculaires, au format RVB, avec des valeurs qui doivent \^etre comprises entre 0 et 1.

La couleur diffuse est la couleur sous laquelle vont appara\^itre les objets non brillants, et la couleur sp\'eculaire est un param\`etre suppl\'ementaire pour les mat\'eriaux r\'efl\'echissants comme le m\'etal ou le verre. Pour une lumi\`ere, on met g\'en\'eralement la m\^eme couleur pour ces deux param\`etres.

Vient ensuite la m\'ethode setPosition(), qui ne devrait pas vous poser de probl\`emes, et enfin une derni\`ere ligne \footnote{quelle derni\`ere ligne ???} permettant d'amplifier ou de diminuer l'intensit\'e lumineuse. Par d\'efaut, ce coefficient est de 1, mais pour notre sc\`ene j'ai voulu l'augmenter pour qu'on y voie un peu plus clair: n'h\'esitez pas \`{a} jouer un peu avec pour faire des essais.

Enfin, sachez qu'il est possible d'attacher une lumi\`ere \`{a} un noeud de sc\`ene. Dans ce cas, la m\'ethode Light::setPosition() d\'efinit la position relative de la lumi\`ere par rapport au noeud.
\begin{lstlisting}
	node->attachObject(light);
\end{lstlisting}


Vous pouvez donc facilement placer une lumi\`ere \`{a} la position d'un mesh cens\'e \'emettre de la lumi\`ere - par exemple les phares d'une voiture - et les d\'eplacer en m\^eme temps gr\^{a}ce \`{a} une seule commande vers le noeud de sc\`ene!





\subsection{Les types de lumi\`eres}

Ogre peut g\'erer diff\'erents types de lumi\`eres selon l'effet d\'esir\'e. Ils sont au nombre de 3:

\begin{itemize}
\item la lumi\`ere ponctuelle: cette lumi\`ere \'emet dans toutes les directions \`{a} partir de sa position;
\item la lumi\`ere directionnelle: une lumi\`ere dont les rayons vont dans une direction unique et qui n'a pas de position. C'est le genre de lumi\`ere qui permet de reproduire l'\'eclairage du soleil par exemple;
\item le projecteur ou spot: c'est une lumi\`ere qui \'emet un c\^one lumineux \`{a} partir de sa position, \`{a} la fa\c{c}on d'une lampe-torche.
\end{itemize}
    




\subsection{Lumi\`ere ponctuelle}
C'est le type de lumi\`ere cr\'e\'e par d\'efaut que l'on a vu plus haut. Pour le modifier manuellement, il faut utiliser le type LT\_POINT.

\begin{lstlisting}
	light->setType(Light::LT_POINT);
\end{lstlisting}


Avec ce type de lumi\`ere, il existe une m\'ethode nous permettant aussi de limiter la port\'ee de notre \'eclairage. Voici le prototype:
\begin{lstlisting}
Light::setAttenuation( Real range, Real constant, Real linear, Real quadratic )
\end{lstlisting}



C'est plus d\'elicat car les param\`etres doivent \^etre choisis avec soin. 
\begin{itemize}
\item Le premier est la distance caract\'eristique d'att\'enuation, c'est-\`{a}-dire la distance \`{a} partir de laquelle la luminosit\'e diminue. 
\item constant est une constante d'att\'enuation comprise entre 0 et 1. Plus elle est proche de 0, et plus le passage de la lumi\`ere \`{a} l'ombre est brutal.
\item linear et quadratic sont les param\`etres de la courbe d'att\'enuation, et doivent \^etre assez faibles, sinon la lumi\`ere s'att\'enue trop rapidement.
\end{itemize}
	


Ca ne marche pas! L'ogre est bien \'eclair\'e mais le sol reste d\'esesp\'er\'ement noir!

L'att\'enuation ajoute une caract\'eristique un peu diff\'erente pour la gestion de la lumi\`ere. En effet, l'\'eclairage des surfaces est calcul\'e en fonction des vertices situ\'es dans la zone d'\'eclairage. Lorsqu'un vertice est dans la zone d'\'eclairage du spot, la surface autour de lui est \'eclair\'ee, sinon elle est dans l'ombre. Ogre se charge ensuite de faire les d\'egrad\'es entre les vertices plus ou moins \'eclair\'es.

Mais ici, notre plan n'est constitu\'e que de quatre vertices (les coins), dont aucun n'est \'eclair\'e par le spot. Le sol n'est donc pas \'eclair\'e.

Pour r\'egler \c{c}a, il suffit de modifier notre sol pour qu'il poss\`ede plus de vertices. Retrouvez la d\'efinition du plan et modifiez les param\`etres de d\'ecoupage pour obtenir 10 segments en largeur et en longueur (les deux param\`etres avant le true).

\begin{lstlisting}
Plane plan(Vector3::UNIT_Y, 0);
MeshManager::getSingleton().createPlane(''sol'', ResourceGroupManager::DEFAULT_RESOURCE_GROUP_NAME, plan, 500, 500, 10, 10, true, 1, 1, 1, Vector3::UNIT_Z);
\end{lstlisting}



Notez que plus il y aura de vertices sur le mod\`ele, plus ce sera pr\'ecis, mais ce sera un peu plus co\^uteux en ressources.




\subsection{Lumi\`ere directionnelle}

Etant donn\'e que cette lumi\`ere est de type ''soleil'' et qu'elle \'emet \`{a} l'infini, il n'est pas utile de renseigner sa position. En revanche, la m\'ethode setDirection() permet de d\'efinir le vecteur directeur des rayons lumineux.

\begin{lstlisting}
light->setType(Light::LT_DIRECTIONAL);
light->setDirection(10.0, -20.0, -5);
\end{lstlisting}






\subsection{Projecteur}

Le projecteur permet g\'en\'eralement de simuler un \'eclairage artificiel en proposant une lumi\`ere directionnelle d\'efinie dans un c\^one central et un c\^one ext\'erieur, avec deux intensit\'es diff\'erentes. Le c\^one central d\'efinit une lumi\`ere plus forte que le c\^one ext\'erieur, o\`{u} la lumi\`ere est quelque peu att\'enu\'ee. Ces deux c\^ones sont d\'efinis par leur angle d'ouverture, ainsi que par un falloff, c'est-\`{a}-dire un coefficient indiquant si la transition entre les deux c\^ones doit \^etre plus ou moins rapide:
\begin{lstlisting}
light->setType(Light::LT_SPOTLIGHT);
light->setPosition(0, 150, -100);
light->setDirection(0, -1, 1);
light->setSpotlightRange(Degree(30), Degree(60), 1.0);
\end{lstlisting}


Notez que l'\'eclairage du spot ob\'eit aux m\^emes r\`egles que pour l'att\'enuation d'une lumi\`ere ponctuelle: il faut que les vertices soient \'eclair\'es pour que l'\'eclairage soit visible.


De m\^eme que pour les lumi\`eres ponctuelles, vous pouvez ajouter une port\'ee limit\'ee \`{a} votre projecteur avec la m\'ethode setAttenuation().






\section{Les ombres}


\subsection{Activer les ombres}

Tout d'abord, on doit param\'etrer nos lumi\`eres et nos entit\'es pour projeter (ou non) des ombres.
Que ce soit pour les lumi\`eres ou les entit\'es, on utilise la m\^eme m\'ethode pour choisir d'activer ou non la projection:

\begin{lstlisting}
light->setCastShadows(true);
head->setCastShadows(true);
\end{lstlisting}


Si vous voulez qu'une entit\'e ne projette aucune ombre, il suffit de mettre le param\`etre \`{a} false. De m\^eme si vous voulez qu'une lumi\`ere ne projette aucune ombre pour les entit\'es (pour une lumi\`ere d'ambiance ou d'ajustement, par exemple).

N'oubliez pas de d\'esactiver la projection d'ombres pour le sol. D'une part parce que celui-ci n'a pas besoin de projeter d'ombres, d'autre part parce que certaines techniques n\'ecessitent d'avoir ce param\`etre d\'esactiv\'e pour avoir une ombre sur le mesh.

Avant de pouvoir afficher les ombres, il faut les activer. Cela se fait dans le sc\`ene Manager, par exemple:

\begin{lstlisting}
msceneMgr->setShadowTechnique(Ogre::SHADOWTYPE_STENCIL_ADDITIVE);
\end{lstlisting}



On d\'efinit ici la technique de rendu qui sera utilis\'ee pour les ombres dans la sc\`ene (voir ci-dessous).




\subsection{Les diff\'erents types d'ombres}

Ogre permet de g\'en\'erer diff\'erents types d'ombres selon les besoins, qui d\'ependent g\'en\'eralement des mod\`eles concern\'es par la projection d'ombres.

Il existe deux techniques pour la g\'en\'eration d'ombres:
\begin{itemize}
\item le type ''Stencil'' (pochoir en anglais)
\item le type ''Texture''
\end{itemize}

Les ombres de type Stencil sont tr\`es pr\'ecises dans les contours et permettent une tr\`es bonne projection d'ombre lorsque l'on y regarde de pr\`es. En revanche, elles sont assez co\^uteuses en ressources, notamment lorsque les mesh sont anim\'es. Enfin, elles ne prennent pas du tout en compte la transparence des textures, un cube de texture transparente projettera donc une ombre si ce param\`etre est activ\'e.
Les ombres de type Texture permettent de g\'erer la transparence des textures et sont moins co\^uteuses en ressources, mais leur pr\'ecision est plus faible.

Enfin, chacune de ces deux cat\'egories est compos\'ee de deux techniques, l'une dite modulative, l'autre additive. On obtient ainsi quatre techniques possibles:
\begin{itemize}
\item SHADOWTYPE\_TEXTURE\_MODULATIVE
\item SHADOWTYPE\_TEXTURE\_ADDITIVE
\item SHADOWTYPE\_STENCIL\_MODULATIVE
\item SHADOWTYPE\_STENCIL\_ADDITIVE
\end{itemize}


Notez que dans chacun des cas, la technique additive est la meilleure, notamment pour une approche de type Stencil. La diff\'erence pour les techniques de type Texture est minime; en revanche, la technique Stencil additive permet d'obtenir des ombres plus ou moins sombres en fonction de l'\'eclairage gr\^{a}ce \`{a} des passes successives, tandis que la m\'ethode Stencil modulative ne fait que projeter le mod\`ele au sol une seule fois pour chaque lumi\`ere.

C'est donc dans le sc\`ene Manager que l'on s'occupe de d\'eterminer la technique de rendu des ombres. Par d\'efaut, celles-ci ne sont pas rendues.
\begin{lstlisting}
msceneMgr->setShadowTechnique(Ogre::SHADOWTYPE_STENCIL_MODULATIVE);
\end{lstlisting}


Il n'est possible d'avoir qu'une seule technique enregistr\'ee \`{a} la fois dans un sc\`ene Manager. Il n'est pas possible de choisir les types d'ombres \`{a} g\'en\'erer pour chaque lumi\`ere de la sc\`ene. Il faut donc faire un choix pour l'ensemble de vos ombres.

Ci-dessous, l'approche bas\'ee sur la texture, peu pr\'ecise mais peu co\^uteuse en ressources:

Image utilisateur \url{http://fr.openclassrooms.com/informatique/cours/decouvrez-ogre-3d/les-ombres-1}



\subsection{Code}
Dans le code ci-dessous a \'et\'e ajout\'ee une m\'ethode pour la cr\'eation d'une lumi\`ere et les ombres. Les objets qui proj\`etent des ombres doivent \^etre activ\'e, i\c{c}i seule la t\^ete de ogre et la lumi\`ere elle-m\^eme projetent des ombres

\subsubsection{PremiereApplication.h}



\begin{lstlisting}[caption={PremiereApplication.h: ajout d'une m\'ethode pour la gestion de lumi\`ere et des ombres}]
using namespace std;

#include <ExampleApplication.h>
#include <OgreMovableObject.h>


class PremiereApplication : public ExampleApplication
{
    public:
        void createScene();
        void createCamera();
        void createViewports();

        void createLux(std::string, MovableObject *);
};

\end{lstlisting}







\subsubsection{PremiereApplication.cpp}



\begin{lstlisting}[caption={PremiereApplication.cpp: ajout d'une m\'ethode pour la gestion de lumi\`ere et des ombres}]
#include "PremiereApplication.h"

void PremiereApplication::createScene()
{
    //creation d une entite
    Entity *head= mSceneMgr->createEntity("Tete", "ogrehead.mesh" );
    
    //creation d un noeud
    SceneNode *node= mSceneMgr->getRootSceneNode( )->createChildSceneNode( "nodeTete " , Vector3::ZERO, Quaternion::IDENTITY);
    
    node->yaw(Radian(Math::PI));
    node->yaw(Radian(Math::PI));

    //setPosition place le noeud aux coord passees en parametres
    Vector3 position = Vector3(30.0, 50.0, 0.0);
    node->setPosition(position);

    node->setPosition(30.0, 50.0, 0.0); 
    /*equivalent a
    Vector3 position = Vector3(30.0, 50.0, 0.0);
    node->setPosition(position);
    */

    //deplace le noeud par rapport a sa position actuelle
    node->translate(-30.0, 50.0, 0.0); //par defaut la trnslt se fait par rap a TS_WORLD
   
    //attachement de l entite au noeud
    node->attachObject(head);

    //creation d un plan
    Plane plan(Vector3::UNIT_Y, 0);

    //creation d un mesh cad l objet 3d visible ds la scene
    MeshManager::getSingleton().createPlane("sol",  ResourceGroupManager::DEFAULT_RESOURCE_GROUP_NAME, plan, 500, 500, 10, 10, true, 1, 1, 1, Vector3::UNIT_Z); 

    //entite qui representera le plan
    Entity *ent= mSceneMgr->createEntity("EntiteSol", "sol");

    //ajout du materiau a l entite
    ent->setMaterialName("Examples/GrassFloor");//texture de pelouse
    /*les differents materiaux sont sous /media/materials/scritps, par ex:
    ent->setMaterialName("Examples/WaterStream");//texture d eau animee*/

    //creation d un noeud
    node = mSceneMgr->getRootSceneNode()->createChildSceneNode();
    node->attachObject(ent);

    createLux("ponctuelle", head);//lumiere ponctuelle
    //createLux("directionnelle", head);//lumiere directionnelle
    //createLux("spot", head);//lumiere projecteur
}

/*
cree une lumiere selon le parametre passe:
    createLux("ponctuelle"); -> lumiere ponctuelle
    createLux("directionnelle"); -> lumiere directionnelle
    createLux("spot"); -> lumiere projecteur
    
    le second parametre est l entite pour laquelle on active les ombres

une lumiere noire est cree au debut de la methode

une ombre est cree en fin de methode
*/
void PremiereApplication::createLux(std::string prmLightType, MovableObject * prmEnt)
{
    //application d une couleur noire
    mSceneMgr->setAmbientLight(ColourValue(0.0, 0.0, 0.0)); 

    //definition d une lumiere 
    Light *light= mSceneMgr->createLight("lumiere1");

    if (prmLightType == "ponctuelle")
    {
        //definition du type de lumiere
        light->setType(Light::LT_POINT);//lumiere ponctuelle

        //definition de la position de la lumiere
        light->setPosition(-100, 200, 100);
    }
    else if (prmLightType == "directionnelle")
    {
        light->setType(Light::LT_DIRECTIONAL);//lumiere directionnelle
        light->setDirection(10.0, -20.0, -5);//vecteur directeur de la lumiere directionnelle

        //definition de la position de la lumiere
        light->setPosition(-100, 200, 100);
    }
    else
    {
        light->setType(Light::LT_SPOTLIGHT);//lumiere directionnelle
        light->setDirection(0.0, -1, 1);//vecteur directeur de la lumiere directionnelle
        light->setSpotlightRange(Degree(30), Degree(60), 1.0);
    }

    //definition des couleur des lumieres diffuse
    light->setDiffuseColour(1.0, 0.7, 0.1);
    //et speculaire
    light->setSpecularColour(1.0, 0.7, 0.1);

    //ombre
    //activation de la projection des ombres
    light->setCastShadows(true);
    prmEnt->setCastShadows(true);

    //activation des ombres
    mSceneMgr->setShadowTechnique(Ogre::SHADOWTYPE_STENCIL_ADDITIVE);

}

/*definit la position de notre point de vue*/
void PremiereApplication::createCamera()
{
    //creation de la camera
    mCamera = mSceneMgr->createCamera("Ma Camera");

    //position de la camera
    mCamera->setPosition(Vector3(-100.0, 150.0, 200.0));

    //permet de determiner le point de la scene que regarde notre camera
    mCamera->lookAt(Vector3(0.0, 100.0, 0.0));

    //definition des distances de near clip et de far clip, qui
    //sont les distances minimale et maximale auxquelles doit se
    //trouver un objet pour être affichr à l'écran.
    mCamera->setNearClipDistance(1);
    mCamera->setFarClipDistance(1000);
}

void PremiereApplication::createViewports()
{
    //la creation du Viewport, appelee par la fenetre et prenant en parametre la
    //camera concernee, le premier parametre est la camera de laqll le contenu
    //du viewport sera rendu, ce paramatre est le seul obligatoire
    Viewport *vue = mWindow->addViewport(mCamera);
    //Viewport *vue = mWindow->addViewport(mCamera, 0, 0, 0, 0.8, 0.8);

    //Grace a ce Viewport nouvellement cree, nous allons faire coincider
    //le rapport largeur / hauteur de notre camera avec celui du
    //Viewport, pour avoir une image non deformee
    mCamera->setAspectRatio(Real(vue->getActualWidth()) /  Real(vue->getActualHeight()));

    //on definit ici la couleur de fond
    //vue->setBackgroundColour(ColourValue(0.0, 0.0, 1.0));     //bleu
    vue->setBackgroundColour(ColourValue(0.980, 0.502, 0.447)); //saumon

   // creation d'un viewport dans le coin bas gauche
   //les parametres autres que le premier sont obligatoires pour la definition
   //de plusieurs viewport
   Viewport* vue2 = mWindow->addViewport(mCamera, 1, 0, 0.8, 0.2, 0.2);
   vue2->setBackgroundColour(ColourValue(0.561, 0.737, 0.561 ));  //darkseagreen
}


\end{lstlisting}





%---------------------------------------------------------------------------------------------------------------
\chapter{La gestion des entr\'ees}




\section{Les frame listeners}





\subsection{Des ''\'ecouteurs d'images'' ?}
\subsubsection{Utilit\'e}
Lorsque vous g\'erez les entr\'ees de l'utilisateur (et m\^eme pour faire des calculs divers durant l'ex\' ecution de votre programme), l'ordinateur effectue les instructions n\'ecessaires entre deux images (ou frames en anglais), donc pendant un temps tr\`es court.

En pratique, le moteur fonctionne dans une boucle, qui ne fait qu'afficher une image, puis fait des calculs; et ainsi de suite, sans s'arr\^eter. Il est donc possible pour le programmeur de donner ses instructions avant qu'une image soit rendue, ou bien apr\`es, ou bien m\^eme pendant que la carte graphique fait le rendu graphique.

Lorsque l'on aura vu comment cr\'eer cette boucle de rendu, nous serons \`{a} m\^eme de donner les instructions de la mani\`ere dont nous le d\'esirons. En attendant, je vais vous pr\'esenter une classe qui a l'avantage de permettre de faire tout ce que je viens de vous expliquer de fa\c{c}on tr\`es simple: le frame listener.



\subsubsection{Les m\'ethodes \`{a} conna\^itre}
Un frame listener est une classe interface qui poss\`ede trois m\'ethodes, dont voici les prototypes:

\begin{itemize}
\item virtual bool frameStarted(const FrameEvent\& evt): est appel\'ee avant que la frame ne soit rendue (frameStarted)
\item virtual bool frameRenderingQueued(const FrameEvent\& evt): est appel\'ee apr\`es que la frame ait \'et\'e rendue (frameEnded)
\item virtual bool frameEnded(const FrameEvent\& evt): est appel\'ee apr\`es que le processeur graphique ait re\c{c}u les instructions pour le rendu (frameRenderingQueued)\newline
\end{itemize}
    


En cr\'eant un objet d\'eriv\'e de la classe frame listener dans votre application et en r\'eimpl\'ementant ces m\'ethodes virtuelles, vous avez donc la possibilit\'e de demander \`{a} Ogre d'effectuer les calculs dont vous avez besoin \`{a} chaque image.



frameStarted() et frameEnded() sont tr\`es similaires, \'etant donn\'e qu'elles ont pour seule diff\'erence d'\^etre appel\'ees respectivement au d\'ebut et \`{a} la fin de la boucle de rendu. Mais comme on est dans une boucle, en r\'ealit\'e il ne se passe quasiment rien entre l'appel \`{a} frameEnded() et celui \`{a} frameStarted(). La diff\'erence peut \^etre utile par exemple si vous avez un calcul qui semble plus logique d'effectuer apr\`es que l'image soit rendue plut\^ot qu'avant, mais ce n'est qu'une question de lecture du code selon moi.


En revanche la derni\`ere (frameRenderingQueued) est plus subtile. Comme je l'ai dit, elle est appel\'ee d\`es que la carte graphique re\c{c}oit les instructions n\'ecessaires pour afficher l'image \`{a} rendre.


Vous le savez probablement, c'est une op\'eration tr\`es co\^uteuse en ressources et c'est souvent ce qui ralentit les jeux vid\'eo mettant en jeu de nombreux effets graphiques. Pendant ce temps-l\`{a}, le processeur central attend que \c{c}a se passe. Par cons\'equent, si vous appelez la m\'ethode frameRenderingQueued() pour faire des calculs, vous \'evitez d'avoir un processeur peu occup\'e pendant que la carte graphique fait son boulot !

De mani\`ere g\'en\'erale, on pourra utiliser cette m\'ethode pour des op\'erations lourdes dont on sait qu'elles seront r\'ep\'et\'ees \`{a} chaque image, afin de rentabiliser l'utilisation du processeur.




\subsubsection{Contr\^ole de l'ex\'ecution}

La valeur de retour des m\'ethodes d'un frame listener est un bool\'een r\'ecup\'er\'e par Ogre pour savoir s'il doit continuer (true) ou non (false) l'ex\'ecution du programme.



\subsubsection{Utiliser plusieurs frame listeners}

Il est possible de cr\'eer autant de frame listeners que vous le d\'esirez, pour effectuer des op\'erations diverses. En revanche, il est conseill\'e de ne pas en abuser pour \'eviter de trop segmenter votre application, il peut \^etre int\'eressant d'appeler d'autres fonctions \`{a} partir d'un frame listener plut\^ot que d'en cr\'eer trop.

Enfin, et c'est l\`{a} le plus important:
\textbf{L'ordre d'ex\'ecution des frame listeners est laiss\'e aux soins du moteur. Vous n'avez AUCUN contr\^ole dessus !}

En d'autres termes, si vous avez besoin d'effectuer des op\'erations dans un ordre pr\'ecis, ne les mettez pas dans des frame listeners diff\'erents, car vous ne pourrez pas d\'ecider de l'ordre d'ex\'ecution. Il faut alors laisser un seul frame listener g\'erer les op\'erations, ou ne pas passer par eux (ce sera possible lorsque nous attaquerons la boucle de rendu).



\subsection{Le frame listener en pratique}

Comme nous allons vouloir red\'efinir les m\'ethodes du frame listener, il faut en faire une classe d\'eriv\'ee. Vu que nous sommes dans la gestion des entr\'ees, nous allons tout de suite pr\'eparer le terrain en cr\'eant une classe InputListener d\'erivant de ExampleFrameListener.

Je d\'erive ici de la classe ExampleFrameListener, elle-m\^eme d\'eriv\'ee de FrameListener, car la classe ExampleApplication met d\'ej\`{a} en place une gestion des entr\'ees et attend un ExampleFrameListener. Cette classe s'occupe aussi de construire les objets n\'ecessaires \`{a} l'\'ecoute des entr\'ees souris/clavier, ce que nous aborderons ult\'erieurement.
Cependant, la m\'ethode de traitement reste identique, nous allons juste devoir red\'efinir frameRenderingQueued() pour impl\'ementer notre propre gestion des entr\'ees \`{a} la place de celle pr\'evue par ExampleFrameListener.


\begin{lstlisting}[caption={InputListener.h}]
#include ''ExampleFrameListener.h''
class InputListener: public ExampleFrameListener
{
public:
    InputListener(RenderWindow* win, Camera* cam, sceneManager *sceneMgr, bool bufferedKeys = false, bool bufferedMouse = false, bool bufferedJoy = false );
    virtual bool frameRenderingQueued(const FrameEvent& evt);

private:
    Ogre::sceneManager *msceneMgr; //pointeur sur le scene Manager, qui servira a retrouver des objets dans la scene.
    bool mToucheAppuyee;	 //pour garder une trace de l etat dans lequel se trouve une touche particuliere.

    /*distance de deplacement de la camera et sa vitesse, puis des angles de rotation.*/
    Ogre::Real mMouvement;
    Ogre::Real mVitesse;
    Ogre::Real mVitesseRotation;

	/*angles de rotation.*/
    Ogre::Radian mRotationX;
    Ogre::Radian mRotationY;
};
\end{lstlisting}



Dans votre fichier InputListener.cpp, pr\'eparez le constructeur ainsi que l'impl\'ementation des m\'ethodes, avec un corps vide pour l'instant:

%je ne sais pas si [language=cpp] passe et s il est utile
\begin{lstlisting}[caption={InputListener.cpp}]
/*
Arguments:
RenderWindow* win: votre RenderWindow pour l'application
Camera* cam: la camera que vous utilisez
sceneManager *sceneMgr: le scene Manager
bool bufferedKeys: indique si vous desirez utiliser le buffer pour le clavier
bool bufferedMouse: indique si vous desirez utiliser le buffer pour la souris
bool bufferedJoy: indique si vous desirez utiliser le buffer pour le joystick
*/
InputListener::InputListener(RenderWindow* win, Camera* cam, sceneManager *sceneMgr, bool bufferedKeys, bool bufferedMouse, bool bufferedJoy) 
      : ExampleFrameListener(win, cam, bufferedKeys, bufferedMouse, bufferedJoy)
{
    msceneMgr = sceneMgr;
    mVitesse = 100;
    mVitesseRotation = 0.3;
    mToucheAppuyee = false;
}
\end{lstlisting}



Il n'y a pas de constructeur \'ecrit pour les frame listeners, car c'est une classe interface\footnote{que je sache il n'y a pas d'interface en C++}\footnote{pas de constructeur parce que c'est une classe interface ??}, seules les trois m\'ethodes que j'ai pr\'esent\'ees plus haut importent. En revanche, la classe ExampleFrameListener poss\`ede un constructeur qui pr\'epare le terrain pour l'utilisation des entr\'ees, que j'appelle dans ma classe d\'eriv\'ee. 

Les trois bool\'eens en param\`etres indiquent si vous d\'esirez utiliser le buffer respectivement pour le clavier, la souris et le joystick. Comme ce sera l'objet de la derni\`ere partie de ce chapitre, je mets par d\'efaut false (dans le .h).

Dans le corps du constructeur, j'initialise mes attributs mSceneMgr, mVitesse et mToucheAppuyee, qui nous serviront par la suite.

Il n'y a pas d'attribut \`{a} ajouter dans notre classe PremiereApplication, la classe ExampleApplication contient d\'ej\`{a} un pointeur sur un ExampleFrameListener.

       




































\section{OIS}


Pour g\'erer les entr\'ees de l'utilisateur, nous allons utiliser la biblioth\`eque OIS \footnote{Object Oriented Input System, c'est \`a dire: Syst\`eme d'entr\'ees orient\'e objet}, qui est distribu\'ee par d\'efaut avec le SDK d'Ogre. Comme je l'ai dit en introduction de ce cours, un moteur 3D n'a pas de m\'ethodes pour g\'erer autre chose que ce qui s'affiche sur votre \'ecran. C'est pourquoi nous utiliserons cette biblioth\`eque pour r\'ecup\'erer les actions du joueur.\newline


Pour ce faire, OIS repr\'esente les p\'eriph\'eriques d'entr\'ee par des objets, qui sont les suivants:

\begin{itemize}
\item Mouse pour la souris;
\item Keyboard pour le clavier;
\item Joystick pour les joysticks ou manettes de jeu.\newline
\end{itemize}
    



Qui dit nouvelle biblioth\`eque dit nouveau namespace ! Ces classes se trouvent donc dans l'espace de nom OIS.\newline

Les touches du clavier et les boutons de la souris sont des \'enum\'erations, d\'efinies comme ceci:

\begin{itemize}
\item OIS::KC\_NOMDELATOUCHE pour le clavier;
\item OIS::MB\_NOMDUBOUTON pour la souris.
\end{itemize}


Ce qui donne par exemple OIS::KC\_A pour la touche 'A'\footnote{la touche 'Entr\'ee' s'appelle 'Return' en anglais, ne cherchez donc pas OIS::KC\_ENTER, vous ne trouverez pas.}, ou OIS::MC\_Left pour le clic gauche.\newline

Une derni\`ere chose bonne \`{a} savoir: les codes de touches d'OIS correspondent aux touches physiques d'un clavier QWERTY. Ce qui signifie par exemple que OIS::KC\_A correspond \`{a} la touche Q sur votre clavier AZERTY.\newline

Afin d'utiliser OIS, il faut inclure le header correspondant. Celui-ci se trouve dans le dossier OIS du dossier include, on ajoutera donc la ligne de pr\'eprocesseur suivante en t\^ete du header de InputListener:

\begin{lstlisting}[caption={Include OIS}]
#include <OIS/OIS.h>
\end{lstlisting}

Bien s\^ur, vous pouvez aussi ajouter le r\'epertoire OIS \`{a} la liste des includes de votre IDE pour \'eviter d'avoir \`{a} le pr\'eciser dans le code.




























\section{Allons-y sans buffer}\footnote{Le tutorial du site de Ogre ''Frame Listeners and Unbuffered Input'' pr\'esente le point trait\'e ici d'une autre mani\`ere \href{http://www.ogre3d.org/tikiwiki/Basic+Tutorial+4}{Frame Listeners and Unbuffered Input}}
\subsection{Explications}
Dans le constructeur de notre frame listener, je vous ai dit que l'on avait mis les param\`etres concernant l'utilisation du buffer \`a false, car c'est l'objet de la derni\`ere partie de ce chapitre. Mais que signifie le fait d'utiliser ou non le buffer ?

Lorsque vous appuyez sur une touche, le clavier envoie un signal \`a l'ordinateur pour lui dire qu'une touche est actuellement press\'ee, en pr\'ecisant quelle touche est concern\'ee. Ce signal est envoy\'e tant que la touche reste enfonc\'ee.

De son cot\'e, notre programme effectue sa boucle infinie, rendant les images et faisant les calculs demand\'es. Supposons que vous appuyez sur une touche \`a un instant donn\'e. Lorsque l'ordinateur arrivera \`a l'instruction lui demandant de regarder ce qui se passe sur le clavier, il va voir qu'une touche est enfonc\'ee et cherchera \`a effectuer les op\'erations demand\'ees, et ceci tant que la touche reste enfonc\'ee. Mais il n'est pas possible pour l'ordinateur de faire seul la diff\'erence entre une touche enfonc\'ee et une touche qui vient d'\^etre enfonc\'ee.

Nous allons donc voir comment r\'egler ce probl\`eme ''\`a la main'', puis nous verrons l'utilisation du buffer, qui constitue une autre fa\c{c}on de traiter l'entr\'ee.

\subsection{Cr\'eation du frame listener}

Avant de passer \`a la suite, r\'eimpl\'ementez la m\'ethode createFrameListener() pr\'esente dans ExampleApplication dans la classe PremiereApplication en ajoutant le prototype et la d\'efinition :

\begin{lstlisting}[caption={PremiereApplication::createFrameListener}]
void PremiereApplication::createFrameListener()
{
    //creation du framelistener en utilisant le ctor prepare plus tot
    mFrameListener= new InputListener(mWindow, mCamera, mSceneMgr, false, false, false);
    //signale a l objet root que ns avons un nv frame listener et qu il faudra l'appeler
    mRoot->addFrameListener(mFrameListener);
}
\end{lstlisting}

Le root est l'\'el\'ement de base de l'application Ogre qui s'occupe notamment de g\'erer les frame listeners, nous le reverrons plus tard en approfondissant le fonctionnement du moteur.
 









\subsection{D\'eplacer la cam\'era}


Tout d'abord, nous allons consid\'erer le d\'eplacement de cam\'era, pour lequel nous n'avons pas besoin de savoir si la touche vient d'\^etre appuy\'ee : c'est simplement son \'etat actuel qui compte.

Premi\`erement, il nous faut r\'ecup\'erer l'\'etat actuel du clavier et de la souris. Localisez la m\'ethode frameRenderingQueued() de votre InputListener et ins\'erez-y ceci :


\begin{lstlisting}[caption={}]
if(mMouse)
    mMouse->capture();
if(mKeyboard)
    mKeyboard->capture();
\end{lstlisting}

Ces deux lignes permettent de mettre \`a jour nos objets pour obtenir le nom des touches enfonc\'ees.\newline

V\'erifions d'abord si la touche Echap est utilis\'ee, auquel cas nous quitterons l'application.


\begin{lstlisting}[caption={}]
if(mKeyboard->isKeyDown(OIS::KC_ESCAPE))
    return false;
\end{lstlisting}

Nous devons ensuite mettre \`a jour la valeur de mMouvement, qui sera la distance parcourue par la cam\'era si une direction est choisie. Comme le nombre d'images par seconde est variable, nous utilisons la propri\'et\'e timeSinceLastFrame de l'\'ev\'enement, multipli\'ee par la vitesse de la cam\'era. Le produit de la vitesse par le temps \'ecoul\'e nous donne donc la distance parcourue.

J'ai aussi cr\'e\'e un vecteur dans lequel nous allons enregistrer les d\'eplacements \`a effectuer. En effet, on peut utiliser plusieurs touches en m\^eme temps, il faut donc additionner les directions demand\'ees, et les conserver pour d\'eplacer la cam\'era en une seule fois.


\begin{lstlisting}[caption={}]
Ogre::Vector3 deplacement = Ogre::Vector3::ZERO;
mMouvement = mVitesse * evt.timeSinceLastFrame;
\end{lstlisting}

Nous allons utiliser les fl\`eches du clavier et les touches Z, S, Q, D pour nous d\'eplacer ; j'ai aussi impl\'ement\'e les touches fl\'ech\'ees, qui sont une configuration alternative pour le d\'eplacement. Il faut donc v\'erifier si les touches qui nous int\'eressent sont enfonc\'ees :


\begin{lstlisting}[caption={}]
// La touche A d'un clavier QWERTY correspond au Q sur un AZERTY
if(mKeyboard->isKeyDown(OIS::KC_LEFT) || mKeyboard->isKeyDown(OIS::KC_A)) 
    deplacement.x -= mMouvement;

if(mKeyboard->isKeyDown(OIS::KC_RIGHT) || mKeyboard->isKeyDown(OIS::KC_D))
    deplacement.x += mMouvement;
    
// W correspond au Z du AZERTY
if(mKeyboard->isKeyDown(OIS::KC_UP) || mKeyboard->isKeyDown(OIS::KC_W)) 
    deplacement.z -= mMouvement;

if(mKeyboard->isKeyDown(OIS::KC_DOWN) || mKeyboard->isKeyDown(OIS::KC_S))
    deplacement.z += mMouvement;
\end{lstlisting}

Attention aux signes ! Vous devez respecter ce que nous avons vu dans le chapitre sur les d\'eplacements !\footnote{hein? de quoi parle t il?}

Le d\'eplacement de la cam\'era fonctionne, il ne manque plus que la rotation de celle-ci. Il nous suffit pour cela de r\'ecup\'erer le d\'eplacement relatif depuis la derni\`ere fois que la souris a boug\'e (depuis le dernier appel \`a frameRenderingQueued() donc).

Pour retrouver cette valeur, on passe successivement par les attributs suivants :
\begin{itemize}
\item le mouseState contenu dans l'objet souris, contenant diverses informations sur l'\'etat de la souris ;
\item l'axe que l'on d\'esire observer : ici, ce sera X ou Y pour le yaw ou le pitch ;
\item le d\'eplacement relatif de la souris suivant cet axe.
\end{itemize}

Maintenant, occupons-nous du mouvement de la souris. Pour r\'ecup\'erer le d\'eplacement de celle-ci, nous devons r\'ecup\'erer son \'etat, comme indiqu\'e dans le code suivant.


\begin{lstlisting}[caption={}]
const OIS::MouseState &mouseState = mMouse->getMouseState();
\end{lstlisting}

\`A partir de cette r\'ef\'erence on peut notamment r\'ecup\'erer le d\'eplacement de la souris depuis la derni\`ere image, en appelant l'axe X ou Y puis l'attribut rel.


\begin{lstlisting}[caption={}]
mRotationX = Degree(-mouseState.Y.rel * mVitesseRotation);
mRotationY = Degree(-mouseState.X.rel * mVitesseRotation);
\end{lstlisting}

Il faut particuli\`erement faire attention aux axes et aux signes. Je consid\`ere que mRotationX (respectivement mRotationY) correspond \`a la rotation autour de l'axe X (respectivement Y), c'est-\`a-dire lorsque je d\'eplace ma souris en avant ou en arri\`ere (respectivement \`a gauche ou \`a droite). Or, le d\'eplacement vers l'avant ou l'arri\`ere de la souris correspond \`a son axe Y, c'est pour \c{c}a que je demande l'axe Y de la souris pour trouver la rotation autour de X dans l'espace 3D.

On rajoute une multiplication par la vitesse de rotation voulue et on convertit le tout en degr\'es, sinon le mouvement est bien trop rapide.

Enfin, on appelle les m\'ethodes de rotation et de d\'eplacement de la cam\'era :

\begin{lstlisting}[caption={}]
mCamera->yaw(mRotationY);
mCamera->pitch(mRotationX);
mCamera->moveRelative(deplacement);
\end{lstlisting}

La derni\`ere ligne, comme vous le remarquez, d\'eplace la cam\'era par rapport \`a son rep\`ere local, ce qui \'evite de faire la transformation de la variable deplacement \`a la main. Il est aussi possible de demander un d\'eplacement par rapport au rep\`ere absolu avec Camera::move().


















\subsection{Et avec un noeud de sc\`ene ?}

J'en profite pour vous montrer comment on aurait proc\'ed\'e pour d\'eplacer un noeud de sc\`ene par exemple, qui utilise la m\'ethode translate().

Les deux lignes suivantes sont \'equivalentes :


\begin{lstlisting}[caption={}]
node->translate(deplacement, TS_LOCAL);
node->translate(node->getOrientation() * deplacement, TS_PARENT);
\end{lstlisting}


La premi\`ere ligne est tr\`es similaire \`a celle utilis\'ee pour la cam\'era, il suffit de pr\'eciser que l'on se d\'eplace par rapport au rep\`ere local du noeud de sc\`ene.

La seconde solution indique un d\'eplacement relatif au noeud parent, mais utilise le quaternion retourn\'e par la m\'ethode getOrientation() multipli\'e par le vecteur de d\'eplacement pour obtenir la direction souhait\'ee dans ce rep\`ere. En pratique, on utilisera seulement la premi\`ere ligne, plus courte et plus propre dans le code.







\subsection{}
Mini-TP : cr\'eer un interrupteur

Pour g\'erer un \'ev\'enement qui ne doit arriver qu'une fois lorsque la touche est appuy\'ee, il y a une pr\'ecaution suppl\'ementaire \`a prendre. Je vous ai dit plus haut que votre ordinateur ne retenait pas l'\'etat dans lequel se trouvait votre clavier ou votre souris \`a l'image pr\'ec\'edente. Cependant, rien ne nous emp\^eche de le faire nous-m\^emes !

\`A titre d'exemple, disons que l'on veut utiliser la touche T pour allumer et \'eteindre la lumi\`ere de notre sc\`ene. Il va donc falloir v\'erifier \`a chaque image si la touche T est enfonc\'ee, et si en plus ce n'\'etait pas d\'ej\`a le cas \`a l'image pr\'ec\'edente. Pour cela, on utilisera l'attribut mToucheAppuyee de notre classe InputListener.

Un indice: le SceneManager poss\`ede une m\'ethode getLight() qui permet de r\'ecup\'erer un pointeur sur une lumi\`ere \`a partir du nom de celle-ci...

\`A vos claviers ! La r\'eponse se trouve juste apr\`es.



\begin{lstlisting}[caption={Capture de l'\'etat ponctuel d'une touche}]
bool etatTouche = mKeyboard->isKeyDown(OIS::KC_T);
if(etatTouche && !mToucheAppuyee)
{
    Ogre::Light *light = mSceneMgr->getLight("lumiere1");
    light->setVisible(!light->isVisible());
}
mToucheAppuyee = etatTouche;
\end{lstlisting}





Tout d'abord, je r\'ecup\`ere l'\'etat actuel de ma touche T dans une variable locale. Je v\'erifie si la touche est actuellement enfonc\'ee et si elle ne l'\'etait pas d\'ej\`a \`a l'aide de l'attribut mToucheAppuyee. Si ma condition est v\'erifi\'ee, je r\'ecup\`ere ma lumi\`ere, et je change son \'etat (visible ou non).

Enfin, j'enregistre l'\'etat actuel de ma touche T dans mToucheAppuyee, en pr\'evision de la prochaine image!







\subsection{Code}


\begin{lstlisting}[caption={InputListener.h}]
#include "ExampleFrameListener.h"

//nous utiliserons OIS pour gérer les entrées de l'utilisateur
#include <OIS/OIS.h>

//la classe ExampleFrameListener ExampleApplication met déjà en oeuvre une gestion des entrées eet attend un ExampleFrameListener
//ExampleFrameListener s'occupe aussi de construire les objets nécessaires à l'écoute des entrées
class InputListener : public ExampleFrameListener
{
    public:
        InputListener(RenderWindow* win, Camera* cam, SceneManager *sceneMgr, 
                        bool bufferedKeys = false, bool bufferedMouse = false, 
                        bool bufferedJoy = false);
        
        //nous allons juste devoir redefinir frameRenderingQueued() pour implementer notre propre gestion des entrees a la place de celle prevue par ExampleFrameListener.
        virtual bool frameRenderingQueued(const FrameEvent& evt);

        private:
            Ogre::SceneManager *mSceneMgr;
            Ogre::Camera *mCamera;
            
            bool mContinuer;
            bool mToucheAppuyee;

            Ogre::Real mMouvement;
            Ogre::Real mVitesse;
            Ogre::Real mVitesseRotation;

            Ogre::Radian mRotationX;
            Ogre::Radian mRotationY;
};
\end{lstlisting}


Dans la m\'ethode InputListener::frameRenderingQueued, les touches press\'ees et les mouvements de la souris permettent soit le d\'eplacement de la t\^ete soit le d\'eplacement de la cam\'era selon les lignes qui en fin de m\'ethodes sont comment\'ees.
\begin{lstlisting}[caption={InputListener.cpp}]
#include <OIS/OIS.h>
#include "InputListener.h"



InputListener::InputListener(RenderWindow* win, Camera* cam, SceneManager *sceneMgr,
                                bool bufferedKeys, bool bufferedMouse, bool bufferedJoy) 
    : ExampleFrameListener(win, cam, bufferedKeys, bufferedMouse, bufferedJoy)

    {
        mCamera = cam;
        mSceneMgr = sceneMgr;
        mVitesse = 100;
        mVitesseRotation = 0.3;
        mToucheAppuyee = false;
    }



bool InputListener::frameRenderingQueued(const FrameEvent& evt)
{
    //ces lignes permettent la mise à jour de nos objets pour obtenir le nom des touches enfoncées
   if(mMouse){
       mMouse->capture();
   }
   if(mKeyboard){
       mKeyboard->capture();
   }
   
   
   if(mKeyboard->isKeyDown(OIS::KC_ESCAPE)){
       mContinuer = false;
   }
   else{
       mContinuer= true;
   }
   
   Ogre::Vector3 deplacement = Ogre::Vector3::ZERO;
   mMouvement = mVitesse * evt.timeSinceLastFrame;
   
   
    // La touche A d un clavier QWERTY correspond au Q sur un AZERTY
    if ( mKeyboard->isKeyDown(OIS::KC_LEFT) || mKeyboard->isKeyDown(OIS::KC_A) ){
        deplacement.x -= mMouvement;
    }
    
    if ( mKeyboard->isKeyDown(OIS::KC_RIGHT) || mKeyboard->isKeyDown(OIS::KC_D) ){
        deplacement.x += mMouvement;
    }
    

    // W correspond au Z du AZERTY
    if ( mKeyboard->isKeyDown(OIS::KC_UP) || mKeyboard->isKeyDown(OIS::KC_W)){
        deplacement.z -= mMouvement;
    }
    
    if ( mKeyboard->isKeyDown(OIS::KC_DOWN) || mKeyboard->isKeyDown(OIS::KC_S)){
        deplacement.z += mMouvement;
    }
    
    //*
    const OIS::MouseState &mouseState = mMouse->getMouseState();
    mRotationX = Degree(-mouseState.Y.rel * mVitesseRotation);
    mRotationY = Degree(-mouseState.X.rel * mVitesseRotation);
    
    //pour que la camera bouge selon les touches pressees et le mouvement de la souris
    //mCamera->yaw(mRotationY);
    //mCamera->pitch(mRotationX);        
    //mCamera->moveRelative(deplacement);
    
    //pour que le noeud "nodeTete" bouge selon les touches pressees et le mouvement de la souris
    mSceneMgr->getSceneNode("nodeTete")->yaw(mRotationY);
    mSceneMgr->getSceneNode("nodeTete")->pitch(mRotationX);
    mSceneMgr->getSceneNode("nodeTete")->translate(deplacement, Ogre::Node::TS_LOCAL);
    
    return mContinuer;
}
\end{lstlisting}


\begin{lstlisting}[caption={PremiereApplication.h}]
using namespace std;

#include <ExampleApplication.h>
#include <OgreMovableObject.h>

#include "InputListener.h"


class PremiereApplication : public ExampleApplication
{
    public:
        void createScene();
        void createCamera();
        void createViewports();

        void createFrameListener();
        
        void createLux(std::string, MovableObject *);
};
\end{lstlisting}


\begin{lstlisting}[caption={PremiereApplication.cpp}]
#include "PremiereApplication.h"



void PremiereApplication::createFrameListener()
{
    mFrameListener = new InputListener(mWindow, mCamera, mSceneMgr, false, false, false);
    
    //root est l'élément de base de l'application Ogre qui s'occupe notamment de gérer les frames listeners
    mRoot -> addFrameListener(mFrameListener);
}


void PremiereApplication::createScene()
{
    //creation d une entite
    Entity *head= mSceneMgr->createEntity("Tete", "ogrehead.mesh" );
    
    //creation d un noeud
    SceneNode *node= mSceneMgr->getRootSceneNode()->createChildSceneNode("nodeTete" , Vector3::ZERO, Quaternion::IDENTITY);
    
    node->yaw(Radian(Math::PI));
    node->yaw(Radian(Math::PI));

    //setPosition place le noeud aux coord passees en parametres
    Vector3 position = Vector3(30.0, 50.0, 0.0);
    node->setPosition(position);

    node->setPosition(30.0, 50.0, 0.0); 
    /*equivalent a
    Vector3 position = Vector3(30.0, 50.0, 0.0);
    node->setPosition(position);
    */

    //deplace le noeud par rapport a sa position actuelle
    node->translate(-30.0, 50.0, 0.0); //par defaut la trnslt se fait par rap a TS_WORLD
   
    //attachement de l entite au noeud
    node->attachObject(head);

    //creation d un plan
    Plane plan(Vector3::UNIT_Y, 0);

    //creation d un mesh cad l objet 3d visible ds la scene
    MeshManager::getSingleton().createPlane("sol",
                ResourceGroupManager::DEFAULT_RESOURCE_GROUP_NAME,
                plan, 500, 500, 10, 10, true, 1, 1, 1, Vector3::UNIT_Z); 

    //entite qui representera le plan
    Entity *ent= mSceneMgr->createEntity("EntiteSol", "sol");

    //ajout du materiau a l entite
    ent->setMaterialName("Examples/GrassFloor");//texture de pelouse
    /*les differents materiaux sont sous /media/materials/scritps, par ex:
    ent->setMaterialName("Examples/WaterStream");//texture d eau animee*/

    //creation d un noeud
    node = mSceneMgr->getRootSceneNode()->createChildSceneNode();
    node->attachObject(ent);

    createLux("ponctuelle", head);//lumiere ponctuelle
    //createLux("directionnelle", head);//lumiere directionnelle
    //createLux("spot", head);//lumiere projecteur

}

/*
cree une lumiere selon le parametre passe:
    createLux("ponctuelle"); -> lumiere ponctuelle
    createLux("directionnelle"); -> lumiere directionnelle
    createLux("spot"); -> lumiere projecteur

une lumiere noire est cree au debut de la methode

une ombre est cree en fin de methode
*/
void PremiereApplication::createLux(std::string prmLightType, MovableObject * prmEnt)
{
    //application d une couleur noire
    mSceneMgr->setAmbientLight(ColourValue(0.0, 0.0, 0.0)); 

    //definition d une lumiere 
    Light *light= mSceneMgr->createLight("lumiere1");

    if (prmLightType == "ponctuelle")
    {
        //definition du type de lumiere
        light->setType(Light::LT_POINT);//lumiere ponctuelle

        //definition de la position de la lumiere
        light->setPosition(-100, 200, 100);
    }
    else if (prmLightType == "directionnelle")
    {
        light->setType(Light::LT_DIRECTIONAL);//lumiere directionnelle
        light->setDirection(10.0, -20.0, -5);//vecteur directeur de la lumiere directionnelle

        //definition de la position de la lumiere
        light->setPosition(-100, 200, 100);
    }
    else
    {
        light->setType(Light::LT_SPOTLIGHT);//lumiere directionnelle
        light->setDirection(0.0, -1, 1);//vecteur directeur de la lumiere directionnelle
        light->setSpotlightRange(Degree(30), Degree(60), 1.0);
    }

    //definition des couleur des lumieres diffuse
    light->setDiffuseColour(1.0, 0.7, 0.1);
    //et speculaire
    light->setSpecularColour(1.0, 0.7, 0.1);

    //ombre
    //activation de la projection des ombres
    light->setCastShadows(true);
    prmEnt->setCastShadows(true);

    //activation des ombres
    mSceneMgr->setShadowTechnique(Ogre::SHADOWTYPE_STENCIL_ADDITIVE);
}

/*definit la position de notre point de vue*/
void PremiereApplication::createCamera()
{
    //creation de la camera
    mCamera = mSceneMgr->createCamera("Ma Camera");

    //position de la camera
    mCamera->setPosition(Vector3(-100.0, 150.0, 200.0));

    //permet de determiner le point de la scene que regarde notre camera
    mCamera->lookAt(Vector3(0.0, 100.0, 0.0));

    //definition des distances de near clip et de far clip, qui
    //sont les distances minimale et maximale auxquelles doit se
    //trouver un objet pour être affichr à l'écran.
    mCamera->setNearClipDistance(1);
    mCamera->setFarClipDistance(1000);
}

void PremiereApplication::createViewports()
{
    //la creation du Viewport, appelee par la fenetre et prenant en parametre la
    //camera concernee, le premier parametre est la camera de laqll le contenu
    //du viewport sera rendu, ce paramatre est le seul obligatoire
    Viewport *vue = mWindow->addViewport(mCamera);
    //Viewport *vue = mWindow->addViewport(mCamera, 0, 0, 0, 0.8, 0.8);

    //Grace a ce Viewport nouvellement cree, nous allons faire coincider
    //le rapport largeur / hauteur de notre camera avec celui du
    //Viewport, pour avoir une image non deformee
    mCamera->setAspectRatio(Real(vue->getActualWidth()) /  Real(vue->getActualHeight()));

    //on definit ici la couleur de fond
    vue->setBackgroundColour(ColourValue(0.0, 0.0, 1.0));     //bleu
    //vue->setBackgroundColour(ColourValue(0.980, 0.502, 0.447)); //saumon

   // creation d'un viewport dans le coin bas gauche
   //les parametres autres que le premier sont obligatoires pour la definition
   //de plusieurs viewport
   Viewport* vue2 = mWindow->addViewport(mCamera, 1, 0, 0.8, 0.2, 0.2);
   vue2->setBackgroundColour(ColourValue(0.561, 0.737, 0.561 ));  //darkseagreen
}


\end{lstlisting}






Puisque nous avons ajout\'e un fichier source il est n\'ecessaire de modifier le fichier CMakeLists.txt tel que suit:

\begin{lstlisting}[caption={CMakeLists.txt}]
project(helloworld)
cmake_minimum_required(VERSION 2.6)

set(CMAKE_MODULE_PATH "/usr/share/OGRE/cmake/modules")


# Il faut indiquer a cmake ou se trouvent les includes en question
#include_directories ("/home/adkoba/Workspace/ogre_src_v1-8-1/Samples/Common/include")
include_directories ("include")

# Bien sur, pour compiler Ogre, il faut le chercher, et definir le repertoire contenant les includes.
find_package(OGRE REQUIRED)
include_directories (${OGRE_INCLUDE_DIRS})

# L'exemple depend aussi de OIS, une lib pour gerer la souris, clavier, joystick...
find_package(OIS REQUIRED)

# On definit les sources qu'on veut compiler
SET( SOURCES
  src/InputListener.cpp  
  src/PremiereApplication.cpp
  src/main.cpp
)

# On les compile
add_executable (
  premiereapp ${SOURCES}
)

# Et pour finir, on lie l'excutable avec les librairies que find_package nous a gentillement trouve.
target_link_libraries(premiereapp ${OGRE_LIBRARY} ${OIS_LIBRARY} -lboost_system)

set( RESOURCES_FILE
  media/
  plugins/
  resources/ogre.cfg
  resources/plugins.cfg
  resources/resources.cfg
)

# do the copying
foreach( file_i ${RESOURCES_FILE})
    add_custom_command(
      TARGET premiereapp
      POST_BUILD
      COMMAND cp -R ${CMAKE_SOURCE_DIR}/${file_i} ${CMAKE_BINARY_DIR}
      COMMENT "copy file ${file_i}"
      )
endforeach( file_i )
\end{lstlisting}

Comme nous pouvons le voir nous avons juste modifi\'e la liste des sources du projet












\section{Avec buffer, c'est plus simple ?}

La m\'ethode pr\'esent\'ee ci-avant pour contr\^oler si une touche vient ou non d'\^etre appuy\'ee fonctionne mais est peu pratique si on doit surveiller quinze touches.

Heureusement, OIS a pens\'e \`a tout, nous allons donc voir une autre fa\c{c}on de faire ce que l'on vient juste d'\'ecrire. On va commencer comme pr\'ec\'edemment par le d\'eplacement de la cam\'era, puis on verra comment g\'erer notre interrupteur.


\subsection{Mise en place}


Nous allons commencer par activer l'utilisation du buffer pour la souris et le clavier lors de la construction de notre frame listener. Il suffit pour cela de mettre les param\`etres correspondants \`a true.


\begin{lstlisting}[caption={Activation du buffer pour la souris et le clavier}]
void PremiereApplication::createFrameListener()
{
    mFrameListener= new InputListener(mWindow, mCamera, mSceneMgr, true, true, false);    mRoot->addFrameListener(mFrameListener);
}
\end{lstlisting}


C'est quasiment tout , il va simplement falloir rajouter deux petites lignes dans le constructeur pour que tout soit pr\^et.

Afin d'utiliser le buffer, il faut fournir un objet (un ''\'ecouteur'' d\'erivant d'une des classes OIS::***Listener selon le p\'eriph\'erique \`a \'ecouter) qui sera celui qui recevra les \'ev\'enements du type ''cette touche vient d'\^etre appuy\'ee, que dois-je faire ?''. Pour cela, OIS fournit une m\'ethode pour chacun de trois p\'eriph\'eriques d'entr\'ee (clavier, souris, joystick) :


\begin{itemize}
\item virtual void OIS::Mouse::setEventCallback(OIS::MouseListener* mouseListener);
\item  virtual void OIS::Keyboard::setEventCallback(OIS::KeyListener* keyListener);
\item  virtual void OIS::JoyStick::setEventCallback(OIS::JoyStickListener* joystickListener);
\end{itemize}

Cette m\'ethode prend donc en param\`etre un pointeur sur un listener du p\'eriph\'erique que vous voulez utiliser. Nous allons donc rajouter deux classes m\`eres \`a notre InputListener : OIS::MouseListener et OIS::KeyListener. Modifiez donc la d\'eclaration de la classe InputListener :

\begin{lstlisting}[caption={Classes m\`eres pour gestion des Listeners}]
class InputListener : public ExampleFrameListener, OIS::KeyListener, OIS::MouseListener
\end{lstlisting}

Dans le constructeur, vous pouvez maintenant ins\'erer les deux lignes suivantes (mMouse et mKeyboard sont d\'eclar\'ees dans ExampleFrameListener) :

\begin{lstlisting}[caption={Enregistrement des listener}]
mMouse->setEventCallback(this);
mKeyboard->setEventCallback(this);
\end{lstlisting}

Vous ne pouvez enregistrer qu'un seul \'ecouteur par p\'eriph\'erique d'entr\'ee. En cas d'appels multiples \`a la m\'ethode setEventCallback(), c'est le dernier appel qui d\'efinit l'\'ecouteur \`a utiliser. Pour que diff\'erents objets re\c{c}oivent les \'ev\'enements, il faudra donc les redistribuer \`a partir de l'\'ecouteur receveur.

Au chapitre des modifications, supprimez les anciens attributs d'InputListener et mettez ceux-ci :

\begin{lstlisting}[caption={Attributs d'InputListener}]
private:
    Ogre::SceneManager *mSceneMgr;
    bool mContinuer;
    Ogre::Vector3 mMouvement;
    Ogre::Real mVitesse;
    Ogre::Real mVitesseRotation;
\end{lstlisting}

En initialisant ces attributs, votre constructeur devrait maintenant ressembler \`a ceci :

\begin{lstlisting}[caption={Constructeur d'InputListener}]
InputListener(RenderWindow* win, Camera* cam, SceneManager *sceneMgr, bool bufferedKeys = false, bool bufferedMouse = false, bool bufferedJoy = false )   : ExampleFrameListener(win, cam, bufferedKeys, bufferedMouse, bufferedJoy)
{
    mSceneMgr = sceneMgr;
    mContinuer = true;  //permettra d'enregistrer l'appui sur la touche Echap
    mMouvement = Ogre::Vector3::ZERO;//vecteur de la direction ds lqulle se d\'eplacer
    mVitesse = 100;
    mVitesseRotation = 0.2;//facteur multiplicatif pr ajuster la vitesse de la cam 
    mMouse->setEventCallback(this);
    mKeyboard->setEventCallback(this);
}
\end{lstlisting}

L'attribut mContinuer permettra d'enregistrer l'appui sur la touche Echap, mMouvement sera le vecteur de la direction dans laquelle on doit se d\'eplacer et mVitesseRotation un facteur multiplicatif permettant d'ajuster la vitesse de rotation de la cam\'era.

On met \`a jour la valeur de retour de la m\'ethode frameRenderingQueued().

\begin{lstlisting}[caption={}]
bool InputListener::frameRenderingQueued(const Ogre::FrameEvent& evt)
{
    if(mMouse)
        mMouse->capture();
    if(mKeyboard)
        mKeyboard->capture();

    return mContinuer;
}
\end{lstlisting}

Enfin, il y a des m\'ethodes virtuelles pures \`a r\'eimpl\'ementer dans notre classe. Ces m\'ethodes seront appel\'ees lors de l'\'ev\'enement correspondant sur le clavier (touche enfonc\'ee ou rel\^ach\'ee) ou sur la souris (bouton appuy\'e ou rel\^ach\'e, d\'eplacement). De m\^eme que les m\'ethodes des frame listeners d'Ogre, elles renvoient un bool\'een que l'on utilisera pour savoir si l'on doit interrompre le programme.

Ajoutons donc les prototypes dans notre header et un simple retour de valeur dans le corps des m\'ethodes pour commencer.

Contrairement aux m\'ethodes des frame listeners, la valeur de retour ne d\'etermine pas si l'on doit continuer ou non l'ex\'ecution. C'est pour cela que l'on devra passer par l'attribut mContinuer pour surveiller l'appui sur la touche Echap.


\begin{lstlisting}[caption={M\'ethodes virtuelles appel\'ees lors d'un \'ev\`enement sur un p\'eriph\'erique}]
bool InputListener::mouseMoved(const OIS::MouseEvent &e)
{
    return true;
}

bool InputListener::mousePressed(const OIS::MouseEvent &e, OIS::MouseButtonID id)
{
    return true;
}

bool InputListener::mouseReleased(const OIS::MouseEvent &e, OIS::MouseButtonID id)
{
    return true;
}

bool InputListener::keyPressed(const OIS::KeyEvent &e)
{
    return true;
}

bool InputListener::keyReleased(const OIS::KeyEvent &e)
{
    return true;
}
\end{lstlisting}

On commence par impl\'ementer la touche Echap. Si elle est appuy\'ee, on passe simplement l'attribut mContinuer \`a false.

\begin{lstlisting}[caption={Impl\'ementation de l'appuie sur ECHAP}]
bool InputListener::keyPressed(const OIS::KeyEvent &e)
{
    switch(e.key)
    {
        case OIS::KC_ESCAPE:
            mContinuer = false;
            break;
    }

    return mContinuer;
}
\end{lstlisting}

On g\`ere ensuite l'appui sur les touches de d\'eplacement en modifiant les composantes de mMouvement en fonction de la touche. On va aussi multiplier la vitesse de d\'eplacement par deux lorsque l'on appuie sur la touche majuscule gauche.

\begin{lstlisting}[caption={Impl\'ementation de l'appuie sur les touches de d\'eplacement}]
bool InputListener::keyPressed(const OIS::KeyEvent &e)
{
    switch(e.key)
    {
        case OIS::KC_ESCAPE:
            mContinuer = false;
            break;
        case OIS::KC_W:
            mMouvement.z -= 1;
            break;
        case OIS::KC_S:
            mMouvement.z += 1;
            break;
        case OIS::KC_A:
            mMouvement.x -= 1;
            break;
        case OIS::KC_D:
            mMouvement.x += 1;
            break;
        case OIS::KC_LSHIFT:
            mVitesse *= 2;
            break;
    }

    return mContinuer;
}
\end{lstlisting}

Enfin, dans la m\'ethode keyReleased, on va ''retirer'' la composante que l'on ajoute lors de l'appui sur une touche. Le code est donc semblable, seuls les signes changent.

\begin{lstlisting}[caption={}]
bool InputListener::keyReleased(const OIS::KeyEvent &e)
{
    switch(e.key)
    {
        case OIS::KC_W:
            mMouvement.z += 1;
            break;
        case OIS::KC_S:
            mMouvement.z -= 1;
            break;
        case OIS::KC_A:
            mMouvement.x += 1;
            break;
        case OIS::KC_D:
            mMouvement.x -= 1;
            break;
        case OIS::KC_LSHIFT:
            mVitesse /= 2;
            break;
    }

    return true;
}
\end{lstlisting}

Maintenant que l'on g\`ere correctement l'\'evolution de nos variables de mouvement et de vitesse de d\'eplacement, il faut \'ecrire le d\'eplacement de la cam\'era dans la m\'ethode frameRenderingQueued().

\begin{lstlisting}[caption={Impl\'ementation du d\'eplacement de la cam\'era dans la m\'ethode frameRenderingQueued}]
virtual bool frameRenderingQueued(const FrameEvent& evt)
{
    if(mMouse)
        mMouse->capture();

    if(mKeyboard)
        mKeyboard->capture();

    Ogre::Vector3 deplacement = Ogre::Vector3::ZERO;
    deplacement = mMouvement * mVitesse * evt.timeSinceLastFrame;
    mCamera->moveRelative(deplacement);

    return mContinuer;
}
\end{lstlisting}

Pour la rotation de la cam\'era, tout se passe dans la m\'ethode mouseMoved(), dont l'\'ev\'enement re\c{c}u en param\`etre contient l'\'etat de la souris, permettant comme pr\'ec\'edemment de retrouver le d\'eplacement relatif de la souris.

On multiplie cette valeur relative par la vitesse de rotation, on fait attention aux signes, et voici ce qu'on obtient :

\begin{lstlisting}[caption={Impl\'ementation de la rotation de la cam\'era dans la m\'ethode mouseMoved}]
bool InputListener::mouseMoved(const OIS::MouseEvent &e)
{
    mCamera->yaw(Ogre::Degree(-mVitesseRotation * e.state.X.rel));
    mCamera->pitch(Ogre::Degree(-mVitesseRotation * e.state.Y.rel));

    return true;
}
\end{lstlisting}

Vous pouvez maintenant compiler et ex\'ecuter votre application ; les commandes de d\'eplacement sont maintenant g\'er\'ees enti\`erement par notre classe InputListener, n'h\'esitez donc pas \`a adapter les variables initialis\'ees dans le constructeur si vous voulez acc\'el\'erer ou ralentir les mouvements par exemple.\newline

Cette m\'ethode de gestion des entr\'ees permet donc de g\'erer plus facilement l'appui ponctuel sur une touche, tout en conservant une gestion simple des touches qui peuvent rester enfonc\'ees (pour le d\'eplacement ici).

Gardez cependant bien \`a l'esprit qu'\textbf{il ne peut y avoir qu'un seul \'ecouteur par p\'eriph\'erique d'entr\'ee} et qu'il faudra donc penser \`a rapporter l'appui sur les touches \`a des \'ecouteurs annexes lorsque votre application grossira, sinon vous allez vite vous retrouver avec un code lourd et mal organis\'e.\newline

Dans ce chapitre, nous avons vu comment g\'erer nous-m\^emes les entr\'ees de l'utilisateur avec la biblioth\`eque OIS, ainsi que le principe des frame listeners, qui nous ont ici \'et\'e bien utiles alors que nous n'avons pas encore vu le fonctionnement de la boucle de rendu.

Le prochain chapitre promet d'\^etre int\'eressant : nous allons en effet utiliser le module de terrain d'Ogre qui a \'et\'e refait dans la version 1.7 et qui offre une gestion beaucoup plus optimis\'ee des terrains par rapport aux versions pr\'ec\'edentes. Je n'en dis pas plus, on se retrouve de l'autre c\^ot\'e.


\subsection{Code}

Le CMakeLists.txt n'est pas modifi\'e par rapport \`a la gestion des entr\'ees sans buffer.


\begin{lstlisting}[caption={PremiereApplication::createFrameListener()}]
void PremiereApplication::createFrameListener()
{
    //activation du buffer pour la souris et le clavier
    mFrameListener = new InputListener(mWindow, mCamera, mSceneMgr, true, true, true);
    
    //root est l'élément de base de l'application Ogre qui s'occupe notamment de gérer les frames listeners
    mRoot -> addFrameListener(mFrameListener);
}

\end{lstlisting}














\begin{lstlisting}[caption={InputListener.h}]
#include "ExampleFrameListener.h"

//nous utiliserons OIS pour gérer les entrées de l'utilisateur
#include <OIS/OIS.h>

//la classe ExampleFrameListener ExampleApplication met déjà en oeuvre une gestion des entrées eet attend un ExampleFrameListener
//ExampleFrameListener s'occupe aussi de construire les objets nécessaires à l'écoute des entrées
class InputListener : public ExampleFrameListener, OIS::KeyListener, OIS::MouseListener
{
    public:
        InputListener(RenderWindow* win, Camera* cam, SceneManager *sceneMgr, 
                        bool bufferedKeys = false, bool bufferedMouse = false, 
                        bool bufferedJoy = false);
        
        //nous allons juste devoir redefinir frameRenderingQueued() pour implementer notre propre gestion des entrees a la place de celle prevue par ExampleFrameListener.
        virtual bool frameRenderingQueued(const FrameEvent& evt);
        
        //les 5 methodes suivantes sont des methodes virtuelles pures a reimplementer
        //ces methodes seront appelees lors de levenmnt correspondant, leur valeur de retour ne dit pas si on doit continuer ou pas, d'ou l'interet de mContinuer
        bool mouseMoved(const OIS::MouseEvent &e);      //evenmnt: la souris a bouge
        bool mousePressed(const OIS::MouseEvent &e, OIS::MouseButtonID id);//evenmnt: un bouton de la souris a ete presse
        bool mouseReleased(const OIS::MouseEvent &e, OIS::MouseButtonID id);//evenmnt: un bouton de la souris a ete relache 
        bool keyPressed(const OIS::KeyEvent &e);//evenmnt: un bouton a ete presse 
        bool keyReleased(const OIS::KeyEvent &e);//evenmnt: un bouton a ete relache 
        

        private:
            Ogre::SceneManager *mSceneMgr;
            Ogre::Camera *mCamera;
            
            //permettra d'enregistrer l'appuie sur ECHAP
            bool mContinuer;

            //vecteur de la direction ds lqll on doit bouger
            Ogre::Vector3 mMouvement;
            Ogre::Real mVitesse;
            
            //pour ajuster la vitesse de rotation de la souris
            Ogre::Real mVitesseRotation;

            Ogre::Radian mRotationX;
            Ogre::Radian mRotationY;
};

\end{lstlisting}













\begin{lstlisting}[caption={InputListener.cpp}]



#include <OIS/OIS.h>
#include "InputListener.h"



InputListener::InputListener(RenderWindow* win, Camera* cam, SceneManager *sceneMgr,
                                bool bufferedKeys, bool bufferedMouse, bool bufferedJoy) 
    : ExampleFrameListener(win, cam, bufferedKeys, bufferedMouse, bufferedJoy)
    {
        mCamera = cam;
        mSceneMgr = sceneMgr;
        
        mContinuer = true;//permettra l enregistrmnt de l appuie sur ECHAP
        mMouvement = Ogre::Vector3::ZERO;//sera le vct de la direection ds laqll on doit se deplacer
        
        mVitesse = 100;
        mVitesseRotation = 0.3;//fcteur multiplicatif pour ajuster la rotation de la camera
       
        //enregistrement d un ecouteur pour la souris et pour le clavier
        mMouse->setEventCallback(this);
        mKeyboard->setEventCallback(this);
    }



bool InputListener::frameRenderingQueued(const FrameEvent& evt)
{
    //ces lignes permettent la mise à jour de nos objets pour obtenir le nom des touches enfoncées
   if(mMouse){
       mMouse->capture();
   }
   if(mKeyboard){
       mKeyboard->capture();
   }
   
   //gestion du mouvement de la camera
   Ogre::Vector3 deplacement = Ogre::Vector3::ZERO;
   deplacement = mMouvement * mVitesse * evt.timeSinceLastFrame;
   mCamera->moveRelative(deplacement);
    
   return mContinuer;
}



//les 5 methodes suivantes sont des methodes virtuelles pures a reimplementer
//si la souris a bouge
bool InputListener::mouseMoved(const OIS::MouseEvent &e)
{    
    mCamera->yaw(Ogre::Degree(-mVitesseRotation * e.state.X.rel));
    mCamera->pitch(Ogre::Degree(-mVitesseRotation * e.state.Y.rel));
    
    return true;
}

//si une touche de la souris est pressee
bool InputListener::mousePressed(const OIS::MouseEvent &e, OIS::MouseButtonID id)
{
    return true;
}

//si une touche de la souris est relachee 
bool InputListener::mouseReleased(const OIS::MouseEvent &e, OIS::MouseButtonID id)
{
    return true;
}

//si une touche est pressee
bool InputListener::keyPressed(const OIS::KeyEvent &e)
{
    switch(e.key)
    {
        case OIS::KC_ESCAPE:
            mContinuer = false;
            break;
        case OIS::KC_W:
            mMouvement.z -= 1;
            break;
        case OIS::KC_S:
            mMouvement.z += 1;
            break;
        case OIS::KC_A:
            mMouvement.x -= 1;
            break;
        case OIS::KC_D:
            mMouvement.x += 1;
            break;
        case OIS::KC_LSHIFT:
            mVitesse *= 2;
            break;
    }
    
    return true;
}

//si une touche est relachee
bool InputListener::keyReleased(const OIS::KeyEvent &e)
{
    
    switch(e.key)
    {
        case OIS::KC_W:
            mMouvement.z += 1;
            break;
        case OIS::KC_S:
            mMouvement.z -= 1;
            break;
        case OIS::KC_A:
            mMouvement.x += 1;
            break;
        case OIS::KC_D:
            mMouvement.x -= 1;
            break;
        case OIS::KC_LSHIFT:
            mVitesse /= 2;
            break;
    } 
    
    return true;
}
\end{lstlisting}

%---------------------------------------------------------------------------------------------------------------

\chapter{Gardez les pieds sur Terre}
Sans outil particulier, cr\'eer un terrain pourrait relever d'un travail de longue haleine si l'on devait cr\'eer le mesh dans les moindres d\'etails pour l'ins\'erer ensuite \`a la sc\`ene. La gestion d'une telle entit\'e serait aussi relativement complexe, puisqu'il doit pouvoir interagir avec tous les objets susceptibles de se d\'eplacer dans la sc\`ene.

Heureusement, plut\^ot que de placer notre terrain comme une simple entit\'e dans la sc\`ene, Ogre propose de passer par une classe Terrain qui, comme son nom l'indique si bien, permet de g\'erer des terrains dans la sc\`ene.

Le terrain n'est g\'en\'eralement pas seul et le ciel joue un r\^ole important pour le r\'ealisme de la sc\`ene. L\`a encore, quelques outils bienvenus offrent diff\'erentes solutions pour obtenir un r\'esultat convaincant.



\section{Cr\'eer un terrain}


\subsection{Pr\'eparation}

Avant de commencer, il va falloir modifier un peu notre projet avec de nouvelles d\'ependances pour que les terrains soient utilisables.

Dans l'\'editeur de lien, ajoutez le fichier librairie OgreTerrain.lib (ou bien OgreTerrain\_d.lib si vous compilez en debug).

Il faut maintenant ajouter un fichier en-t\^ete dans notre classe :

\begin{lstlisting}[caption={Ajout du fichier d'ent\^ete pour la gestion des terrains}]
#include <Ogre/Terrain/OgreTerrain.h>
\end{lstlisting}

Enfin, si vous ne les avez pas d\'ej\`a plac\'ees dans le r\'epertoire de votre ex\'ecutable, copiez ces 2 DLL dans le dossier correspondant \`a votre configuration (les DLL de debug se terminent aussi par ''\_d'') :

\begin{itemize}
\item OgreTerrain.dll
\item OgrePaging.dll
\end{itemize}




\subsection{Quelques param\`etres \`a r\'egler}

\subsubsection{Ajout d'attributs obligatoires pour le terrain}

Commen\c{c}ons par ajouter deux attributs dans notre classe PremiereApplication.

\begin{itemize}
\item un objet Terrain, qui g\'erera les propri\'et\'es de notre terrain,
\item un objet TerrainGlobalOptions\index{Terrain!TerrainGlobalOptions}, qui d\'efinit des propri\'et\'es g\'en\'erales pour les terrains dans notre application, notamment l'\'eclairage.\newline
\end{itemize}

La pr\'esence de cet objet TerrainGlobalOptions\index{TerrainGlobalOptions} est obligatoire lorsque vous voulez utiliser les terrains dans votre sc\`ene, sous peine d'erreur \`a l'ex\'ecution.

\begin{lstlisting}[caption={Attributs pour la gestion de terrain}]
Ogre::Terrain *mTerrain;
Ogre::TerrainGlobalOptions *mGlobals;
\end{lstlisting}

Au d\'ebut de la m\'ethode createScene(), nous allons r\'egler quelques param\`etres pour que notre terrain apparaisse sous son meilleur jour.

\subsubsection{R\'eglage de la cam\'era}
Premi\`erement, je vous conseille d'augmenter la distance de vue de la cam\'era et de la positionner en hauteur :

\begin{lstlisting}[caption={R\'eglage de la cam\'era}]
mCamera->setFarClipDistance(20000);
mCamera->setPosition(0, 500, 0);
\end{lstlisting}

Il est aussi possible de r\'egler la distance de vue \`a l'infini\index{cam\'era!distance de vue \`a l'infini}, en mettant 0 comme param\`etre. Cependant, cela d\'epend de votre machine, il faut donc v\'erifier si vous pouvez vous le permettre avant de l'appliquer.

\begin{lstlisting}[caption={V\'erification et r\'eglages de vue \`a l'infini}]
if (mRoot->getRenderSystem()->getCapabilities()->hasCapability(Ogre::RSC\_INFINITE\_FAR\_PLANE))
    mCamera->setFarClipDistance(0);
\end{lstlisting}




\subsubsection{D\'efinition des \'eclairages}
Plut\^ot que de g\'erer l'\'eclairage de fa\c{c}on distincte, on peut d\'efinir un \'eclairage d'ambiance particulier pour le terrain. Pour cela, il suffit de d\'efinir une lumi\`ere directionnelle avec les param\`etres qui vous paraissent adapt\'es \`a votre environnement. 

Ici cette lumi\`ere est ajout\'ee en tant qu'attribut de la classe, car je l'utiliserai dans une autre fonction.

\begin{lstlisting}[caption={D\'efinition de l'\'eclairage pour le terrain}]
Ogre::Vector3 lightdir(0.55f, -0.3f, 0.75f);
mLight = mSceneMgr->createLight(''terrainLight'');
mLight->setType(Ogre::Light::LT\_DIRECTIONAL);
mLight->setDirection(lightdir);
mLight->setDiffuseColour(Ogre::ColourValue::White);
mLight->setSpecularColour(Ogre::ColourValue(0.4f, 0.4f, 0.4f));
\end{lstlisting}




\subsubsection{D\'efinition d'une m\'ethode pour la prise en charge de l'initialisation du terrain}

Nous allons d\'efinir une m\'ethode createTerrain(), qui sera appel\'ee dans createScene() et qui prendra en charge toute l'initialisation du terrain.

\`A l'int\'erieur de celle-ci, commencez par cr\'eer le TerrainGlobalOptions :

\begin{lstlisting}[caption={M\'ethode pour la prise en charge de l'initialisation du terrain}]
void PremiereApplication::createTerrain()
{
    mGlobals = OGRE_NEW Ogre::TerrainGlobalOptions();
    mGlobals->setMaxPixelError(8);
}
\end{lstlisting}

mGlobals est le premier objet d'Ogre que nous allons cr\'eer nous-m\^emes, sans passer par l'interm\'ediaire du Scene Manager.

Le moteur fournit diff\'erentes macros comme OGRE\_NEW\index{OGRE\_NEW} pour allouer l'espace en m\'emoire lorsque vous instanciez une classe d'Ogre. De fa\c{c}on g\'en\'erale, les d\'eveloppeurs conseillent d'utiliser ces macros lorsque vous devez faire de l'allocation dynamique sur les objets du moteur qui d\'erivent de Ogre::AllocatedObject. Vous pouvez voir la hi\'erarchie des classes dans la documentation.

En ce qui concerne la destruction des objets, l'utilisation de l'op\'erateur OGRE\_NEW implique l'utilisation de l'op\'erateur OGRE\_DELETE\index{OGRE\_DELETE} pour lib\'erer l'espace m\'emoire.

La seconde ligne appelle la m\'ethode setMaxPixelError()\index{setMaxPixelError()} qui donne la pr\'ecision avec laquelle le terrain est rendu. La valeur indiqu\'ee est l'erreur tol\'er\'ee, en pixels, pour l'affichage du terrain. Plus la valeur est faible, plus le terrain correspond au mod\`ele donn\'e, plus elle est forte et plus il sera impr\'ecis, en donnant l'impression d'un terrain nivel\'e.

On applique maintenant les r\'eglages concernant l'\'eclairage \`a nos options globales.

\begin{lstlisting}[caption={Application des r\'eglages}]
mGlobals->setLightMapDirection(mLight->getDerivedDirection());
mGlobals->setCompositeMapDistance(3000);
mGlobals->setCompositeMapAmbient(mSceneMgr->getAmbientLight());
mGlobals->setCompositeMapDiffuse(mLight->getDiffuseColour());
\end{lstlisting}




\subsection{Le terrain}


Maintenant, passons \`a la cr\'eation du terrain lui-m\^eme. Comme pour les options du terrain, nous ne passons pas par une m\'ethode du Scene Manager pour cr\'eer le terrain, mais par le constructeur de la classe, qui prend tout de m\^eme en param\`etre le Scene Manager de votre sc\`ene.

\begin{lstlisting}[caption={Cr\'eation du terrain}]
mTerrain = OGRE_NEW Ogre::Terrain(mSceneMgr);
\end{lstlisting}

Il faut \`a pr\'esent d\'efinir pr\'ecis\'ement les param\`etres de notre terrain :

\begin{itemize}
\item son relief, 
\item sa taille,
\item ses textures.
\end{itemize}

Il est temps que je vous parle des heightmaps pour la mod\'elisation du terrain !


\subsubsection{Les heightmaps}

Une heightmaps\index{heightmaps} est une image qui contient des informations de relief. On les utilise pour stocker le relief d'un terrain, mais aussi pour connaitre le relief d'une texture, afin de donner l'impression de relief sur un mesh alors qu'en r\'ealit\'e il est plat ou simplement lisse (c'est le principe du bump mapping\index{bump mapping}).

Cette m\'ethode a le principal avantage d'\^etre tr\`es l\'eg\`ere en terme de stockage, puisque l'on a simplement une image \`a la place d'un mod\`ele 3D complet qui prendrait beaucoup plus d'espace \`a stocker.

Voici l'image que nous allons utiliser pour notre terrain :
Image utilisateur

Comme vous le voyez c'est une simple image en noir et blanc, et pourtant cela suffit amplement !

En effet, si l'on utilise uniquement des niveaux de gris dans une image, chaque pixel peut prendre 256 valeurs, 0 correspondant au noir et \`a la hauteur la plus faible, 255 au blanc et \`a la plus forte altitude. On a donc 256 altitudes possibles pour notre terrain, ce qui est tout \`a fait honn\^ete et suffit \`a la majorit\'e des cas.

En r\'ealit\'e, chaque pixel poss\`ede trois valeurs, correspondant \`a la quantit\'e de rouge, de vert et de bleu, chacune de ces valeurs allant de 0 \`a 255. Or pour les niveaux de gris, ces trois valeurs doivent \^etre identiques, ce qui laisse 255 triplets de valeurs : (0, 0, 0) pour le noir, puis les niveaux de gris et enfin (255, 255, 255) pour le blanc).

Lors de la cr\'eation d'un fichier heightmap, on fait en sorte que le point le plus haut de notre carte soit blanc et que le point le plus bas soit noir, afin d'utiliser toute la plage de valeurs disponibles dans la carte et \'eviter les d\'enivellations peu naturelles.

Pour charger un fichier heightmap, on passe par un objet Image qui va chercher le nom du fichier que vous voulez dans les ressources d\'ej\`a charg\'ees. J'utilise le fichier terrain.png, que vous pouvez trouver dans ''OgreSDK.media.materials.textures''.\footnote{Comment ecrire un chemin avec des Slashs?}

\begin{lstlisting}[caption={Chargement du fichier heightmap}]
Ogre::Image img;
img.load(''terrain.png'', Ogre::ResourceGroupManager::DEFAULT_RESOURCE_GROUP_NAME);
\end{lstlisting}






\subsubsection{Les param\`etres g\'eom\'etriques}


Pour fournir toutes les informations dont le terrain a besoin pour \^etre g\'en\'er\'e, on utilise sa m\'ethode prepare()\index{prepare()}\index{Terrain!prepare()} qui prend en param\`etre un Terrain::ImportData\index{ImportData}\index{Terrain!ImportData}, qui est en gros une classe contenant l'ensemble des param\`etres \`a fournir au terrain. On va donc commencer par cr\'eer cet objet :

\begin{lstlisting}[caption={Cr\'eation de l'objet ImportData pour la d\'efinition des param\`etres \`a fournir au terrain}]
Ogre::Terrain::ImportData imp;
imp.inputImage = &img;  //recuperation de l'image
imp.terrainSize = img.getWidth();  //recuperation de la taille de l'image
imp.worldSize = 8000;  //indique la taille du terrain
imp.inputScale = 600;  //echelle adoptee pour l'altitude du terrain
imp.minBatchSize = 33; //taille min du batch pour le terrain
imp.maxBatchSize = 65; //taille max du batch pour le terrain
\end{lstlisting}

On commence par r\'ecup\'erer l'image et sa taille avec les lignes 2 et 3. \textbf{\'Etant donn\'e que les terrains sont carr\'es, votre image doit elle aussi \^etre carr\'ee}, faites attention \`a cela.

Ensuite, le param\`etre worldSize\index{worldSize}\index{Terrain!worldSize} indique la taille du terrain, c'est-\`a-dire la longueur de ses c\^ot\'es en unit\'es de la sc\`ene. Plus ce nombre est grand, plus l'image est agrandie.

inputScale\index{inputScale}\index{Terrain!inputScale} correspond \`a l'\'echelle adopt\'ee pour l'altitude du terrain. C'est la hauteur qui s\'epare un point de la carte repr\'esent\'e par un pixel noir d'un point repr\'esent\'e par un pixel blanc. Il doit donc \^etre choisi en parall\`ele avec la taille du monde, puisque s'il est trop \'elev\'e et que le monde est trop petit, vous aurez un relief tr\`es escarp\'e.

Les deux derni\`eres valeurs minBatchSize \index{minBatchSize}\index{Terrain!minBatchSize} et maxBatchSize \index{maxBatchSize}\index{Terrain!maxBatchSize} renseignent les tailles minimale et maximale de batch pour notre terrain.



\subsubsection{La Batch Size\index{Batch Size}\index{Terrain!Batch}}


Le mot anglais batch signifie ''lot'' ou ''paquet''. L'affichage de mod\`ele 3D \`a l'\'ecran consommant beaucoup de ressources, plut\^ot que de chercher \`a calculer dans les moindres d\'etails la fa\c{c}on dont appara\^it la pelouse \`a l'autre bout du paysage, pour ensuite ne l'afficher que sur une toute petite surface de l'\'ecran, le moteur va simplifier les choses et calculer de fa\c{c}on grossi\`ere l'affichage de ces objets.

Ainsi, les textures peuvent \^etre simplifi\'ees, mais aussi les meshs, dont l'ordinateur va r\'eduire le nombre de vertices pour avoir moins de calculs \`a faire, vu que vous ne voyez pas les d\'etails (on parle aussi de niveau de d\'etail\index{niveau de d\'etail}, ou LOD\index{LOD}).

Pour un terrain, le maillage pourrait donc avoir un aspect similaire \`a celui-ci (image issue du wiki d'Ogre3D.org) :
Image utilisateur

Le terrain est divis\'e en lots dont la taille varie en fonction de la distance de la cam\'era \`a ces lots. Plus on s'\'eloigne, plus le lot est simplifi\'e par suppression de vertices. Lorsque plusieurs lots atteignent une taille minimale, ils sont regroup\'es en un seul lot, qui est \`a son tour simplifi\'e progressivement si la cam\'era continue de reculer.

La zone o\`u se situe la cam\'era est la plus d\'etaill\'ee, le reste est simplifi\'e.

Si la taille minimum de batch est faible, les lots adjacents auront plus facilement un niveau de d\'etail \'equivalent, mais il y aura plus de lots \`a g\'erer par l'ordinateur. En revanche, si elle est \'elev\'ee, on regroupe plus rapidement les lots, mais les fronti\`eres entre ceux-ci sont plus facilement visibles, car le niveau de d\'etail peut varier plus fortement.

Les valeurs que j'ai mises sont des valeurs courantes, sachez juste que la taille maximum est de 65 et qu'\textbf{elles doivent ob\'eir \`a la formule suivante :
taille=2n+1}



\subsubsection{Mise en place des textures\index{texture}\index{Terrain!texture}}

Pour g\'erer les textures, l'outil Terrain d'Ogre utilise des calques\index{calque}\index{Terrain!calque}. Chacun de ces calques correspond \`a une texture, que vous pourrez ensuite appliquer o\`u bon vous semblera.

Comme on parle de calques, autant vous dire tout de suite qu'il est possible de les superposer, de donner plus ou moins d'intensit\'e \`a un calque, pour cr\'eer des effets \'elabor\'es.

Nous allons commencer avec une seule texture pour faire simple et assimiler le principe. Tout se fait \`a l'aide de notre importateur de donn\'ees :

\begin{lstlisting}[caption={Mise en place d'une texture pour le terrain}]
//donne la taille de la liste de calques
imp.layerList.resize(1);

//donne la taille de la texture dans le monde
imp.layerList[0].worldSize = 100;  

//les deux lignes suivantes inserent chacune une texture dans notre calque
imp.layerList[0].textureNames.push\_back(''grass\_green-01\_diffusespecular.dds'');
imp.layerList[0].textureNames.push\_back(''grass\_green-01\_normalheight.dds'');
\end{lstlisting}

Ici les fonctions sont relativement explicites.

La premi\`ere ligne donne la taille de la liste de calques, ici je n'en ai mis qu'un seul. La seconde ligne donne la taille de la texture dans le monde. \textbf{Plus le nombre\footnote{ce nombre est-il le nombre affect\'e \`a la taille de la texture?} est important, plus la texture sera zoom\'ee, et inversement.}

Les deux lignes suivantes ins\`erent chacune une texture dans notre calque (textureNames est un vector).

Deux textures ? Je croyais qu'on mettait les textures sur des calques diff\'erents ?

En fait, on devrait plut\^ot dire qu'un calque contient un mat\'eriau\index{mat\'eriau}.

Les mat\'eriaux sont faits avec deux textures :

\begin{itemize}
\item une texture diffuse, qui contient les couleurs, les motifs du mat\'eriau ;
\item une texture normale, contenant des informations sur le relief du mat\'eriau.
\end{itemize}
   

La combinaison de ces deux textures permet d'avoir un mat\'eriau complet.



\subsubsection{Filtrage anisotrope\index{Filtrage anisotrope}\index{Terrain!Filtrage anisotrope}}


Je vais revenir rapidement sur le niveau de d\'etail\index{niveau de d\'etail}\index{Terrain!niveau de d\'etail}, qui est r\'eglable pour les mat\'eriaux via le filtrage de texture.

Vous pouvez r\'egler la nettet\'e du placage de textures sur vos meshs via un niveau de filtrage\index{niveau de filtrage}\index{Terrain!niveau de filtrage}.

On peut distinguer quatre options de filtrages, de la plus grossi\`ere \`a la plus pr\'ecise :

\begin{itemize}
\item aucun filtrage ;
\item bilin\'eaire\index{niveau de filtrage!bilin\'eaire} ;
\item trilin\'eaire\index{niveau de filtrage!trilin\'eaire} ;
\item anisotrope\index{niveau de filtrage!anisotrope}.
\end{itemize}


Voici la diff\'erence entre un filtrage anisotrope (\`a gauche) et une texture sans filtrage (\`a droite). La diff\'erence est relativement subtile ici mais visible tout de m\^eme.

Image utilisateur

Je vous propose donc d'opter pour un filtrage anisotrope, avec une valeur de 8 (la valeur par d\'efaut est 1 et \'equivaut \`a l'absence de filtrage). Vous devez donc rajouter ces deux lignes dans votre code, au d\'ebut de la m\'ethode createScene() par exemple.

\begin{lstlisting}[caption={Choix d'un filtrage anisotrope}]
Ogre::MaterialManager::getSingleton().setDefaultTextureFiltering(Ogre::TFO\_ANISOTROPIC);
Ogre::MaterialManager::getSingleton().setDefaultAnisotropy(8);
\end{lstlisting}

Le filtrage de textures n'est pas propre uniquement aux terrains, mais affecte toutes les textures affich\'ees par le moteur.


\subsubsection{Chargement et nettoyage}


Une fois que les param\`etres ont \'et\'e d\'efinis dans l'ImportData, il ne reste qu'\`a pr\'eparer et charger le terrain :

\begin{lstlisting}[caption={Pr\'eparation et chargement du terrain}]
mTerrain->prepare(imp);
mTerrain->load();
\end{lstlisting}

Pour terminer et faire un peu de place en m\'emoire, il est conseill\'e d'appeler la m\'ethode suivante qui se chargera de lib\'erer la m\'emoire allou\'ee temporairement pour la cr\'eation de votre terrain. Placez donc cette ligne \`a la fin de la m\'ethode createTerrain().

\begin{lstlisting}[caption={Lib\'eration de place en m\'emoire}]
mTerrain->freeTemporaryResources();
\end{lstlisting}

Compilez et lancez l'application pour obtenir un joli paysage !
































%---------------------------------------------------------------------------------------------------------------


\section{Le plaquage de textures}


Ogre nous permet d'utiliser diff\'erents calques pour nos mat\'eriaux. On va pouvoir mettre les calques les uns sur les autres, modifier leur opacit\'e pour avoir une texture plus ou moins visible, tout en d\'ecidant de la zone o\`u l'on veut appliquer la texture.

La premi\`ere \'etape consiste \`a rajouter des calques dans notre liste, avant de cr\'eer le terrain. Remplacez le bloc que vous aviez par le code suivant, afin d'ajouter deux nouveaux mat\'eriaux.



\begin{lstlisting}[caption={Ajout de calques}]
imp.layerList.resize(3);

imp.layerList[0].worldSize = 100;
imp.layerList[0].textureNames.push_back("grass_green-01_diffusespecular.dds");
imp.layerList[0].textureNames.push_back("grass_green-01_normalheight.dds");

imp.layerList[1].worldSize = 30;
imp.layerList[1].textureNames.push_back("growth_weirdfungus-03_diffusespecular.dds");
imp.layerList[1].textureNames.push_back("growth_weirdfungus-03_normalheight.dds");

imp.layerList[2].worldSize = 200;
imp.layerList[2].textureNames.push_back("dirt_grayrocky_diffusespecular.dds");
imp.layerList[2].textureNames.push_back("dirt_grayrocky_normalheight.dds");

mTerrain->prepare(imp);
mTerrain->load();
\end{lstlisting}

Par d\'efaut, c'est uniquement le premier mat\'eriau qui est affich\'e donc si vous ex\'ecutez le code maintenant, rien n'aura chang\'e sur votre terrain.  Il est donc n\'ecessaire d'ajouter quelques lignes pour dire de quelle fa\c{c}on nous voulons faire notre plaquage de texture.\newline

Le principe est simple \`a comprendre, chaque calque d'un terrain poss\`ede un objet TerrainLayerBlendMap\index{TerrainLayerBlendMap}\index{Terrain!TerrainLayerBlendMap} (que j'appellerai dor\'enavant Blend Map\index{Blend Map}) qui exprime la fa\c{c}on dont le calque est fusionn\'e avec les calques inf\'erieurs (le calque le plus bas est le calque 0).

Cette fusion\index{fusion de calques}\index{Terrain!fusion de calques} est simplement une affaire de transparence. Les calques sont plac\'es les uns sur les autres, et la composante transparente indique si la texture en dessous est plus ou moins visible. Il est de plus possible de faire varier la transparence du calque en chaque point de celui-ci, ce qui permet d'avoir un placage par zone.\newline

\`A la suite du bloc de code pr\'ec\'edent, commencez par r\'ecup\'erer les Blend Map correspondant au calque num\'ero 1, que nous venons d'ajouter (nous nous occuperons du second ensuite).

\begin{lstlisting}[caption={R\'ecup\'eration du Blend Map pour le premier terrain}]
Ogre::TerrainLayerBlendMap* blendMap1 = mTerrain->getLayerBlendMap(1);
\end{lstlisting}

Nous allons commencer par plaquer une seule texture au-dessus de l'herbe, sur toute la surface de notre terrain.

L'id\'ee est de parcourir l'ensemble des points du calque avec deux boucles for (il y a autant de points qu'il y avait de pixels dans notre heightmap) et de leur attribuer la transparence d\'esir\'ee, entre 1 (totalement opaque) et 255 (transparent).

La valeur la plus faible est bien 1 et non 0. Si vous mettez 0, la texture est transparente !

\begin{lstlisting}[caption={Attribution de la transparence d\'esir\'ee sur tous les points du calque}]
float* pBlend1 = blendMap1->getBlendPointer();

for (Ogre::uint16 y = 0; y < mTerrain->getLayerBlendMapSize(); ++y)
{
    for (Ogre::uint16 x = 0; x < mTerrain->getLayerBlendMapSize(); ++x)
    {   
        // opacite desiree pour le point courant
        *pBlend1++ = 150;
    }
}
\end{lstlisting}















Pour terminer, il faut mettre \`a jour notre Blend Map en appelant les m\'ethodes dirty()\index{dirty()}\index{Blend Map!dirty()} puis update()\index{update()}\index{Blend Map!update()}. La premi\`ere sert \`a pr\'eciser que les donn\'ees de la Blend map sont obsol\`etes et doivent \^etre mises \`a jour, tandis que la seconde fait effectivement la mise \`a jour.

Si vous n'appelez pas d'abord dirty(), update() n'aura aucun effet.

\begin{lstlisting}[caption={Mise \`a jour de la Blend Map}]
blendMap1->dirty();
blendMap1->update();
\end{lstlisting}

Je vous mets le code complet de la m\'ethode createTerrain() si vous voulez v\'erifier que tout est en ordre :

\begin{lstlisting}[caption={createTerrain (code complet)}]
mTerrain = OGRE_NEW Ogre::Terrain(mSceneMgr);

// options globales
mGlobals = OGRE_NEW Ogre::TerrainGlobalOptions();
mGlobals->setMaxPixelError(10);
mGlobals->setCompositeMapDistance(8000);
mGlobals->setLightMapDirection(mLight->getDerivedDirection());
Globals->setCompositeMapAmbient(mSceneMgr->getAmbientLight());
mGlobals->setCompositeMapDiffuse(mLight->getDiffuseColour());
Ogre::Image img;
img.load("terrain.png", Ogre::ResourceGroupManager::DEFAULT_RESOURCE_GROUP_NAME);

// informations g\'eom\'etriques
Ogre::Terrain::ImportData imp;
imp.inputImage = &img;
imp.terrainSize = img.getWidth();
imp.worldSize = 8000;
imp.inputScale = 600;
imp.minBatchSize = 33;
imp.maxBatchSize = 65;

// textures
imp.layerList.resize(3);
imp.layerList[0].worldSize = 100;
imp.layerList[0].textureNames.push_back("grass_green-01_diffusespecular.dds");
imp.layerList[0].textureNames.push_back("grass_green-01_normalheight.dds");
imp.layerList[1].worldSize = 30;
imp.layerList[1].textureNames.push_back("growth_weirdfungus-03_diffusespecular.dds");
imp.layerList[1].textureNames.push_back("growth_weirdfungus-03_normalheight.dds");
imp.layerList[2].worldSize = 200;
imp.layerList[2].textureNames.push_back("dirt_grayrocky_diffusespecular.dds");
imp.layerList[2].textureNames.push_back("dirt_grayrocky_normalheight.dds");
mTerrain->prepare(imp);35mTerrain->load();

// plaquage de texture
Ogre::TerrainLayerBlendMap* blendMap1 = mTerrain->getLayerBlendMap(1);
float* pBlend1 = blendMap1->getBlendPointer();
for (Ogre::uint16 y = 0; y < mTerrain->getLayerBlendMapSize(); ++y)
{
    for (Ogre::uint16 x = 0; x < mTerrain->getLayerBlendMapSize(); ++x)
    {
        *pBlend1++ = 150;
    }
}

blendMap1->dirty();
blendMap1->update();

mTerrain->freeTemporaryResources();
\end{lstlisting}

Vous pouvez maintenant ex\'ecuter l'application ! Profitez-en pour v\'erifier en vous approchant du sol que l'on distingue bien l'herbe et par-dessus la terre.
Image utilisateur


D'accord, la forme n'est pas terrible. Mais le principe y est, c'est un d\'ebut.

On va tout de suite essayer de faire quelque chose de plus esth\'etique.

Pour cela, maintenant que vous avez saisi les grandes \'etapes, je vous propose un mini-TP pour vous entra\^iner.








\subsection{Au boulot !}



\subsubsection{Objectif}

Une image vaut s\^urement mieux qu'un long discours, voici donc ce que vous allez devoir obtenir :
Image utilisateur

L'id\'ee est de texturer le terrain en fonction de l'altitude. Si l'on est en dessous d'un certain seuil, il n'y a que de la terre qui appara\^t ; au-dessus, on a de l'herbe.



\subsubsection{Indications}

Je vous donne tout de m\^eme les m\'ethodes qui sont utiles, notamment pour trouver l'altitude du terrain en fonction de la position :

\begin{lstlisting}[caption={m\'ethode getHeightAtTerrainPosition pour trouver l'altitude du terrain en fonction de la position }]
float Ogre::Terrain::getHeightAtTerrainPosition(Ogre::Real x, Ogre::Real y)
\end{lstlisting}

Cette fonction retourne l'altitude en fonction de la position, sachant que x et y sont compris entre 0 et 1.

Pour r\'ecup\'erer ces deux valeurs, on utilise une m\'ethode de TerrainLayerBlendMap qui convertit les coordonn\'ees de l'image en coordonn\'ees du terrain (celles dont vous avez besoin) :

\begin{lstlisting}[caption={M\'ethode convertImageToTerrainSpace pour convertir les coordonn\'ees de l'image en coordonn\'ees du terrain}]
void Ogre::TerrainLayerBlendMap::convertImageToTerrainSpace(size_t x, size_t y, Ogre::Real * outX, Ogre::Real * outY)
\end{lstlisting}


Enfin, je vous laisse choisir l'altitude limite entre l'herbe et la terre. Vous avez toutes les cartes en mains maintenant.

























\subsubsection{Correction}

Comme avez d\^u le deviner, tout se passe dans les deux boucles for.

Pour chaque point parcouru, on recherche ses coordonn\'ees dans le rep\`ere du terrain, puis on r\'ecup\`ere la hauteur, que l'on compare \`a notre hauteur limite. Si on est en dessous, on affiche la texture de terre avec une opacit\'e maximum, sinon on ne fait qu'incr\'ementer le pointeur de la Blend map.

Voici donc le code modifi\'e :

\begin{lstlisting}[caption={Attribution de la transparence d\'esir\'ee sur tous les points du calque selon leur position}]
for (Ogre::uint16 y = 0; y < mTerrain->getLayerBlendMapSize(); ++y)
{
    for (Ogre::uint16 x = 0; x < mTerrain->getLayerBlendMapSize(); ++x)
    {
        Ogre::Real terrainX, terrainY;
        blendMap1->convertImageToTerrainSpace(x, y, &terrainX, &terrainY);
        Ogre::Real height = mTerrain->getHeightAtTerrainPosition(terrainX, terrainY);
        if(height < 200)
            *pBlend1 = 1;
        pBlend1++;
    }
}
\end{lstlisting}

Vous pouvez aussi bien s\^ur r\'ecup\'erer la Blend map num\'ero 2 plus haut pour voir le r\'esultat avec une autre texture, c'est ce que j'ai fait pour obtenir l'image r\'ef\'erence pour le TP.





\subsection{Pour aller plus loin}

Sur le m\^eme principe, nous allons voir comment appliquer deux textures diff\'erentes sur une petite largeur, \`a une altitude donn\'ee.

Ceci peut \^etre utile si vous voulez r\'ealiser des \'etendues d'eau dans votre terrain : au bord de l'eau, il y a de la boue, un peu au-dessus, de la terre s\`eche, puis ensuite l'herbe reprend ses droits. C'est ce que nous allons faire, avec une opacit\'e progressive, mais sans l'eau, ce sera pour plus tard.

Nous devons commencer par r\'ecup\'erer un pointeur sur notre seconde Blend Map pour pouvoir g\'erer la seconde texture en plus de la premi\`ere. Il y a donc deux lignes \`a rajouter en cons\'equence avant les boucles.




\begin{lstlisting}[caption={R\'ecup\'eration des Blend Map pour le premier et le second terrain}]
Ogre::TerrainLayerBlendMap* blendMap1 = mTerrain->getLayerBlendMap(1);
Ogre::TerrainLayerBlendMap* blendMap2 = mTerrain->getLayerBlendMap(2);

float* pBlend1 = blendMap1->getBlendPointer();
float* pBlend2 = blendMap2->getBlendPointer();
\end{lstlisting}

D\'efinissons aussi deux variables pour chacune des textures : la hauteur \`a laquelle se situe la texture et la largeur de la bande que l'on veut obtenir.

\begin{lstlisting}[caption={}]
Ogre::Real minHeight1 = 70;
Ogre::Real fadeDist1 = 40;
Ogre::Real minHeight2 = 70;
Ogre::Real fadeDist2 = 15;
\end{lstlisting}

Dans la boucle, on d\'eclare trois variables : deux coordonn\'ees du terrain et la transparence pour le point actuel.

\begin{lstlisting}[caption={}]
Ogre::Real terrainX, terrainY, transparence;
\end{lstlisting}

On r\'ecup\`ere ensuite la hauteur du terrain comme pr\'ec\'edemment.

\begin{lstlisting}[caption={}]
blendMap1->convertImageToTerrainSpace(x, y, &terrainX, &terrainY);
Ogre::Real height = mTerrain->getHeightAtTerrainPosition(terrainX, terrainY);
\end{lstlisting}

Ensuite, pour chaque texture, on calcule la diff\'erence entre la hauteur du point actuel et la hauteur que l'on veut pour la texture, divis\'ee par la largeur de la bande. Si le point est cens\'e \^etre recouvert par la texture, ce nombre sera donc compris entre 0 et 1.

On utilise ensuite la m\'ethode statique Clamp() qui a pour prototype :

\begin{lstlisting}[caption={Utilisation de la m\'ethode statique Clamp()}]
static T Ogre::Math::Clamp(T val, T minval, T maxval)
\end{lstlisting}

Si val est inf\'erieure \`a minval, la fonction retourne minval ; si val est sup\'erieure \`a maxval, on retourne maxval. Si val est dans l'intervalle, on la retourne directement.

Comme on ne veut afficher que les points dont la valeur calcul\'ee pr\'ec\'edemment est comprise entre 0 et 1, on va utiliser cette m\'ethode pour ''couper'' toutes les valeurs en dehors de l'intervalle.

\begin{lstlisting}[caption={}]
transparence = (height - minHeight1) / fadeDist1;
transparence = Ogre::Math::Clamp(transparence, (Ogre::Real)0, (Ogre::Real)1);
\end{lstlisting}

Pour terminer, on multiplie transparence par 255 pour avoir une valeur comprise entre 0 et 255.

\begin{lstlisting}[caption={}]
*pBlend1++ = transparence * 255;
\end{lstlisting}

On observe que si transparence est \`a l'ext\'erieur de l'intervalle [0 ; 1] apr\`es le premier calcul, Clamp retournera 0 ou 1. Quand on multiplie par 255, on obtient donc 0 ou 255, qui sont les deux valeurs pour lesquelles la texture est transparente. Mission accomplie !

On copie ces trois lignes pour la seconde texture, et on obtient le code suivant dans nos boucles :

\begin{lstlisting}[caption={}]
for (Ogre::uint16 y = 0; y < mTerrain->getLayerBlendMapSize(); ++y)
{
    for (Ogre::uint16 x = 0; x < mTerrain->getLayerBlendMapSize(); ++x)
    {
        Ogre::Real terrainX, terrainY, transparence;
        blendMap1->convertImageToTerrainSpace(x, y, &terrainX, &terrainY);
        Ogre::Real height = mTerrain->getHeightAtTerrainPosition(terrainX, terrainY);
        transparence = (height - minHeight1) / fadeDist1;
        transparence = Ogre::Math::Clamp(transparence, (Ogre::Real)0, (Ogre::Real)1);
        *pBlend1++ = transparence * 255;
        transparence = (height - minHeight2) / fadeDist2;
        transparence = Ogre::Math::Clamp(transparence, (Ogre::Real)0, (Ogre::Real)1);
        *pBlend2++ = transparence * 255;
    }
}
\end{lstlisting}

Pensez \`a mettre \`a jour la seconde Blend Map une fois que les modifications sont termin\'ees :

\begin{lstlisting}[caption={}]
blendMap1->dirty();
blendMap2->dirty();
blendMap1->update();
blendMap2->update();
\end{lstlisting}








%140522
%A relire, corriger et indexer à partir d ici
%---------------------------------------------------------------------------------------------------------------
\section{Les groupes de terrains}

Un groupe de terrains\index{groupe de terrains} (ou TerrainGroup\index{TerrainGroup}) range les terrains comme dans un tableau \`a deux dimensions. Dans le groupe, tous les terrains doivent avoir la m\^eme taille, afin de pouvoir les aligner dans une grille.



\subsection{Cr\'eation}

On commence par ajouter un include au d\'ebut de la classe puis on remplace notre instance de Terrain par un TerrainGroup, ensuite on l'initialisera dans notre m\'ethode createTerrain() :

\begin{lstlisting}[caption={TerrainGroup: include et cr\'eation}]
#include <Ogre/Terrain/OgreTerrainGroup.h>
//...
Ogre::TerrainGroup *mTerrainGroup;
\end{lstlisting}

Apr\`es les lignes permettant de charger l'image heightmap dans la m\'ethode createTerrain()\index{createTerrain()}, ins\'erez les lignes suivantes.

\begin{lstlisting}[caption={}]
mTerrainGroup = OGRE_NEW Ogre::TerrainGroup(mSceneMgr, Ogre::Terrain::ALIGN_X_Z, img.getWidth(), 8000);

//On definit ensuite la position de l'origine du groupe de terrains
mTerrainGroup->setOrigin(Ogre::Vector3::ZERO);

//nom (et l'extension) que l'on veut attribuer a nos fichiers qui seront crees pour sauvegarder les terrains
mTerrainGroup->setFilenameConvention(Ogre::String("TerrainDuZero"), Ogre::String("dat"));
\end{lstlisting}

Les param\`etres \`a fournir au TerrainGroup sont
\begin{itemize}
\item le Scene manager,
\item l'alignement du terrain\index{ALIGN\_X\_Z}\index{Terrain!ALIGN\_X\_Z}\index{alignement du terrain} par rapport au rep\`ere global (vous pouvez cr\'eer un terrain vertical, par exemple),
\item la taille des heightmaps utilis\'ees\index{getWidth()}\index{heightmaps!getWidth()}, 
\item la taille d'un terrain.
\end{itemize}

On d\'efinit ensuite la position de l'origine du groupe de terrains\index{setOrigin()}\index{TerrainGroup!setOrigin()}, puis le nom\index{setFilenameConvention()}\index{TerrainGroup!setFilenameConvention()} (et l'extension) que l'on veut attribuer \`a nos fichiers qui seront cr\'e\'es pour sauvegarder les terrains par la suite.

Pour les donn\'ees des terrains enregistr\'ees dans un objet ImportData, nous allons modifier un peu le fonctionnement du programme. On r\'ecup\`ere en fait directement une r\'ef\'erence sur un ImportData fourni par le TerrainGroup\index{getDefaultImportSettings()}\index{TerrainGroup!getDefaultImportSettings()}, que l'on modifie directement et qui sera valable pour l'ensemble des terrains du groupe. \`A noter que la ligne de d\'efinition de la heightmap n'est plus utile ici, cela sera indiqu\'e lors de la cr\'eation des terrains.

\begin{lstlisting}[caption={}]
Ogre::Terrain::ImportData& imp = mTerrainGroup->getDefaultImportSettings();
imp.terrainSize = img.getWidth();
imp.worldSize = 8000;
imp.inputScale = 600;
imp.minBatchSize = 33;
imp.maxBatchSize = 65;
\end{lstlisting}

Il est maintenant temps de cr\'eer les terrains du groupe. On d\'efinit la taille du groupe et pour chaque case, on appelle une m\'ethode definirTerrain() d\'efinie plus bas qui s'occupera de cr\'eer chaque terrain ind\'ependamment.

\begin{lstlisting}[caption={Cr\'eation des terrains du groupe}]
int largeur = 2, longueur = 2;

for(int x = 0; x < largeur; x++)
{
    for(int y = 0; y < longueur; y++)
    {
        definirTerrain(x, y);
    }
}
mTerrainGroup->loadAllTerrains(true);
\end{lstlisting}

Le groupe se charge pour terminer d'appeler les m\'ethodes load()\index{load()} de chaque terrain \`a travers la m\'ethode loadAllTerrains()\index{loadAllTerrains()}. Cette m\'ethode prend un bool\'een en param\`etre qui indique si le chargement doit \^etre synchrone, c'est-\`a-dire ex\'ecut\'e dans un seul thread (le thread principal ici). Par d\'efaut cette valeur est fausse, c'est-\`a-dire que les terrains sont charg\'es dans plusieurs threads si c'est possible.

Le chargement devient vite tr\`es lourd si l'on ajoute beaucoup de terrains aux groupes. Nous verrons plus bas comment acc\'el\'erer le chargement.

Maintenant, nous devons \'ecrire la m\'ethode definirTerrain() qui utilise la m\'ethode defineTerrain()\index{defineTerrain()}\index{TerrainGroup!defineTerrain()} de TerrainGroup. Celle-ci va prendre 3 param\`etres : les deux coordonn\'ees du terrain dans le groupe de terrains (sa position sur la grille, donc) et l'image heightmap utilis\'ee pour ce terrain. Les coordonn\'ees du terrain au sein du groupe peuvent \^etre n\'egatives.

Juste avant d'appeler le terrain, on va faire une v\'erification sur les coordonn\'ees : si l'abscisse du terrain est impaire, on inverse l'image suivant l'axe Y, si l'ordonn\'ee est impaire, on inverse l'image cette fois-ci selon l'axe X. Cela permet aux terrains du groupe de ne pas avoir de diff\'erence d'altitude lors des jointures. Si vous utilisez des heightmaps\index{heightmaps} diff\'erents sur les terrains du groupe (ce qui sera probablement le cas), vous devrez faire attention \`a ce que les altitudes des bords correspondent pour \'eviter les trous \`a ces endroits.

\begin{lstlisting}[caption={PremiereApplication.definirTerrain}]
void PremiereApplication::definirTerrain(int x, int y)
{
    Ogre::Image img;
    img.load("terrain.png", Ogre::ResourceGroupManager::DEFAULT_RESOURCE_GROUP_NAME);

    if(x % 2 != 0)
        img.flipAroundY();

    if(y % 2 != 0)
        img.flipAroundX();

    mTerrainGroup->defineTerrain(x, y, &img);
}
\end{lstlisting}

Une fois que les terrains sont charg\'es, il faut leur appliquer les textures d\'efinies. On utilise pour cela un it\'erateur sur le groupe de terrains et, pour chaque terrain, on appelle une m\'ethode initBlendMaps() qui contient le code pour texturer les terrains.

\begin{lstlisting}[caption={}]
Ogre::TerrainGroup::TerrainIterator ti = mTerrainGroup->getTerrainIterator();

while(ti.hasMoreElements())
{
    Ogre::Terrain* t = ti.getNext()->instance;
    initBlendMaps(t);
}
\end{lstlisting}

La m\'ethode initBlendMaps() contient uniquement du code que l'on a d\'ej\`a vu mais que j'ai d\'eplac\'e pour plus de clart\'e. Elle prend en param\`etre le terrain dont on doit modifier les Blend maps.

\begin{lstlisting}[caption={}]
void PremiereApplication::initBlendMaps(Ogre::Terrain *terrain)
{
    Ogre::TerrainLayerBlendMap* blendMap1 = terrain->getLayerBlendMap(1);
    Ogre::TerrainLayerBlendMap* blendMap2 = terrain->getLayerBlendMap(2);
    Ogre::Real minHeight1 = 70;
    Ogre::Real fadeDist1 = 40;
    Ogre::Real minHeight2 = 70;
    Ogre::Real fadeDist2 = 15;
    float* pBlend1 = blendMap1->getBlendPointer();
    float* pBlend2 = blendMap2->getBlendPointer();

    for (Ogre::uint16 y = 0; y < terrain->getLayerBlendMapSize(); ++y)
    {
        for (Ogre::uint16 x = 0; x < terrain->getLayerBlendMapSize(); ++x)
        {
            Ogre::Real terrainX, terrainY, transparence;
            blendMap1->convertImageToTerrainSpace(x, y, &terrainX, &terrainY);
            Ogre::Real height = terrain->getHeightAtTerrainPosition(terrainX, terrainY);
            transparence = (height - minHeight1) / fadeDist1;
            transparence = Ogre::Math::Clamp(transparence, (Ogre::Real)0, (Ogre::Real)1);
            *pBlend1++ = transparence * 255;
            transparence = (height - minHeight2) / fadeDist2;
            transparence = Ogre::Math::Clamp(transparence, (Ogre::Real)0, (Ogre::Real)1);
            *pBlend2++ = transparence * 255;
        }
    }
    blendMap1->dirty();
    blendMap2->dirty();
    blendMap1->update();
    blendMap2->update();
}
\end{lstlisting}

Pour terminer, comme avec un terrain seul, on lib\`ere la m\'emoire utilis\'ee par le TerrainGroup.

\begin{lstlisting}[caption={}]
mTerrainGroup->freeTemporaryResources();
\end{lstlisting}

Votre sc\`ene doit maintenant avoir une surface plus grande que la première fois (on a maintenant quatre terrains). Vous pouvez encore augmenter le nombre de terrains, mais attention, le temps de chargement augmente rapidement !



\subsection{Optimiser le temps de chargement}

Vous avez certainement remarqu\'e que la g\'en\'eration du terrain prend un temps cons\'equent lorsque le groupe s'agrandit. La cr\'eation du terrain \`a partir du fichier heightmap n\'ecessite en effet de convertir les donn\'ees de l'image en donn\'ees exploitables par le moteur.

Afin de r\'eduire le temps de chargement, il est possible d'enregistrer un fichier qui contient toutes les informations sur le terrain construit pour \'eviter de relire l'image \`a chaque lancement de l'application. L'inconv\'enient majeur est la place occup\'ee par ces fichiers g\'en\'er\'es, qui contiennent beaucoup plus d'informations qu'une simple heightmap.

En regardant la cr\'eation du groupe de terrains, vous voyez que l'on a d\'efini une convention de nommage pour des fichiers, mais qui est pour l'instant inutilis\'ee.

\begin{lstlisting}[caption={}]
mTerrainGroup->setFilenameConvention(Ogre::String("TerrainDuZero"), Ogre::String("dat"));
\end{lstlisting}

Cette ligne sert lors de la sauvegarde de fichiers de terrain : ceux-ci seront nomm\'es en commençant par « TerrainDuZero » suivi d'un nombre permettant d'identifier le terrain, puis de l'extension de fichier « dat ».

Pour utiliser la sauvegarde des terrains, nous allons ajouter un attribut mTerrainCreated \`a la classe PremiereApplication qui permettra de savoir si l'on a g\'en\'er\'e le terrain \`a partir d'une image ou bien si l'on a lu un fichier terrain. Dans le premier cas, on saura qu'\`a la fin de la m\'ethode createTerrain() il faut penser \`a sauvegarder les fichiers de terrain pour le prochain lancement de l'application.

\begin{lstlisting}[caption={}]
bool mTerrainCreated;
\end{lstlisting}

Initialisez sa valeur \`a false au d\'ebut de la m\'ethode createTerrain().

Maintenant, dans notre m\'ethode definirTerrain(), il faut v\'erifier si le fichier terrain existe d\'ej\`a ou bien s'il faut faire la g\'en\'eration depuis le heightmap comme le faisait jusqu'alors. Dans le second cas, on passe la variable mTerrainCreated \`a true.

\begin{lstlisting}[caption={}]
void PremiereApplication::definirTerrain(int x, int y)
{
    if(Ogre::ResourceGroupManager::getSingleton().resourceExists(mTerrainGroup->getResourceGroup(), mTerrainGroup->generateFilename(x, y)))
    {
        mTerrainGroup->defineTerrain(x, y);
    }
    else
    {
        Ogre::Image img;
        img.load("terrain.png", Ogre::ResourceGroupManager::DEFAULT_RESOURCE_GROUP_NAME);

        if(x % 2 != 0)
            img.flipAroundY();

        if(y % 2 != 0)
            img.flipAroundX();

        mTerrainGroup->defineTerrain(x, y, &img);
        mTerrainCreated = true;
    }
}
\end{lstlisting}

Revenons sur la condition \`a tester pour v\'erifier l'existence du fichier g\'en\'er\'e.

Gr\^ace au Ogre::ResourceGroupManager, on peut v\'erifier s'il existe une ressource pr\'ecise dans l'ensemble des ressources charg\'ees au d\'emarrage du programme. Les param\`etres de la m\'ethode sont le groupe de ressources dans lequel on veut chercher la ressource ainsi que le nom du fichier recherch\'e. Vous voyez que le nom du fichier g\'en\'er\'e par le groupe de terrains d\'epend de ses coordonn\'ees X et Y, ainsi que de la convention que l'on a d\'efinie au d\'ebut.

Si le fichier est trouv\'e, on appelle la m\'ethode definieTerrain() avec seulement les coordonn\'ees en param\`etres. Dans ce cas, Ogre va aller chercher directement le fichier correspondant \`a ces coordonn\'ees. Dans le cas contraire, on ex\'ecute le bloc que l'on avait pr\'ec\'edemment et qui charge le terrain \`a partir de l'image de heightmap.

Il ne reste plus qu'\`a demander la sauvegarde des fichiers si l'on a g\'en\'er\'e les terrains juste avant de lib\'erer les ressources dans la m\'ethode createTerrain() :

\begin{lstlisting}[caption={}]
if(mTerrainCreated)
    mTerrainGroup->saveAllTerrains(true);
\end{lstlisting}


Lancez l'application, le temps de chargement doit \^etre un peu plus long qu'auparavant car l'ordinateur sauvegarde en m\^eme temps les fichiers g\'en\'er\'es sur le disque. Une fois que l'application est lanc\'ee, fermez-la puis relancez-la. Le temps de chargement doit normalement \^etre meilleur.

Vous devriez trouver les fichiers g\'en\'er\'es dans le dossier OgreSDK.media. Pour information, les miens font chacun une taille de 12 Mo.



























\printindex
\end{document}







\colorbox{graun}{\lstinline!!}
\colorbox{graun}{\lstinline!!}
\colorbox{graun}{\lstinline!!}

\begin{lstlisting}[caption={}, language=C++]

\end{lstlisting}


\begin{lstlisting}[caption={}, language=C++]

\end{lstlisting}


\begin{lstlisting}[caption={}, language=C++]

\end{lstlisting}



%\begin{figure}[hbtp]
%\caption{}
%\centering
%\includegraphics[width=10cm]{Django_Officiel/Part3/}
%\end{figure}


%Pour inserer des images
%\begin{figure}[hbtp]
%\caption{XXX}
%\centering
%\includegraphics[width=16cm]{18-IteratorPattern/IteratorPattern_Illustration-A.png}
%DjangoAdministration.png
%\end{figure}