

\chapter{Insérer des objets dans la scène }

\section{les entités}

\subsection{Une entité c'est quoi}
Une entité est l'ensemble des informations attachées aux polygones constituant un objet 3D.\newline
Un Mesh étant l'ensemble des polygones constituant un objet 3D, une entité est l'ensemble des informations attachées à un Mesh:

\begin{itemize}
\item les vertices (sommets des polygones),
\item les textures,
\item un squelette si le modèle est sujet à des animations
\item...\newline
\end{itemize}


Tous les objets solides qui apparaissent à l'écran sont donc des entités et sont représentés par une seule et même classe dans Ogre: la classe Entity.

\subsection{Le sceneManager}
C'est au sceneManager que revient la tâche de créer tous les objets que peut contenir la scène (tous les modèles, lumières, caméra et autres objets) et de nous permettre ensuite d'y accéder.\newline
Par conséquent, l'insertion d'objets dans la scène passe toujours par lui (ou par l'un des éléments déjà insérés) et par les méthodes qu'il propose pour cela.
Il en existe diverses variantes en fonction de la scène que l'on veut réaliser; selon que celle-ci sera par exemple en intérieur ou en extérieur, par exemple.

Toutes les applications Ogre doivent donc avoir un sceneManager pour pouvoir fonctionner, puisque c'est lui qui s'occupe de tout ! La classe ExampleApplication ne fait pas exception et possède donc un attribut msceneMgr, qui est un pointeur sur le sceneManager de l'application et qui nous permettra dans les parties suivantes d'agrémenter notre scène avec des objets.


\subsection{Créer une entité}

L'entité doit être créée par le sceneManager pour pouvoir être ajoutée à la scène. La méthode createEntity() permet de faire cela.
\begin{lstlisting}
Entity *head= msceneMgr->createEntity(''Tete'',''ogrehead.mesh'');
\end{lstlisting}\footnote{Pourquoi utilise t on -> pour appeler une méthode d'une classe?}

\begin{itemize}
\item Le premier paramètre est le nom que vous souhaitez donner au mesh. Ce nom doit être unique!
\item Le second paramètre est le nom du fichier que vous voulez charger. Notez l'extension.mesh, qui est le format de fichiers pour les modèles reconnu par Ogre.
\end{itemize}




Le fichier spécifié (avec le deuxième paramètre) se trouve, accompagné d'autres modèles d'exemples, dans le dossier OgreSDK/media/models, qui doit être correctement renseigné dans le fichier resources.cfg pour qu'Ogre puisse le trouver lors de l'exécution.\footnote{Mais à quoi sert ce fichier .mesh?}

Le mesh a été ajouté à la scène mais il nous manque encore une chose avant de pouvoir l'afficher à l'écran...



\section{Les Nœuds de la scène}



\subsection{L'utilité des Nœuds}

Dans Ogre, lorsque l'on souhaite manipuler une entité (un personnage, une lumière, une caméra...), les déplacements que l'on veut effectuer se font par l'intermédiaire d'un nœud de scène, ou \textit{sceneNode}.

Un sceneNode est un objet invisible auquel on va pouvoir attacher un nombre indéfini d'entités, lesquelles deviennent solidaires de ce nœud et subissent donc les même transformations que lui. C'est donc une sorte de conteneur qui contient les informations de positionnement de chacune des entités de la scène qui lui sont rattachées.

Bien sûr on pourrait déplacer nos entités directement mais avec un nœud on pourra déplacer en une fois toutes les entités attachées au nœud.

Quoi qu'il en soit, attacher chaque entité à un nœud est primordial, sans quoi elle ne s'affichera pas dans votre scène!



\subsection{Créer un nœud}

Pour créer un nœud de scène on devra passer par un nœud déjà existant, ce qui va nous permettre d'avoir des relations d'héritage entre nos nœuds.\newline

On utilisera une méthode dédiée:

\begin{lstlisting}
sceneNode *noeudEnfant = noeudParent->createChildSceneNode(''enfant'', Vector3::ZERO, Quaternion::IDENTITY);
\end{lstlisting}


Mais comment fait-on pour le premier noeud\index{noeud} qu'on va créer ? Le nœud ''racine'' existe dès que le sceneManager est créé, c'est un nœud comme un autre avec les même \footnote{''les même'' faut il un ''s'' à ''même'' ?} méthodes mais ce nœud est unique.\newline

Nous récupérons ce nœud racine de la scène  à l'aide de l'instance du sceneManager:
\begin{lstlisting}
sceneNode *node= msceneMgr->getRootsceneNode()->createChildSceneNode(''nodeTete'', Vector3::ZERO, Quaternion::IDENTITY);
\end{lstlisting}

La méthode getRootsceneNode() nous permet de récupérer un pointeur sur le nœud racine unique de la scène. On appelle ensuite sa méthode createChildSceneNode pour lui ajouter un nouveau nœud fils.

Notez qu'aucun des paramètres de la méthode n'est obligatoire, pour information:
\begin{itemize}
\item Le premier argument est le nom que vous voulez donner à votre nœud. Sur le même principe que les entités ce nom pourra être utilisé pour récupérer un pointeur vers le nœud en question. 
\item Le deuxième argument est la position initiale du nœud.
\item Le troisième argument est le quaternion avec lequel vous voulez initialiser votre nœud.
\end{itemize}
	

Sachez pour le moment qu'un quaternion est un objet mathématique qui permet de faire faire aux objets des rotations dans l'espace\footnote{Que des rotations?}. Quaternion::IDENTITY \footnote{comment écrire Quaternion::IDENTITY de manière élégante?}lui dit de ne pas faire de rotation.\newline

Avec ce code en main, vous pouvez simplement attacher l'entité précédemment créée au nœud avec la ligne suivante:

\begin{lstlisting}
node->attachObject(head);
\end{lstlisting}

En compilant, vous devriez voir la tête d'un ogre au milieu de l'écran. Vous pouvez déplacer la caméra avec Z, S, Q, D et la souris pour voir ce que ça donne de plus près.



\subsection{Code}

\subsubsection{PremiereApplication.cpp}
\begin{lstlisting}[caption={PremiereApplication.cpp: Instanciation d'entité}]
#include "PremiereApplication.h"

void PremiereApplication::createScene()
{
    //creation d une entite
    Entity *head= mSceneMgr->createEntity("Tete", "ogrehead.mesh" );
    
    //creation d un noeud
    SceneNode *node= mSceneMgr->getRootSceneNode( )->createChildSceneNode( "nodeTete " , Vector3::ZERO, Quaternion::IDENTITY);
    
    node->yaw(Radian(Math::PI));
    node->yaw(Radian(Math::PI));

    Vector3 position = Vector3(30.0, 50.0, 0.0);
    node->setPosition(position);

    node->setPosition(30.0, 50.0, 0.0); 
    node->translate(-30.0, 50.0, 0.0); 
    
    
    //attachement de l entite au noeud
    node->attachObject ( head );
}
\end{lstlisting}








\section{Créer un mesh}

Nous pouvons rajouter un sol à notre scène, pour cela nous allons créer nous-mêmes un nouveau mesh. étant donné que nous n'avons besoin que d'un plan pour le sol, le mesh peut très simplement être créer dans le code de notre application.



\subsection{Le mesh}

Il existe une classe Plane\footnote{de même que pour quaternion.entity marquer les noms de classe de manière élégante pour les discerner du texte serait bien} qui va nous permettre de générer... un plan qui représentera le sol.\footnote{En plus de la classe Plane, vous trouverez aussi des classes Box (pour les cubes) et Sphere qui fonctionnent sur le même principe.}\newline

Pour créer un plan nous appelons Plane avec les paramètres suivants:
\begin{itemize}
\item le premier paramètre permet de définir le vecteur normal au plan à créer (ici l'axe Y pour que notre plan soit horizontal).
\item le second paramètre est la distance à l'origine de la scène dans le sens du vecteur normal  (ici, je mets 0 pour que mon mesh plan soit centré). 
\end{itemize}

\begin{lstlisting}
	Plane plan(Vector3::UNIT_Y, 0);
\end{lstlisting}

Une fois le plan créé, il faut que l'on crée un mesh, c'est-à-dire l'objet 3D en lui-même (la représentation du plan) qui sera visible dans la scène.
Pour cela, on utilise le Mesh Manager, qui va s'occuper de créer les faces de notre mesh.

\begin{lstlisting}
MeshManager::getSingleton().createPlane(''sol'', ResourceGroupManager::DEFAULT_RESOURCE_GROUP_NAME, plan, 500, 500, 1, 1, true, 1, 1, 1, Vector3::UNIT_Z);
\end{lstlisting}

Quelques explications sur cette ligne s'imposent. 
\begin{itemize}
\item Tout d'abord, la méthode statique getSingleton() permet de récupérer un objet instancié de façon unique, donc ici notre MeshManager.
\item Les deux premiers paramètres correspondent respectivement au nom que l'on veut donner à notre mesh et au nom du groupe auquel on veut qu'il appartienne. 
\item Suivent ensuite le nom du plan à modéliser, puis la largeur et la hauteur qu'il doit avoir, puis le nombre de subdivisions du plan dans ces deux sens. Plus il y a de subdivisions, plus il y a de polygones dans notre mesh. 
\item Le booléen suivant indique que les normales sont perpendiculaires au plan.
\item Les trois paramètres suivants sont le nombre de textures que l'on va pouvoir assigner au plan, puis le nombre de fois que la texture sera répétée dans les deux directions. 
\item le dernier paramètre est le vecteur indiquant la direction du haut du mesh. Attention: il ne faut pas le confondre avec la normale du plan, qui est différente.
\end{itemize}

Il reste encore des paramètres par défaut que l'on verra plus tard.

Enfin, nous allons revenir vers un code connu: nous allons créer l'entité qui représentera le plan. C'est le même principe que tout à l'heure:
\begin{itemize} 
\item tout d'abord, on crée une entité à partir du scène Manager en la nommant et en lui indiquant le mesh à utiliser. 
\item On crée ensuite un nouveau nœud à partir du nœud racine et on l'attache à notre entité.
\end{itemize}

\begin{lstlisting}
//creation d'une entite
Entity *ent= msceneMgr->createEntity(''EntiteSol'', ''sol'');

//creation d'un nouveau noeud
node = msceneMgr->getRootsceneNode()->createChildSceneNode();

//on attache le noeud a notre entite
node->attachObject(ent);
\end{lstlisting}




\subsection{Le matériau}

Nous allons finir en ajoutant une texture au sol: de l'herbe. Pour cela, il suffit de rajouter la ligne suivante après la création de l'entité:
\begin{lstlisting}
ent->setMaterialName(''Examples/GrassFloor'');
\end{lstlisting}

Si vous voulez connaître les matériaux fournis avec Ogre, il vous suffit d'aller dans le dossier media/materials/scripts. Ici, on prend le matériau GrassFloor enregistré dans le fichier Examples.material. 

Les textures correspondantes se trouvent dans le dossier media/materials/textures, si vous voulez faire des essais.


Vous pouvez maintenant exécuter votre programme.

Lancez l'application et remontez la caméra avec la souris et les touches de déplacement, vous devriez voir quelque chose ressemblant à la capture suivante.

Image utilisateur

Euh... La tête d'Ogre est coupée par le sol en herbe...

En effet, notre plan est centré sur l'origine de la scène, et l'on a aussi placé notre tête à l'origine. Mais quelle partie de la tête est à l'altitude 0 ?

Ici, c'est donc un point au milieu de la tête, puisque le plan d'herbe passe par là.
Cependant, ce point n'est pas nécessairement au milieu de l'objet que vous intégrez. Cela dépend de la personne qui a modélisé l'objet et qui a donc décidé par rapport à quel point on allait définir la position du mesh. Pour un personnage, on pourrait mettre ce point à ses pieds, pour que l'altitude 0 corresponde effectivement au moment où le personnage touche le sol avec ses pieds.

Pour corriger cela, il va falloir remonter notre nœud lié à notre entité. C'est l'objet du prochain chapitre.





\subsection{Code}

\subsubsection{PremiereApplication.cpp}
\begin{lstlisting}[caption={PremiereApplication.cpp: Création d'un sol}]
#include "PremiereApplication.h"

void PremiereApplication::createScene()
{
    //creation d une entite
    Entity *head= mSceneMgr->createEntity("Tete", "ogrehead.mesh" );
    
    //creation d un noeud
    SceneNode *node= mSceneMgr->getRootSceneNode( )->createChildSceneNode( "nodeTete " , Vector3::ZERO, Quaternion::IDENTITY);
    
    node->yaw(Radian(Math::PI));
    node->yaw(Radian(Math::PI));

    //setPosition place le noeud aux coord passees en parametres
    Vector3 position = Vector3(30.0, 50.0, 0.0);
    node->setPosition(position);

    node->setPosition(30.0, 50.0, 0.0); 
    /*equivalent a
    Vector3 position = Vector3(30.0, 50.0, 0.0);
    node->setPosition(position);
    */

    //deplace le noeud par rapport a sa position actuelle
    node->translate(-30.0, 50.0, 0.0); //par defaut la trnslt se fait par rap a TS_WORLD
   
    //attachement de l entite au noeud
    node->attachObject ( head );

    //creation d un plan
    Plane plan(Vector3::UNIT_Y, 0);

    //creation d un mesh cad l objet 3d visible ds la scene
    MeshManager::getSingleton().createPlane("sol",
                ResourceGroupManager::DEFAULT_RESOURCE_GROUP_NAME,
                plan, 500, 500, 1, 1, true, 1, 1, 1, Vector3::UNIT_Z); 

    //entite qui representera le plan
    Entity *ent= mSceneMgr->createEntity("EntiteSol", "sol");

    //ajout du materiau a l entite
    ent->setMaterialName("Examples/GrassFloor");//texture de pelouse
    /*les differents materiaux sont sous /media/materials/scritps, par ex:
    ent->setMaterialName("Examples/WaterStream");//texture d eau animee*/

    //creation d un noeud
    node = mSceneMgr->getRootSceneNode()->createChildSceneNode();
    node->attachObject(ent);
}


\end{lstlisting}

À cause de la position de la caméra, il se peut alors que le sol ne soit pas visible.

