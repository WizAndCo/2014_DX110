
%\chapter{LaTeX}
\chapter{Latex}

\section{V\'erifier sa configuration latex}
Lors d'une importation de ce document sur un autre pc que celui utilis\'e pour \'ecrire ce document des probl\`emes peuvent appara\^itre lors de la compilation de ce fichier tex.\newline

Pour v\'erifier que aucun package latex ne manque cr\'eer un fichier tex et y copier/coller le code suivant. Ce code reprend tous les packages utilis\'es pour l'\'ecriture de ce pr\'esent document (prise de notes \'a partir du tutorial du site des z\'ero sur Ogre)










%----------------
\begin{lstlisting}[caption={Code Latex minimal pour tester les packages latex n\'ecessaires \'a la compilation du fichier ****-ergo.tex par texmaker}]

\documentclass[10pt,a4paper]{report}
\usepackage[latin1]{inputenc}
\usepackage{amsmath}
\usepackage{amsfonts}
\usepackage{amssymb}
\usepackage{hyperref} %pr inserer des liens internet

\usepackage{verbatim}%pour linsertion brute de commande LaTeX dans le texte
\usepackage{moreverb}

\usepackage[french]{babel}
\usepackage[latin1]{inputenc}

\usepackage{listings}
\usepackage{xcolor}

\usepackage{makeidx}

\title{OGRE}
\author{O}

\begin{document}
This is a MINIMUM WORKING EXAMPLE. hgf\newline

\'e
\`e
\^e
\newline

\'A
\`A
\^S
\end{document}

\end{lstlisting}




















%------------------------



\section{Caract\`eres accentu\'es}

Pour faire un \`{u}
\begin{verbatim}
\`{u}
\end{verbatim}


Pour faire un \^{a}
\begin{verbatim}
\^{a}
\end{verbatim}

Pour faire un \`{a}
\begin{verbatim}
\`{a} ou \`a
\end{verbatim}


Pour faire un \"o
\begin{verbatim}
\''{o} ou \"o
\end{verbatim}

Pour faire un \^{i}
\begin{verbatim}
\^{i}
\end{verbatim}

Pour faire un \`e
\begin{verbatim}
\`e
\end{verbatim}

Pour faire un \^e
\begin{verbatim}
\^e
\end{verbatim}

Pour faire un \'{e}
\begin{verbatim}
\'{e} ou \'e
\end{verbatim}

Pour faire un \c{c}
\begin{verbatim}
\c{c}
\end{verbatim}

Pour faire un \.o
\begin{verbatim}
\.{o} ou \.o
\end{verbatim}


Pour faire un \~u
\begin{verbatim}
\~{u} ou \~u
\end{verbatim}

Pour faire un \=o
\begin{verbatim}
\={o} ou \=o
\end{verbatim}


Pour faire un \^u
\begin{verbatim}
\^u
\end{verbatim}

Pour faire des guillemets
\begin{verbatim}
''guillemets''
\end{verbatim}


\subsection{M\'ethode alternative (non test\'ee)}

Il semble possible d'ins\'erer directement tous les caract\`eres fran\c{c}ais, pour d\'emo tester le code suivant:

\begin{lstlisting}[caption={Insertion de caract\`eres fran\c{c}ais}][language=latex]
\documentclass[10pt,a4paper]{article}
\usepackage[utf8]{inputenc}
\usepackage[T1]{fontenc}
\begin{document}
entrer des caracteres accentues francais
guillemets
\end{document}

\end{lstlisting}

Le probl\`eme est que cel\`a a des cons\'equences sur le code d\'ej\`a \'ecrit.



\section{Notes}
\subsection{Note dans la marge}
Une note dans la marge\marginpar{ceci est une note
dans la marge}
\subsection{Note de bas de marge}
Une note de bas de page\footnote{Comme celle-ci.}.



\section{Liens hyperlien}
Deux m\'ethodes diff\'erentes:

\begin{itemize}
\item Le lien est ajout\' e de mani\`ere brute:

\begin{lstlisting}[caption={Insertion de liens Internet}][language=latex]
ceci est un lien brut \url{http://estcequecestbientotleweekend.fr/}
\end{lstlisting}

et cel\`a donne ceci:\newline
ceci est un lien brut
\url{http://estcequecestbientotleweekend.fr/}\newline
\end{itemize}



\begin{itemize}
\item Un mot m\`ene vers le lien, ci-dessous un clic sur ''lien'' m\`ene au lien spécifi\'e

\begin{lstlisting}[caption={Insertion de liens Internet}][language=latex]
ceci est un lien \href{http://estcequecestbientotleweekend.fr/}{lien}
\end{lstlisting}

et cel\`a donne ceci:\newline
ceci est un \href{http://estcequecestbientotleweekend.fr/}{lien}
\end{itemize}






\section{Bloc comment\'e}
Un bloc comment\'e se fait avec le package verbatim
\begin{verbatim}
\begin{comment}
	bloc comment\'e
\end{comment}
\end{verbatim}



\section{Num\'erotation des chapitres et autres}
Apparemment le fait d'\'ecrire 

\begin{verbatim}
\section
\end{verbatim}

fait que la section sera num\'erot\'ee, 

\begin{verbatim}
\section*
\end{verbatim}

ne sera pas num\'erot\'ee.




\begin{comment}
\section{Underscore}
Il faut penser \`{a} \'echapper les underscore sinon la compilation plante. On \'echappe avec 
\end{comment}


\section{Insertion d'images}
Pour ins\'erer des images la m\'ethode suivie est la suivante:

\begin{lstlisting}[caption={Insertion d'image}][language=latex]
\usepackage{graphicx}

\begin{document}
	\begin{center}
	\includegraphics[scale=0.5]{monimage.jpg} 
	\end{center}
\end{document}

\end{lstlisting}






\section{Exemple d'insertion de code avec lstlisting}
L'insertion de code se fait grace aux packages:

\begin{lstlisting}
\usepackage{listings} %pr inserer du code
\usepackage{xcolor}
\end{lstlisting}

On peut ensuite d\'efinir certain param\`etres d'insertion
\begin{lstlisting}[caption={Commande pour sp\'ecifier les param\`etres d'insertion de code}] [language=tex]
\lstset{
basicstyle=\small, % print whole listing small
keywordstyle=\color{blue}\bfseries\underbar,
% underlined bold black keywords
identifierstyle=, % nothing happens
commentstyle=\color{white}, % white comments
stringstyle=\ttfamily, % typewriter type for strings
showstringspaces=false} % no special string spaces
\end{lstlisting}


\begin{lstlisting}
for i:=maxint to 0 do
begin
{ do nothing }
end;
Write('Case insensitive ');
Write('Pascal keywords.');
\end{lstlisting}


cf \url{http://tex.stackexchange.com/questions/21106/adding-c-code-in-latex}


\section{Cr\'eer un index}
Pour cr\'eer un index:
cf \url{http://www.tuteurs.ens.fr/logiciels/latex/makeindex.html}

pour que l'index soit g\'er\'e par TexMaker:
cf \url{http://www.xm1math.net/doculatex/makeindex.html}




%bizarrement le \part ci-dessous creee une entree ds la structure du document presentee par TexMaker
\section{Cr\'eer une annexe}

Pour cr\'er une annexe il faut utiliser la commande
\begin{lstlisting}[caption={Cr\'eer une annexe}] [language=latex]
\appendix		
%\chapter{Test}	% une "Annexe A Test" sera creee
\chapter{test}	% sera creee "A.1 test"
\end{lstlisting}
