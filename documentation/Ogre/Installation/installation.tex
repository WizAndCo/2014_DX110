
%!!! 3OGRE3 n est pas centre dans le titre qui suit
\chapter{Installation, Premier programme}

\section{Premier pas}
\subsection{Sources}
J'ai d'abord voulu suivre le tutoriel du site du z\'ero
\url{http://fr.openclassrooms.com/informatique/cours/decouvrez-ogre-3d/}\newline

Mais ce tutoriel du site des z\'eros semble plus ax\'e visual studio, par contre un lien est donn\'e pour la compilation de projets sous Ubuntu \url{http://geenux.wordpress.com/2010/03/18/installation-de-ogre-1-7-et-compilation-avec-cmake-sous-ubuntu/}


\subsection{Installation}
Installation avec synaptic des packages suivants:
\begin{enumerate}
\item libogre-dev
\item ogre-samples
\item ogre-samples-data
\item libogre-1.7.4
\item libois-1.3.0
\item libois-dev 
\end{enumerate}






\section{Premiere compilation}

\subsection{Premier programme}

\begin{itemize}
\item Cr\'eation d'un r\'epertoire contenant les fichiers suivants:
\end{itemize}

\begin{lstlisting}
Hraesvelg:~/Documents/workspace/3D/OGRE/ExampleApplication$ ls -l
total 24
-rw-rw-r-- 1 olivier olivier 1109 fvr. 23 12:16 CMakeLists.txt
-rw-rw-r-- 1 olivier olivier  844 fvr. 23 12:02 helloworld.cpp
-rw-rw-r-- 1 olivier olivier 4894 fvr. 23 12:18 Makefile
-rw-r--r-- 1 olivier olivier  446 fvr. 23 12:20 plugins.cfg
-rw-r--r-- 1 olivier olivier 1391 fvr. 23 12:20 resources.cfg
\end{lstlisting}


\begin{itemize}
\item helloworld.cpp:
\end{itemize}

\begin{lstlisting}
#include ''ExampleApplication.h''
 
class TutorialApplication: public ExampleApplication
{
protected:
public:
    TutorialApplication()
    {
    }
 
    ~TutorialApplication()
    {
    }
protected:
    void createsc\`ene(void)
    {
    }
};
 
#if OGRE_PLATFORM == OGRE_PLATFORM_WIN32
#define WIN32_LEAN_AND_MEAN
#include ''windows.h''
 
INT WINAPI WinMain( HINSTANCE hInst, HINSTANCE, LPSTR strCmdLine, INT )
#else
int main(int argc, char **argv)
#endif
{
    // Create application object
    TutorialApplication app;
 
    try {
        app.go();
    } catch( Exception& e ) {
#if OGRE_PLATFORM == OGRE_PLATFORM_WIN32
        MessageBox( NULL, e.what(), ''An exception has occurred!'', MB_OK | MB_ICONERROR | MB_TASKMODAL);
#else
        fprintf(stderr, ''An exception has occurred: %s\n'',
                e.what());
#endif
    }
 
    return 0;
}
\end{lstlisting}



\begin{itemize}
\item CMakeLists.txt est une copie modifi\'ee du CMakeLists.txt donn\'ee dans le tutoriel de compilation sous Ubuntu. Les lignes commencant par \# \~ sont les lignes originales que j'ai du modifier.
\end{itemize}





%!!! CE QUI SUIT EST EN COM PARCE QUE LA COMPIL BUGGE DESSUS
\begin{lstlisting}
project(helloworld)
cmake_minimum_required(VERSION 2.6)

\# \~ set(CMAKE_MODULE_PATH ''/usr/lib/OGRE/cmake/'')
set(CMAKE_MODULE_PATH ''/usr/share/OGRE/cmake/modules'')

#set(CMAKE_CXX_FLAGS ''-Wall -W -Werror -ansi -pedantic -g'')

# Il s sagit du tutoriel d exemple, qui utilise quelques fichiers predefini de Ogre. Il faut indiquer a cmake o\'{u} se trouvent les includes en question
# \~ include_directories (''/usr/share/OGRE/Samples/Common/include/'')
include_directories (''/usr/share/OGRE-1.7.4/Samples/Common/include/'')

# Bien sur, pour compiler Ogre, il faut le chercher, et definir le repertoire contenant les includes.
find_package(OGRE REQUIRED)
include_directories (${OGRE_INCLUDE_DIRS})

# L'exemple depend aussi de OIS, une lib pour gerer la souris, clavier, joystick...
find_package(OIS REQUIRED)

# On definit les sources qu'on veut compiler
SET(SOURCES
helloworld.cpp)

# On les compile
add_executable (
helloworld ${SOURCES}
)

# Et pour finir, on lie l'executable avec les librairies que find_package nous a gentillement trouve.
target_link_libraries(helloworld ${OGRE_LIBRARY} ${OIS_LIBRARY})
%\end{lstlisting}


plugins.cfg et resources.cfg sont copi\'es de /usr/share/OGRE-1.7.4

\subsection{Compilation}
les commandes \`{a} faire sont:
\begin{itemize}
\item cmake.
\item make
\item./helloworld
\end{itemize}

\section{Premiere compilation sous CodeBlocks}
A tenter!!!!\newline
\url{http://fr.openclassrooms.com/informatique/cours/decouvrez-ogre-3d/configuration-d-un-projet-3}

\section{Premiere compilation sous kDevelop}
J'ai suivi\newline
\url{http://www.ogre3d.org/tikiwiki/tiki-index.php?page=Setting+Up+An+Application+-+KDevelop+-+Linux}.\newline
Je copie dans le r\'epertoire ogretest le r\'epertoire d'exemple Sample\_Water. Le processus foire.




\section{Code de base}
\subsection{Source}
\url{http://fr.openclassrooms.com/informatique/cours/decouvrez-ogre-3d/le-code-de-base-1}

Ce code de base est en fait le m\^eme que celui fait dans le chapitre Premiere Compilation, sauf qu'alors on avait que un seul fichier cpp, on rajoute i\c{c}i un fichier.h qui est inclus dans le fichier main.cpp.


